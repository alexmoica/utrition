\documentclass[12pt, titlepage]{article}

\usepackage{amsmath, mathtools}

\usepackage[round]{natbib}
\usepackage{amsfonts}
\usepackage{amssymb}
\usepackage{graphicx}
\usepackage{colortbl}
\usepackage{xr}
\usepackage{hyperref}
\usepackage{longtable}
\usepackage{xfrac}
\usepackage{tabularx}
\usepackage{float}
\usepackage{siunitx}
\usepackage{booktabs}
\usepackage{multirow}
\usepackage[section]{placeins}
\usepackage{caption}
\usepackage{fullpage}
\usepackage{graphicx}

\hypersetup{
bookmarks=true,     % show bookmarks bar?
colorlinks=true,       % false: boxed links; true: colored links
linkcolor=red,          % color of internal links (change box color with linkbordercolor)
citecolor=blue,      % color of links to bibliography
filecolor=magenta,  % color of file links
urlcolor=cyan          % color of external links
}

\usepackage{array}

\externaldocument{../../SRS/SRS}

%% Comments

\usepackage{color}

\newif\ifcomments\commentstrue %displays comments
%\newif\ifcomments\commentsfalse %so that comments do not display

\ifcomments
\newcommand{\authornote}[3]{\textcolor{#1}{[#3 ---#2]}}
\newcommand{\todo}[1]{\textcolor{red}{[TODO: #1]}}
\else
\newcommand{\authornote}[3]{}
\newcommand{\todo}[1]{}
\fi

\newcommand{\wss}[1]{\authornote{blue}{SS}{#1}} 
\newcommand{\plt}[1]{\authornote{magenta}{TPLT}{#1}} %For explanation of the template
\newcommand{\an}[1]{\authornote{cyan}{Author}{#1}}

%% Common Parts

\newcommand{\progname}{Software Engineering} % PUT YOUR PROGRAM NAME HERE
\newcommand{\authname}{Team 16, Durum Wheat Semolina
	\\ Alexander Moica
	\\ Yasmine Jolly
	\\ Jeffrey Wang
	\\ Jack Theriault
	\\ Catherine Chen
	\\ Justina Srebrnjak } % AUTHOR NAMES                 

\usepackage{hyperref}
    \hypersetup{colorlinks=true, linkcolor=blue, citecolor=blue, filecolor=blue,
                urlcolor=blue, unicode=false}
    \urlstyle{same}
                                


\begin{document}

\title{Module Interface Specification for \progname{}}

\author{\authname}

\date{\today}

\maketitle

\pagenumbering{roman}

\section{Revision History}

\begin{tabularx}{\textwidth}{p{3cm}p{2cm}X}
\toprule {\bf Date} & {\bf Version} & {\bf Notes}\\
\midrule
January 18, 2023 & 1.0 & Initial Document\\
March 7, 2023 & 1.1 & Added New Modules - VNV Report\\
April 3, 2023 & 1.2 & Final Document Revision\\
\bottomrule
\end{tabularx}

~\newpage

\section{Symbols, Abbreviations and Acronyms}

See SRS Documentation, \cite{SRS}, at \url{https://github.com/jeff-rey-wang/utrition/blob/3c91ed8d891c50d14bab9dd2f7ddcd5d3d465f56/docs/SRS/SRS.pdf}

\wss{Also add any additional symbols, abbreviations or acronyms}

\newpage

\tableofcontents

\newpage

\pagenumbering{arabic}

\section{Introduction}

The following document details the Module Interface Specifications for Utrition. Utrition is an application that will provide the nutritional facts for an inputted food item. Users can provide input through text, voice, or image. Utrition will also log past input food data for users to easily view their eating habits and nutritional intake.

Complementary documents include the System Requirement Specifications
and Module Guide.  The full documentation and implementation can be
found at \url{https://github.com/jeff-rey-wang/utrition}.  \wss{provide the url for your repo}

\section{Notation}

\wss{You should describe your notation.  You can use what is below as
  a starting point.}

The structure of the MIS for modules comes from \citet{HoffmanAndStrooper1995},
with the addition that template modules have been adapted from
\cite{GhezziEtAl2003}.  The mathematical notation comes from Chapter 3 of
\citet{HoffmanAndStrooper1995}.  For instance, the symbol := is used for a
multiple assignment statement and conditional rules follow the form $(c_1
\Rightarrow r_1 | c_2 \Rightarrow r_2 | ... | c_n \Rightarrow r_n )$.

The following table summarizes the primitive data types used by \progname. 

\begin{center}
\renewcommand{\arraystretch}{1.2}
\noindent 
\begin{tabular}{l l p{7.5cm}} 
\toprule 
\textbf{Data Type} & \textbf{Notation} & \textbf{Description}\\ 
\midrule
character & char & a single symbol or digit\\
integer & $\mathbb{Z}$ & a number without a fractional component in (-$\infty$, $\infty$) \\
natural number & $\mathbb{N}$ & a number without a fractional component in [1, $\infty$) \\
real & $\mathbb{R}$ & any number in (-$\infty$, $\infty$)\\
\bottomrule
\end{tabular} 
\end{center}

\noindent
The specification of \progname \ uses some derived data types: sequences, strings, and
tuples. Sequences are lists filled with elements of the same data type. Strings
are sequences of characters. Tuples contain a list of values, potentially of
different types. Utrition also uses user frontend events to signal some function executions. The type JSON is heavily used to transport data to be displayed in the application interface. In addition, \progname \ uses functions, which
are defined by the data types of their inputs and outputs. Local functions are
described by giving their type signature followed by their specification.

\section{Module Decomposition}

The following table is taken directly from the Module Guide document, \cite{MG}, for this project.

\begin{table}[H]
	\centering
	\begin{tabular}{p{0.3\textwidth} p{0.6\textwidth}}
		\toprule
		\textbf{Level 1} & \textbf{Level 2}\\
		\midrule
		
		{Hardware-Hiding Module} & N/A \\
		\midrule
		
		\multirow{7}{0.3\textwidth}{Behaviour-Hiding Module} 
		& Application Path Module\\
		& Home Page Module\\
		& Upload Page Module\\
		& Profile Page Module\\
		& BMI Page Module\\
		& Settings Page Module\\
		& Upload Container Module \\
		& Image Upload Module\\
		& Text Upload Module \\
		& Voice Upload Module \\
		& Navigation Bar Module\\ 
		\midrule
		
		\multirow{3}{0.3\textwidth}{Software Decision Module} & Input Pre-Processing Module\\
		& Training Dataset Module\\
		& Image Classification Module\\
		& Nutritional Data Retriever Module\\
		& Profile Data Calculation Module\\
		\bottomrule
		
	\end{tabular}
	\caption{Module Hierarchy}
	\label{TblMH}
\end{table}

~\newpage

\section{MIS of Application Path Module} \label{App}
\subsection{Module}
App
\subsection{Uses}
NavBar, Home, Upload, Profile, BMI, Settings
\subsection{Syntax}
\subsubsection{Exported Constants}
None
\subsubsection{Exported Access Programs}
\begin{center}
	\begin{tabular}{p{3cm} p{4cm} p{4cm} p{3cm}}
		
		\hline
		\textbf{Name} & \textbf{In} & \textbf{Out} & \textbf{Exceptions} \\
		\hline
		App & - & App & - \\
		\hline
	\end{tabular}
\end{center}
\subsection{Semantics}
\subsubsection{State Variables}
None
\subsubsection{Environment Variables}
$path$: String
\subsubsection{Assumptions}
Users will not try to purposefully edit the site path to a nonexistent page.
\subsubsection{Access Routine Semantics}
App():
\begin{itemize}
	\item transition: $path := "/"$
	\item output: $out := self$
	\item exception: None
\end{itemize}
\subsubsection{Local Functions}
None

\newpage

\section{MIS of Home Page Module} \label{HomePage}
\subsection{Module}
Home
\subsection{Uses}
N/A
\subsection{Syntax}
\subsubsection{Exported Types}
Home = ?
\subsubsection{Exported Access Programs}
\begin{center}
	\begin{tabular}{p{2cm} p{4cm} p{4cm} p{2cm}}
		\hline
		\textbf{Name} & \textbf{In} & \textbf{Out} & \textbf{Exceptions} \\
		\hline
		Home & - & Home & - \\
		\hline
	\end{tabular}
\end{center}
\subsection{Semantics}
\subsubsection{State Variables}
None
\subsubsection{Environment Variables}
None
\subsubsection{Assumptions}
None
\subsubsection{Access Routine Semantics}
\noindent Home():
\begin{itemize}
	\item transition: Page rendered with general information about Utrition
	\item output: $out := self$
	\item exception: None
\end{itemize}
\subsubsection{Local Functions}
None

\newpage

\section{MIS of Upload Page Module}
\subsection{Module}
Upload
\subsection{Uses}
UploadContainer
\subsection{Syntax}
\subsubsection{Exported Types}
Upload = ?
\subsubsection{Exported Access Programs}
\begin{center}
	\begin{tabular}{p{2cm} p{4cm} p{4cm} p{2cm}}
		\hline
		\textbf{Name} & \textbf{In} & \textbf{Out} & \textbf{Exceptions} \\
		\hline
		Upload & - & Upload & - \\
		changeDisplay & $\mathbb{N}$ & - & -\\
		\hline
	\end{tabular}
\end{center}
\subsection{Semantics}
\subsubsection{State Variables}
$currentUpload$ : $\mathbb{N}$
\subsubsection{Environment Variables}
None
\subsubsection{Assumptions}
None
\subsubsection{Access Routine Semantics}
\noindent Upload():
\begin{itemize}
	\item transition: Page rendered with component from UploadContainer
	\item output: $out := self$
	\item exception: None
\end{itemize}

\noindent changeDisplay(selected):
\begin{itemize}
	\item transition: $currentUpload :=$ selected
	\item exception: None
\end{itemize}
\subsubsection{Local Functions}
None

\newpage

\section{MIS of Profile Page Module} \label{ProfilePage}
\subsection{Module}
Profile
\subsection{Uses}
NurtritionalData
\subsection{Syntax}
\subsubsection{Exported Types}
Profile = ?
\subsubsection{Exported Access Programs}
\begin{center}
	\begin{tabular}{p{4.7cm} p{4cm} p{4cm} p{3cm}}
		\hline
		\textbf{Name} & \textbf{In} & \textbf{Out} & \textbf{Exceptions} \\
		\hline
		Profile & - & Profile & - \\
		handleExplanationClick & - & - & - \\
		getData & - & - & BadResponseError\\
		leftClickForward & - & - & - \\
		leftClickBackward & - & - & -\\
		rightClickForward & - & - & - \\
		rightClickBackward & - & - & -\\
		handleDeleteEntry & $\mathbb{N}$ & - & -\\
		handleCancelDelete & - & - & -\\
		handleConfirmDelete & - & - & -\\
		renderConfirmationDialog & - & HTML & -\\
		\hline
	\end{tabular}
\end{center}
\subsection{Semantics}
\subsubsection{State Variables}
$totalcal$ : JSON\\
$showConfirmationDialog$ : Boolean\\
$entryToDelete$ : $\mathbb{N}$\\
$showTable$ : Boolean\\
$isImageLoaded$ : Boolean
\subsubsection{Environment Variables}
None
\subsubsection{Assumptions}
None
\subsubsection{Access Routine Semantics}
\noindent Profile():
\begin{itemize}
	\item transition: Page rendered with information from NutritionalData
	\item output: $out := self$
	\item exception: None
\end{itemize}
\noindent handleExplanationClick():
\begin{itemize}
	\item transition: $showTable := \lnot showTable$
	\item exception: None
\end{itemize}
\noindent getData():
\begin{itemize}
	\item transition: Listen for response from backend, $totalcal := $ response	
	\item exception: ($responseData ==$ error) $\Rightarrow$ BadResponseError
\end{itemize}
\noindent leftClickForward():
\begin{itemize}
	\item transition: $totalcal.index := totalcal.index + 4$
	\item exception: None
\end{itemize}
\noindent leftClickBackward():
\begin{itemize}
	\item transition: $totalcal.index := totalcal.index - 4$
	\item exception: None
\end{itemize}
\noindent rightClickForward():
\begin{itemize}
	\item transition: $totalcal.right_index := totalcal.right_index + 7$
	\item exception: None
\end{itemize}
\noindent rightClickBackward():
\begin{itemize}
	\item transition: $totalcal.right_index := totalcal.right_index - 7$
	\item exception: None
\end{itemize}
\noindent handleDeleteEntry(entry):
\begin{itemize}
	\item transition: $entryToDelete :=$ entry, $showConfirmationDialog :=$ true
	\item exception: None
\end{itemize}
\noindent handleCancelDelete(entry):
\begin{itemize}
	\item transition: $entryToDelete :=$ null, $showConfirmationDialog :=$ false
\item exception: None
\end{itemize}
\noindent handleConfirmDelete():
\begin{itemize}
	\item transition: Send deleted index to backend, wait for response, 
	$totalcal :=$ response, $entryToDelete :=$ null, $showConfirmationDialog 
	:=$ false
	\item exception: ($responseData ==$ error) $\Rightarrow$ BadResponseError
\end{itemize}
\noindent renderConfirmationDialog():
\begin{itemize}
	\item transition: Confirmation Dialog is shown
	\item output: $out :=$ Confirmation Dialog HTML
	\item exception: None
\end{itemize}
\subsubsection{Local Functions}
None

\newpage

\section{MIS of BMI Page Module}
\subsection{Module}
BMI
\subsection{Uses}
N/A
\subsection{Syntax}
\subsubsection{Exported Types}
BMI = ?
\subsubsection{Exported Access Programs}
\begin{center}
	\begin{tabular}{p{4.8cm} p{4cm} p{4cm} p{4cm}}
		\hline
		\textbf{Name} & \textbf{In} & \textbf{Out} & \textbf{Exceptions} \\
		\hline
		BMI & - & BMI & - \\
		handleBirthSexChange & Event & - & -\\
		handleWeightChange & Event & - & -\\
		handleWeightUnitChange & Event & - & -\\
		handleHeightCmChange & Event & - & -\\
		handleHeightFeetChange & Event & - & -\\
		handleHeightInchesChange & Event & - & -\\
		handleHeightUnitChange & Event & - & -\\
		handleAgeChange & Event & - & -\\
		handleActivityLevelChange & Event & - & -\\
		getData & - & - & BadResponseError\\
		handleSubmit & Event & -&-\\ 
		\hline
	\end{tabular}
\end{center}
\subsection{Semantics}
\subsubsection{State Variables}
$birthSex: $ String \\
$weight: \mathbb{R}$ \\
$weightUnit: $ String \\
$heightCm: \mathbb{R} $ \\
$heightFeet: \mathbb{R}$  \\
$heightInches: \mathbb{R} $  \\
$heightUnit: $ String \\
$age: \mathbb{N}$ \\
$activityLevel: $ String \\
$errorMessage: $ String \\
\subsubsection{Environment Variables}
None
\subsubsection{Assumptions}
None
\subsubsection{Access Routine Semantics}
\noindent BMI():
\begin{itemize}
	\item transition: Page rendered with components relevant to editing user 
	settings
	\item output: $out := self$
	\item exception: None
\end{itemize}
\noindent handleBirthSexChange(e):
\begin{itemize}
	\item transition: $birthSex :=$ e.target.value
	\item exception: None
\end{itemize}
\noindent handleWeightChange(e):
\begin{itemize}
	\item transition: $weight :=$ e.target.value
	\item exception: None
\end{itemize}
\noindent handleWeightUnitChange(e):
\begin{itemize}
	\item transition: $weightUnit :=$ e.target.value
	\item exception: None
\end{itemize}
\noindent handleHeightCmChange(e):
\begin{itemize}
	\item transition: $heightCm :=$ e.target.value
	\item exception: None
\end{itemize}
\noindent handleHeightFeetChange(e):
\begin{itemize}
	\item transition: $heightFeet :=$ e.target.value
	\item exception: None
\end{itemize}
\noindent handleHeightInchesChange(e):
\begin{itemize}
	\item transition: $heightInches :=$ e.target.value
	\item exception: None
\end{itemize}
\noindent handleHeightunitChange(e):
\begin{itemize}
	\item transition: $heightUnit:=$ e.target.value
	\item exception: None
\end{itemize}
\noindent handleAgeChange(e):
\begin{itemize}
	\item transition: $age :=$ e.target.value
	\item exception: None
\end{itemize}
\noindent handleActivityLevelChange(e):
\begin{itemize}
	\item transition: $activityLevel :=$ e.target.value
	\item exception: None
\end{itemize}
\noindent getData():
\begin{itemize}
	\item transition: Send state variable values to backend, return to Settings 
	Page
	\item exception: ($responseData ==$ error) $\Rightarrow$ BadResponseError
\end{itemize}
\noindent handleSubmit(e):
\begin{itemize}
	\item transition: if state variable(s) empty: $errorMessage :=$ "Please 
	fill out all fields.", else: getData()
	\item exception: None
\end{itemize}

\subsubsection{Local Functions}
None

\newpage

\section{MIS of Settings Page Module}
\subsection{Module}
Settings
\subsection{Uses}
N/A
\subsection{Syntax}
\subsubsection{Exported Types}
Upload = ?
\subsubsection{Exported Access Programs}
\begin{center}
	\begin{tabular}{p{2cm} p{4cm} p{4cm} p{2cm}}
		\hline
		\textbf{Name} & \textbf{In} & \textbf{Out} & \textbf{Exceptions} \\
		\hline
		Upload & - & Upload & - \\
		\hline
	\end{tabular}
\end{center}
\subsection{Semantics}
\subsubsection{State Variables}
None
\subsubsection{Environment Variables}
None
\subsubsection{Assumptions}
None
\subsubsection{Access Routine Semantics}
\noindent Upload():
\begin{itemize}
	\item transition: Page rendered with components from ImageUpload, 
	TextUpload, and VoiceUpload
	\item output: $out := self$
	\item exception: None
\end{itemize}
\subsubsection{Local Functions}
None

\newpage

\section{MIS of Upload Container Module} \label{UploadContainer}
\subsection{Template Module}
UploadContainer
\subsection{Uses}
ImageUpload, TextUpload, VoiceUpload
\subsection{Syntax}
\subsubsection{Exported Types}
UploadContainer = ?
\subsubsection{Exported Access Programs}
\begin{center}
	\begin{tabular}{p{3cm} p{5cm} p{5cm} p{2cm}}
		\hline
		\textbf{Name} & \textbf{In} & \textbf{Out} & \textbf{Exceptions} \\
		\hline
		UploadContainer & $\mathbb{Z}$ & UploadContainer & - \\
		\hline
	\end{tabular}
\end{center}
\subsection{Semantics}
\subsubsection{State Variables}
None
\subsubsection{Environment Variables}
None
\subsubsection{Assumptions}
None
\subsubsection{Access Routine Semantics}
None
\subsubsection{Local Functions}
None

\newpage

\section{MIS of Image Upload Module} \label{ImageUpload} 
\subsection{Module}
ImageUpload
\subsection{Uses}
None
\subsection{Syntax}
\subsubsection{Exported Types}
ImageUpload
\subsubsection{Exported Access Programs}
\begin{center}
	\begin{tabular}{p{3cm} p{4cm} p{4cm} p{3cm}}
		\hline
		\textbf{Name} & \textbf{In} & \textbf{Out} & \textbf{Exceptions} \\
		\hline
		ImageUpload & - & ImageUpload &- \\
		handleImage & Event & - & - \\
		getData & - & - & BadResponseError \\
		\hline
	\end{tabular}
\end{center}
\subsubsection{State Variables}
$image$: String\\
$responseData$: JSON
\subsubsection{Environment Variables}
None
\subsubsection{Assumptions}
The input file is of an appropriate type and not empty. The backend of Utrition will always send a response.
\subsubsection{Access Routine Semantics}
ImageUpload():
\begin{itemize}
	\item transition: $image, responseData := "", ""$
	\item output: $out := self$
	\item exception: None
\end{itemize}
\noindent
handleImage(e):
\begin{itemize}
	\item transition: $image :=$ path of uploaded image via $setImage(e)$
	\item exception: None
\end{itemize}
\noindent
getData():
\begin{itemize}
	\item transition: send image path, then listen for a response from 
	backend\\ setResponseData(response)
	\item exception: ($responseData ==$ error) $\Rightarrow$ BadResponseError
\end{itemize}
\subsubsection{Local Functions}
\noindent setImage(s)
\begin{itemize}
	\item transition: $ image :=s $
	\item exception: None
\end{itemize}
\noindent setResponseData(r)
\begin{itemize}
	\item transition: $ responseData:=r $
	\item exception: None
\end{itemize}

\newpage

\section{MIS of Text Upload Module} \label{TextUpload}
\subsection{Module}
TextUpload
\subsection{Uses}
None
\subsection{Syntax}
\subsubsection{Exported Types}
TextUpload = ?
\subsubsection{Exported Access Programs}
\begin{center}
	\begin{tabular}{p{3cm} p{4cm} p{4cm} p{2cm}}
		\hline
		\textbf{Name} & \textbf{In} & \textbf{Out} & \textbf{Exceptions} \\
		\hline
		TextUpload & - & TextUpload & - \\
		handleFoodItem & Event & - & - \\
		getData & - & - & BadResponseError \\
		\hline
	\end{tabular}
\end{center}
\subsection{Semantics}
\subsubsection{State Variables}
$foodDesc$ : (String, $\mathbb{Z}$)\\
$responseData$: JSON
\subsubsection{Environment Variables}
None
\subsubsection{Assumptions}
None
\subsubsection{Access Routine Semantics}
\noindent TextUpload():
\begin{itemize}
	\item transition: $foodDesc, responseData := $ "", ""
	\item output: $out := self$ 
	\item exception: None
\end{itemize}
\noindent handleFoodItem(e)
\begin{itemize}
	\item transition: $foodDesc :=$ the contents of the text fields via 
	$setFoodDesc(e)$
	\item exception: None
\end{itemize}
\noindent getData():
\begin{itemize}
	\item transition: send food item, then listen for a response from 
	backend\\ setResponseData(response) \\ display nutritional output
	\item exception: ($responseData ==$ error) $\Rightarrow$ BadResponseError
\end{itemize}
\subsubsection{Local Functions}
\noindent setFoodDesc((foodName, servings))
\begin{itemize}
	\item transition: $ foodDesc := $ (foodName, servings)
	\item exception: None
\end{itemize}
\noindent setResponseData(r)
\begin{itemize}
	\item transition: $ responseData:=r $
	\item exception: None
\end{itemize}

\newpage

\section{MIS of Voice Upload Module} \label{VoiceUpload} 
\subsection{Module}
VoiceUpload
\subsection{Uses}
None
\subsection{Syntax}
\subsubsection{Exported Types}
VoiceUpload = ?
\subsubsection{Exported Access Programs}
\begin{center}
	\begin{tabular}{p{3cm} p{4cm} p{4cm} p{2cm}}
		\hline
		\textbf{Name} & \textbf{In} & \textbf{Out} & \textbf{Exceptions} \\
		\hline
		VoiceUpload & - & VoiceUpload & - \\
		handleVoiceInput & Event & - & - \\
		getData & - & - & BadResponseError \\
		\hline
	\end{tabular}
\end{center}
\subsection{Semantics}
\subsubsection{State Variables}
$ detectSpeech $: String\\
$responseData$: JSON
\subsubsection{Environment Variables}
None
\subsubsection{Assumptions}
None
\subsubsection{Access Routine Semantics}
\noindent VoiceUpload():
\begin{itemize}
	\item transition: $detectSpeech, responseData :=$ "", ""
	\item output: $out := self$ 
	\item exception: None
\end{itemize}
\noindent handleVoiceInput(e)
\begin{itemize}
	\item transition: $detectSpeech :=$ the detected speech input via 
	setDetectSpeech(e)
	\item exception: None
\end{itemize}
\noindent getData():
\begin{itemize}
	\item transition: send voice input, then listen for a response from 
	backend\\ setResponseData(response)\\ display nutritional output
	\item exception: ($responseData ==$ error) $\Rightarrow$ BadResponseError
\end{itemize}
\subsubsection{Local Functions}
\noindent setDetectSpeech(s)
\begin{itemize}
	\item transition: $ detectSpeech := $ s
	\item exception: None
\end{itemize}
\noindent setResponseData(r)
\begin{itemize}
	\item transition: $ responseData:=r $
	\item exception: None
\end{itemize}

\newpage

\section{MIS of Navigation Bar Module} \label{Navbar}
\wss{You can reference SRS labels, such as R\ref{R_Inputs}.}
\wss{It is also possible to use \LaTeX for hypperlinks to external documents.}
\subsection{Module}
NavBar
\subsection{Uses}
N/A
\subsection{Syntax}
\subsubsection{Exported Types}
NavBar = ?
\subsubsection{Exported Access Programs}
\begin{center}
	\begin{tabular}{p{3cm} p{4cm} p{4cm} p{3cm}}
		\hline
		\textbf{Name} & \textbf{In} & \textbf{Out} & \textbf{Exceptions} \\
		\hline
		NavBar & - & NavBar & - \\
		changePage & Event &  & - \\
		\hline
	\end{tabular}
\end{center}
\subsection{Semantics}
\subsubsection{State Variables}
None
\subsubsection{Environment Variables}
None
\subsubsection{Assumptions}
Users will not try to purposefully change the paths for each button.
\subsubsection{Access Routine Semantics}
\noindent NavBar():
\begin{itemize}
	\item output: $out := self$ 
	\item exception: None
\end{itemize}
\noindent changePage():
\begin{itemize}
	\item transition: $path :=$ "/" $\lor$ "/profile" $\lor$ "/upload" 
	\item exception: None
\end{itemize}
\subsubsection{Local Functions}
None

\newpage

\section{MIS of Input Pre-Processing Module} \label{InputPreProcess}
\subsection{Module}
InputPreProcess
\subsection{Uses}
ImageClassification
\subsection{Syntax}
\subsubsection{Exported Constants}
None
\subsubsection{Exported Access Programs}
\begin{center}
	\begin{tabular}{p{2cm} p{4cm} p{4cm} p{2cm}}
		\hline
		\textbf{Name} & \textbf{In} & \textbf{Out} & \textbf{Exceptions} \\
		\hline
		open & String & String & - \\
		\hline
	\end{tabular}
\end{center}
\subsection{Semantics}
\subsubsection{State Variables}
$filePath$: String\\
$foodIdentified$: String
\subsubsection{Environment Variables}
None
\subsubsection{Assumptions}
It is assumed that there exists a valid image file at the provided image file path.
\subsubsection{Access Routine Semantics}
\noindent open(path):
\begin{itemize}
	\item transition: $filePath := path$
	\item output: $out := foodIdentified$
	\item exception: None
\end{itemize}
\subsubsection{Local Functions}
None

\newpage

\section{MIS of Training Dataset Module} \label{TrainingDataset}
\subsection{Module}
TrainingDataset
\subsection{Uses}
N/A
\subsection{Syntax}
\subsubsection{Exported Constants}
None
\subsubsection{Exported Access Programs}
\begin{center}
	\begin{tabular}{p{2cm} p{5cm} p{5cm} p{2cm}}
		\hline
		\textbf{Name} & \textbf{In} & \textbf{Out} & \textbf{Exceptions} \\
		\hline
		loadData & seq of (seq of $\mathbb{Z}$), $\mathbb{Z}$ & Dictionary & - \\
		\hline
	\end{tabular}
\end{center}
\subsection{Semantics}
\subsubsection{State Variables}
$imageArray$: seq of (seq of $\mathbb{Z}$)\\
$flag$: $\mathbb{Z}$
\subsubsection{Environment Variables}
None
\subsubsection{Assumptions}
It is assumed the file path and file type of the CIFAR-100 datasets are respectively constant and standard.
\subsubsection{Access Routine Semantics}
\noindent loadData(array, f):
\begin{itemize}
	\item transition: $imageArray, flag := array, f$
	\item output: $out :=$ Dictionary consisting of image labels and classes used in the machine learning model
	\item exception: None
\end{itemize}
\subsubsection{Local Functions}
\begin{itemize}
	\item{unpickle(file): takes in a file path and opens it into bytestream. Specific dictionary entries are retrieved and returned depending on the filepath that was passed as an argument}
	\item{main(): used for debugging a single file. Calls loadData(None, None) and prints the resulting retrieved dictionary entries. }
\end{itemize}

\newpage

\section{MIS of Image Classification Module} \label{ImageClassification}
\subsection{Module}
ImageClassification
\subsection{Uses}
TrainingDataset
\subsection{Syntax}
\subsubsection{Exported Constants}
None
\subsubsection{Exported Access Programs}
\begin{center}
\begin{tabular}{p{2cm} p{4cm} p{4cm} p{2cm}}
\hline
\textbf{Name} & \textbf{In} & \textbf{Out} & \textbf{Exceptions} \\
\hline
startModel & seq of (seq of $\mathbb{Z}$) & String & - \\
\hline
\end{tabular}
\end{center}
\subsection{Semantics}
\subsubsection{State Variables}
$weights$: seq of (seq of $\mathbb{Z}$)\\
$imageArray$ : seq of (seq of $\mathbb{Z}$)\\
$foodItem$: String
\subsubsection{Environment Variables}
None
\subsubsection{Assumptions}
It is assumed that there is a relationship between the uploaded image and the image labels that the machine learning model is aware of. It is also assumed that the food in an uploaded image has a one to one relation with a label that the machine learning model is aware of.
\subsubsection{Access Routine Semantics}
\noindent startModel(array):
\begin{itemize}
	\item transition: $imageArray := array$
	\item output: $out := foodItem$
	\item exception: None
\end{itemize}
\subsubsection{Local Functions}
tf.compat.v1.train.GradientDescentOptimizer(learning\_rate).minimize(loss): Execute GradientDescentOptimizer and tries to minimize loss by computing the gradients of its trainable variables. Optimizes weights system variable on pass.

\newpage

\section{MIS of Nutritional Data Retriever Module} \label{NDR}
\subsection{Module}
NutritionalData
\subsection{Uses}
N/A
\subsection{Syntax}
\subsubsection{Exported Constants}
None
\subsubsection{Exported Access Programs}
\begin{center}
	\begin{tabular}{p{4cm} p{2cm} p{5cm} p{4.5cm}}
		\hline
		\textbf{Name} & \textbf{In} & \textbf{Out} & \textbf{Exceptions} \\
		\hline
		getNutritionalData & String & tuple of (food\_name: String, calories: String, total\_fat: String, saturated\_fat: String, cholesterol: String, sodium: String, total\_carbohydrate: String, dietary\_fiber: String, sugars: String, protein: String, potassium: String) & IllegalArgumentException \\
		\hline
	\end{tabular}
\end{center}
\subsection{Semantics}
\subsubsection{State Variables}
$result$: tuple of Strings
\subsubsection{Environment Variables}
None
\subsubsection{Assumptions}
None
\subsubsection{Access Routine Semantics}
\noindent getNutritionalData(\textit{food\_item}):
\begin{itemize}
	\item output: $result :=$ tuple of (food\_name: String, calories: String, 
	total\_fat: String, saturated\_fat: String, cholesterol: String, sodium: 
	String, total\_carbohydrate: String, dietary\_fiber: String, sugars: 
	String, protein: String, potassium: String)
	\item exception: ($food\_item \Rightarrow result$ := NULL) $\Rightarrow$ \mbox{IllegalArgumentException} 
\end{itemize}
\subsubsection{Local Functions}
None

\newpage

\section{MIS of Profile Data Calculation Module} \label{PDC}
\subsection{Module}
ProfileData
\subsection{Uses}
os, datetime
\subsection{Syntax}
\subsubsection{Exported Constants}
None
\subsubsection{Exported Access Programs}
\begin{center}
	\begin{tabular}{p{6.2cm} p{5cm} p{2.5cm} p{2cm}}
		\hline
		\textbf{Name} & \textbf{In} & \textbf{Out} & \textbf{Exceptions} \\
		\hline
		
		logData & JSON  & - & - \\
		calculateTotalNutrients & JSON  & seq of String & - \\
		readFile & -  & seq of JSON & - \\
		readFileAsJson & -  & JSON & - \\
		totalCaloriesPerDay & Date  & $\mathbb{R}$ & - \\
		totalFoodsPerDay & Date  & $\mathbb{Z}$ & - \\
		totalCaloriesPerDaySummaryList & -  & seq of JSON & - \\
		mostEatenFood & -  & $\mathbb{R}$ & - \\
		\hline
	\end{tabular}
\end{center}
\subsection{Semantics}
\subsubsection{State Variables}
None
\subsubsection{Environment Variables}
None
\subsubsection{Assumptions}
None
\subsubsection{Access Routine Semantics}
\noindent logData(foodData):
\begin{itemize}
	\item transition: $nutritionLog += csvRow : String$ 
	\item exception: None
\end{itemize}
\noindent calculateTotalNutrients(foodData):
\begin{itemize}
	\item output: $out :=$\{foodName : String, calories: $\mathbb{R}$, 
	totalFat: $\mathbb{R}$, saturatedFat: $\mathbb{R}$, cholesterol: 
	$\mathbb{R}$, sodium: $\mathbb{R}$, totalCarbohydrate: $\mathbb{R}$, 
	dietaryFibre: $\mathbb{R}$, sugars: $\mathbb{R}$, protein: $\mathbb{R}$, 
	potassium: $\mathbb{R}$\} 
	\item exception: None
\end{itemize}
\noindent readFile():
\begin{itemize}
	\item output: $out :=$ seq of Row: String 
	\item exception: None
\end{itemize}
\noindent readFileAsJson():
\begin{itemize}
	\item output: $out := $\{timeStamp: String, foodName : String, calories: 
	$\mathbb{R}$, 
	totalFat: $\mathbb{R}$, saturatedFat: $\mathbb{R}$, cholesterol: 
	$\mathbb{R}$, sodium: $\mathbb{R}$, totalCarbohydrate: $\mathbb{R}$, 
	dietaryFibre: $\mathbb{R}$, sugars: $\mathbb{R}$, protein: $\mathbb{R}$, 
	potassium: $\mathbb{R}$\} 
	\item exception: None
\end{itemize}
\noindent totalCaloriesPerDay(day):
\begin{itemize}
	\item output: $out := \sum_{i=0}^{|foods|-1}food_i.calories$ 
	\item exception: None
\end{itemize}
\noindent totalFoodsPerDay(day):
\begin{itemize}
	\item output: $out := \sum_{i=1}^{|foods|} 1$ 
	\item exception: None
\end{itemize}
\noindent totalCaloriesPerDaySummaryList():
\begin{itemize}
	\item output: $out := \text{seq of $\{data, sumPerDay, foodsPerDay\}$}$ 
	\item exception: None
\end{itemize}
\noindent mostEatenFood():
\begin{itemize}
	\item output: $out := mode(\text{seq of foods})$ 
	\item exception: None
\end{itemize}

\wss{A module without environment variables or state variables is unlikely to
	have a state transition.  In this case a state transition can only occur if
	the module is changing the state of another module.}

\wss{Modules rarely have both a transition and an output.  In most cases you
	will have one or the other.}

\subsubsection{Local Functions}

None
\wss{As appropriate} \wss{These functions are for the purpose of specification.
	They are not necessarily something that is going to be implemented
	explicitly.  Even if they are implemented, they are not exported; they only
	have local scope.}

\newpage

\bibliographystyle {plainnat}
\bibliography {../../../refs/References}

\newpage

\section{Appendix} \label{Appendix}

\wss{Extra information if required}

\end{document}
