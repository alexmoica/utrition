\documentclass{article}

\usepackage{booktabs}
\usepackage{tabularx}

\title{Development Plan\\\progname}

\author{\authname}

\date{}

%% Comments

\usepackage{color}

\newif\ifcomments\commentstrue %displays comments
%\newif\ifcomments\commentsfalse %so that comments do not display

\ifcomments
\newcommand{\authornote}[3]{\textcolor{#1}{[#3 ---#2]}}
\newcommand{\todo}[1]{\textcolor{red}{[TODO: #1]}}
\else
\newcommand{\authornote}[3]{}
\newcommand{\todo}[1]{}
\fi

\newcommand{\wss}[1]{\authornote{blue}{SS}{#1}} 
\newcommand{\plt}[1]{\authornote{magenta}{TPLT}{#1}} %For explanation of the template
\newcommand{\an}[1]{\authornote{cyan}{Author}{#1}}

%% Common Parts

\newcommand{\progname}{Software Engineering} % PUT YOUR PROGRAM NAME HERE
\newcommand{\authname}{Team 16, Durum Wheat Semolina
	\\ Alexander Moica
	\\ Yasmine Jolly
	\\ Jeffrey Wang
	\\ Jack Theriault
	\\ Catherine Chen
	\\ Justina Srebrnjak } % AUTHOR NAMES                 

\usepackage{hyperref}
    \hypersetup{colorlinks=true, linkcolor=blue, citecolor=blue, filecolor=blue,
                urlcolor=blue, unicode=false}
    \urlstyle{same}
                                


\begin{document}

\begin{table}[hp]
\caption{Revision History} \label{TblRevisionHistory}
\begin{tabularx}{\textwidth}{llX}
\toprule
\textbf{Date} & \textbf{Developer(s)} & \textbf{Change}\\
\midrule
Date1 & Name(s) & Description of changes\\
Date2 & Name(s) & Description of changes\\
... & ... & ...\\
\bottomrule
\end{tabularx}
\end{table}

\newpage

\maketitle

The development plan details the foundation on how team Durum Wheat Semolina will develop the nutrition tracking app Utrition. This document outlines each team member’s roles and responsibilities, as well as the team’s workflow plan and how deadlines will be scheduled. The document describes the biggest challenges the team will face while developing Utrition and their respective solutions. Also, the document lists the coding standards and technology used in developing Utrition.

\section{Team Meeting Plan}

Team meetings will occur twice every week. These meetings will take place from 6:30-7:30 on Mondays and from 1:30-2:30 on Thursdays. The location for the meetings will be on the McMaster University campus, in a private study room that will be booked and notified to all group members 24 hours before the meeting. Catherine Chen is responsible for booking the meeting room. Any additional meetings will be planned in accordance with all members availablity when needed.

All six groups members are expected to attend every meeting. In the event that a group member plans to miss a meeting, they must notify all groups members beforehand and continue to complete their assigned work.

Meetings will follow the meeting agenda created by Yasmine Jolly. Any member can add to the agenda which can be found in a word document inside the team's Microsoft Teams channel. Yasmine will chair all meetings, ensuring all agenda items are discussed in order of highest priority.

\section{Team Communication Plan}

The team will communicate primarily through Facebook Messenger chat. All members have the application downloaded on their phones and will receive notifications when a new message is sent. This ensures replies will be sent in a timely fashion. In emergency cases where a member must get in contact with another member as soon as possible, they may call them on their personal cell phone number. Personal contact information has been shared between all members.

\section{Team Member Roles}

Utrition will not contain a team leader. Team members will have equal authority and influence on major decisions and plans for the project. All important decisions must be unanimously agreed upon by all team members.

Each team member is expected to contribute to both technical developement and written documentation work during this project. Tasks will be distributed throughout the span of the project where each group member is responsible for their assigned work. In addition to these responsibilities, specific roles have been assigned to each member to ensure organization of the project and high quality deliverables. These roles are discussed below. 

\subsection{Alexander Moica}

\begin{itemize}
	\item Backend Developer Lead: responsible for leading development of backend implementation. All needed work for product functionality must be identified and managed by this role. Questions regarding backend work will be directed to this role.
	\item Github Specialist: responsible for managing the project on Github including creating Git issues and managing tags. Additionally, any questions about Github functionality will be directed to this role.
\end{itemize}

\subsection{Yasmine Jolly}

\begin{itemize}
	\item Meeting Chair: responsible for creating meeting agendas and leading all meetings. This role ensures meetings stay on track with minimal distractions and all meeting items are discussed. 
	\item Scribe: responsible for documenting notable discussions and decisions that arise during group meetings.
\end{itemize}

\subsection{Jeffrey Wang}

\begin{itemize}
	\item Team Liaison: responsible for communicating questions and concerns on behalf of the team to the class instructor and/or teaching assistants. 
	\item LaTeX Specialist: responsible for answering all team internal questions related to LaTeX. 
\end{itemize}

\subsection{Jack Theriault}

\begin{itemize}
	\item Testing Lead: responsible for leading testing processes and ensuring other team members are creating complete test cases for their code. Final testing efforts will be done by this role.
\end{itemize}

\subsection{Catherine Chen}

\begin{itemize}
	\item Frontend Developer Lead: responsible for leading development of frontend implementation. All needed work for user interface implementation and design must be identified and managed by this role. Questions regarding frontend work will be directed to this role.
	\item Meeting Room Booker: responsible for booking private study rooms on McMaster University's campus for team meetings. This role will book a room and notify all team members 24 hours before the scheduled meeting.
\end{itemize}

\subsection{Justina Srebrnjak}

\begin{itemize}
	\item Documentation Proof Reader: responsible for reviewing all documentation. This includes ensuring the document is cohesive, correcting any grammar or spelling errors, and grading the document with provided rubics before submission.   
\end{itemize}

\section{Workflow Plan}

\begin{itemize}
	\item How will you be using git, including branches, pull request, etc.?
	\item How will you be managing issues, including template issues, issue
	classificaiton, etc.?
\end{itemize}

\section{Proof of Concept Demonstration Plan}

What is the main risk, or risks, for the success of your project?  What will you
demonstrate during your proof of concept demonstration to convince yourself that
you will be able to overcome this risk?

\section{Technology}

\begin{itemize}
\item Specific programming language
\item Specific linter tool (if appropriate)
\item Specific unit testing framework
\item Investigation of code coverage measuring tools
\item Specific plans for Continuous Integration (CI), or an explanation that CI
  is not being done
\item Specific performance measuring tools (like Valgrind), if
  appropriate
\item Libraries you will likely be using?
\item Tools you will likely be using?
\end{itemize}

\section{Coding Standard}

\section{Project Scheduling}

\wss{How will the project be scheduled?}

\end{document}