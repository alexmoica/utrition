\documentclass{article}

\usepackage{tabularx}
\usepackage{booktabs}

\title{Reflection Report on \progname}

\author{author name}

\date{}

%% Comments

\usepackage{color}

\newif\ifcomments\commentstrue %displays comments
%\newif\ifcomments\commentsfalse %so that comments do not display

\ifcomments
\newcommand{\authornote}[3]{\textcolor{#1}{[#3 ---#2]}}
\newcommand{\todo}[1]{\textcolor{red}{[TODO: #1]}}
\else
\newcommand{\authornote}[3]{}
\newcommand{\todo}[1]{}
\fi

\newcommand{\wss}[1]{\authornote{blue}{SS}{#1}} 
\newcommand{\plt}[1]{\authornote{magenta}{TPLT}{#1}} %For explanation of the template
\newcommand{\an}[1]{\authornote{cyan}{Author}{#1}}

%% Common Parts

\newcommand{\progname}{Software Engineering} % PUT YOUR PROGRAM NAME HERE
\newcommand{\authname}{Team 16, Durum Wheat Semolina
	\\ Alexander Moica
	\\ Yasmine Jolly
	\\ Jeffrey Wang
	\\ Jack Theriault
	\\ Catherine Chen
	\\ Justina Srebrnjak } % AUTHOR NAMES                 

\usepackage{hyperref}
    \hypersetup{colorlinks=true, linkcolor=blue, citecolor=blue, filecolor=blue,
                urlcolor=blue, unicode=false}
    \urlstyle{same}
                                


\begin{document}

\begin{table}[hp]
\caption{Revision History} \label{TblRevisionHistory}
\begin{tabularx}{\textwidth}{llX}
\toprule
\textbf{Date} & \textbf{Developer(s)} & \textbf{Change}\\
\midrule
Date1 & Name(s) & Description of changes\\
Date2 & Name(s) & Description of changes\\
... & ... & ...\\
\bottomrule
\end{tabularx}
\end{table}

\newpage

\maketitle

This document summarizes the overall experience of the creation of the Utrition application. 

\section{Project Overview}

\plt{Summarize the original project goals and requirements}

The idea behind Utrition was to create an application that was ideal for the student lifestyle. We wanted our users to be able understand how to live a healthy life style. In order to accomplish this Utrition would log all foods that the user has eaten. Our goal was to create a user friendly app so we developed multiple different manners of uploading the food such as a voice upload, a text upload and an image upload. Based on this logged information and additional personal physical statistics that the user would input, Utrition would total how many calories the user has eaten per day and tell them how many calories they would need to consume in order to maintain their weight/body mass index. In addition to this we had requirements of reflecting back all of the foods the user logged with a date stamp so they can keep track of what they have eaten everyday. All of these requirements combined to make the final Utrition product, a user friendly application designed for the busy life style.


\section{Key Accomplishments}

\plt{What went well?  This can be what went well with the documentation, the
  coding, the project management, etc.}

Various aspects of the development of the Utrition can be viewed as a success or a key accomplishment. The final design was derived from various levels of user testing. This user testing was key to our success behind Utrition as we were receiving constant feedback and making improvements to our application. These improvements were validated as we did additional testing after our revisions and saw improvements in the ratings of how satisfied our users were. In our Development plan we originally only had the image upload functionality available but after realizing that this would limit the user to very little foods we decided to expand our application to include voice and text upload as they were able to include many more food options. Additionally we had a very effective and organized workflow. As a team we were able to stay on top of deadlines because of our git project board. Using this tool we were able to evenly distribute work, look what work was allocated to which individual and were able to resolve merging conflicts very easily. 
\section{Key Problem Areas}

\plt{What went wrong?  This can be what went wrong with the documentation, the
  technology, the coding, time management, etc.}

  As our goal was to create a user friendly application a couple of changes were made in response to feedback from our instructor and our user testing. We initially planned for our application to have a heavy machine learning aspect therefore, in the first stages of Utrition, there was originally only the ability to upload an image. This was not ideal if we wanted to create a user friendly application as the artificial intelligence we had built was only trained to be able to correctly identify 5 foods. We had not done enough initial research and had to change our requirements quite a bit in the middle of the development process. In addition to that, we also had a couple of issues creating a meeting time that suited everyone so that no one would have to compromise personal time. We had originally started at a set meeting time weekly which was quite helpful but if we did not use that time-slot it got more difficult to find times to meet. 



\section{What Would you Do Differently Next Time}

Although the Utrition team is satisfied with the final product certain goals we originally had set out for Utrition were not able to be met because of the time frame and resources that we had set aside for it. In order to be economic the requirement we had set was that the logged data would be stored locally on the users endpoint. Given the opportunity of creating another application or to continue building Utrition, hosting the users data on an external database would be a great improvement to the application.

\end{document}