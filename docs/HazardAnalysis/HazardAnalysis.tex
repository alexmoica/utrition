\documentclass{article}

\usepackage{booktabs}
\usepackage{tabularx}
\usepackage{hyperref}
\usepackage{float}
\usepackage{multirow}
\usepackage{adjustbox}
\usepackage{graphicx}
\usepackage{pdflscape}
\usepackage[shortlabels]{enumitem}
\usepackage{array}
\usepackage{fancyhdr}
\usepackage{caption}
\usepackage[margin=1in, includeheadfoot]{geometry}

\newcolumntype{C}[1]{>{\centering\arraybackslash}p{#1}}

\fancyhf{} % clear all header and footers
\renewcommand{\headrulewidth}{0pt} % remove the header rule
\lfoot{\thepage}

\hypersetup{
	colorlinks=true,       % false: boxed links; true: colored links
	linkcolor=red,          % color of internal links (change box color with 
	%linkbordercolor)
	citecolor=green,        % color of links to bibliography
	filecolor=magenta,      % color of file links
	urlcolor=cyan           % color of external links
}

\title{Hazard Analysis\\\progname}

\author{\authname}

\date{}

%% Comments

\usepackage{color}

\newif\ifcomments\commentstrue %displays comments
%\newif\ifcomments\commentsfalse %so that comments do not display

\ifcomments
\newcommand{\authornote}[3]{\textcolor{#1}{[#3 ---#2]}}
\newcommand{\todo}[1]{\textcolor{red}{[TODO: #1]}}
\else
\newcommand{\authornote}[3]{}
\newcommand{\todo}[1]{}
\fi

\newcommand{\wss}[1]{\authornote{blue}{SS}{#1}} 
\newcommand{\plt}[1]{\authornote{magenta}{TPLT}{#1}} %For explanation of the template
\newcommand{\an}[1]{\authornote{cyan}{Author}{#1}}

%% Common Parts

\newcommand{\progname}{Software Engineering} % PUT YOUR PROGRAM NAME HERE
\newcommand{\authname}{Team 16, Durum Wheat Semolina
	\\ Alexander Moica
	\\ Yasmine Jolly
	\\ Jeffrey Wang
	\\ Jack Theriault
	\\ Catherine Chen
	\\ Justina Srebrnjak } % AUTHOR NAMES                 

\usepackage{hyperref}
    \hypersetup{colorlinks=true, linkcolor=blue, citecolor=blue, filecolor=blue,
                urlcolor=blue, unicode=false}
    \urlstyle{same}
                                


\pagestyle{fancy}

\begin{document}

\maketitle
\thispagestyle{empty}

~\newpage

\pagenumbering{roman}

\begin{table}[hp]
	\caption{Revision History} \label{TblRevisionHistory}
	\begin{tabularx}{\textwidth}{llX}
		\toprule
		\textbf{Date} & \textbf{Developer(s)} & \textbf{Change}\\
		\midrule
		\today & All & Initial Version\\
		\bottomrule
	\end{tabularx}
\end{table}

~\newpage

\tableofcontents
\listoftables

~\newpage

\pagenumbering{arabic}

\wss{You are free to modify this template.}

\section{Introduction}

This document discusses the hazards associated with Utrition. In the context 
of Utrition, hazards are defined to be a set of circumstances that prevent 
the expected use of the system, leading to an error state. The document will 
communicate the scope, boundaries and assumptions made when completing the 
hazard analysis, and provide a list of identified hazards. In addition, it will 
mention recommended actions to mitigate and circumvent hazards encountered 
while using the system.

\section{Scope and Purpose of Hazard Analysis}
The purpose of conducting a hazard analysis is to document the system conditions, that along with conditions in the environment, can cause harm or damage. Documentation for how to control or mitigate these conditions will also be included. The scope of this hazard analysis will span from the time the user inputs an image into Utrition to the time when the nutritional data results of their image are displayed, including the pre-processing, processing, API request, and data representation steps.

\section{System Boundaries and Components}
The system boundaries for this hazard analysis will include the device that the application is installed on as well as the components of the application itself. These components consist of image upload, image pre-processing, image processing \& identification, API request calling, data logging, data log access, and data display components.

\subsection{Image Upload}
This component allows an image to be uploaded by the user and relayed to the pre-processing component.

\subsection{Image Pre-Processing}
This component takes an uploaded image and applies the algorithms needed to convert the raw image data into a format that can be used by a machine learning image model.

\subsection{Image Processing \& Identification}
This component is where the machine learning model analyzes the pre-processed image to identify the food displayed by comparing it to the images it was exposed to during its supervised learning. 

\subsection{API Request Calling}
This component allows the application to interface with the Nutritionix API to access nutritional data on a given food.  

\subsection{Data Logging}
This component logs past uses of the application by the identified food and the date it was consumed.   

\subsection{Data Log Access}
This component returns the recorded logs of past uses of the application. These logs include past food items consumed by the user.

\subsection{Data Display}
This component displays data visually for the user to see, either in textual or graphical formats.

\section{Critical Assumptions}

\wss{These assumptions that are made about the software or system.  You 
should
	minimize the number of assumptions that remove potential hazards.  For 
	instance,
	you could assume a part will never fail, but it is generally better to 
	include
	this potential failure mode.}

In this Hazard Analysis, two assumptions are made. Firstly, the user will not be intentionally trying to cause errors in Utrition. Secondly, the user is assumed to have sufficient storage space to download and run the application.

\section{Failure Mode and Effect Analysis}

\begin{landscape}

\begin{table}[ht]
	\centering

	\makebox[\linewidth]{
	\begin{tabular}{|p{2cm}|p{3cm}|p{3cm}|p{3cm}|p{3cm}|p{3cm}|p{1cm}|p{1cm}|}
		\multicolumn{8}{c}{}\\
		\hline
		\multicolumn{8}{|c|}{\textbf{Failure Mode and Effects Analysis}}\\
		\multicolumn{8}{|l|}{System: Utrition}\\
		\multicolumn{8}{|l|}{Subsystem: N/A}\\
		\multicolumn{8}{|l|}{Phase/Mode: System Requirements}\\\hline
		\textbf{Design Function} & \textbf{Failure Modes} & \textbf{Causes of Failure} & \textbf{Effects of Failure} & \textbf{Detection} & \textbf{Recommended Actions} & \textbf{SR} & \textbf{Ref} \\ \hline
		Image Upload & Image of incorrect type inputted & Users attempt to upload a file of an unsupported type & No image uploaded & Upload error will occur & Provide error message that informs the user that only file types of type .png, .jpg, and .jpeg can be uploaded & SR1 & H1-1 \\ \cline{2-8}
		& Image size inputted is too large & Image file from user is too large to be uploaded and stored & Same as H1-1 & Same as H1-1 & Provide error message that inputted file is too large & SR2 & H1-2\\ \cline{2-8}
		& User tries to upload more than 3 images at once & User attempts to upload more than 3 images & Same as H1-1 & Same as H1-1 & Provide error message saying up to 3 photos can be uploaded at once & SR3 & H1-3\\ \hline
		Image Processing \& Identification & Food from image is incorrectly identified & a. Poor image quality & a. System will process the incorrectly identified food item & a. User will file a report to the development team with the image that was incorrectly identified & a. If the machine learning model is not confident in result, the system will suggest the user upload another image & SR4 & \multirow{3}{*} {H2-1} \\ 
		& & b. Machine learning model has not been trained to identify inputted food item & b. Same as H2-1a & b. Same as H2-1a & b. Same as H2-1a & & \\
		& & c. Machine learning model accuracy is low & c. Same as H2-1 & c. Same as H2-1a & c. Same as H2-1 & & \\ \cline{2-8}
		& No food is identified in the image & Same as H2-1 & No food item will 
		be identified and the system will not be able to proceed & The food 
		identification machine learning process will return an error to the 
		system & Display error message detailing that system could not identify 
		a food item in the uploaded image & SR4 & H2-2\\ \hline

	\end{tabular}
}

	\caption{FMEA Table Part 1}
	\label{FMEAPart1}
\end{table}
\end{landscape}

\begin{landscape}

	\begin{table}[ht]
		\centering
		
		\makebox[\linewidth]{
			\begin{tabular}{|p{2cm}|p{3cm}|p{3cm}|p{3cm}|p{3cm}|p{3cm}|p{1cm}|p{1cm}|}
				\multicolumn{8}{c}{}\\
				\hline
				\multicolumn{8}{|c|}{\textbf{Failure Mode and Effects Analysis}}\\
				\multicolumn{8}{|l|}{System: Utrition}\\
				\multicolumn{8}{|l|}{Subsystem: N/A}\\
				\multicolumn{8}{|l|}{Phase/Mode: System Requirements}\\\hline
				\textbf{Design Function} & \textbf{Failure Modes}& \textbf{Causes of Failure} & \textbf{Effects of Failure} & \textbf{Detection} & \textbf{Recommended Actions} & \textbf{SR} & \textbf{Ref} \\ \hline
				API Request Calling & API call fails unexpectedly & a. Internet connection error & a. System will not return nutritional data for a food item & a. The API response will be verified by the system. The system will detect if the response is an error & a. Display error message detailing a system error due to poor internet connection & SR5 & \multirow{2}{*} {H3-1} \\ 
				& & b. Too many requests sent to API causing throttling limit to be reached & b. Same as H3-1a & b. Same as H3-1a & b. Display error message detailing a system error due to too many requests being sent & & \\ \cline{2-8}
				& API does not contain nutrition facts for a food item & Food 
				item is not found in API database & Same as H3-1a & The API 
				response will be verified by the system and will detect if the 
				response is empty & Display error message detailing that the 
				food data could not be fetched & SR6 & H3-2\\ \hline
				Data Log Access & Nutritional data is unavailable & Nutritional 
				data has not been put in the database for that particular input 
				yet & System will not output any nutritional information  & 
				Database will return no result for past nutritional data & 
				Display error message explaining that this nutritional data has 
				not yet been added to the database & SR7 & H4-1 \\ \cline{2-8}
				& Nutritional data is deleted unintentionally & Accidental 
				error of database deleting necessary data & Same as H4-1 & Same 
				as H4-1 & Display error message explaining that this 
				nutritional data has been deleted & SR8 & H4-2\\ \cline{2-8}
				& User past data is deleted & a. The data was not successfully 
				stored within the data base & a. System will not output any of 
				the user's information & a. Nothing will be output after user 
				requests past data & a. Display error message saying that the 
				user's past data has been deleted & SR8 & H4-3\\ 
				& & b. The user's past data was deleted unintentionally & b. Same as H4-3a & b. Same as H4-3a & b. Same as H4-3a & &\\\hline
			\end{tabular}
		}
		
		\caption{FMEA Table Part 2}
		\label{FMEAPart2}
	\end{table}
\end{landscape}

\begin{landscape}

	\begin{table}[ht]
		\centering
		
		\makebox[\linewidth]{
			\begin{tabular}{|p{2cm}|p{3cm}|p{3cm}|p{3cm}|p{3cm}|p{3cm}|p{1cm}|p{1cm}|}
				\multicolumn{8}{c}{}\\
				\hline
				\multicolumn{8}{|c|}{\textbf{Failure Mode and Effects Analysis}}\\
				\multicolumn{8}{|l|}{System: Utrition}\\
				\multicolumn{8}{|l|}{Subsystem: N/A}\\
				\multicolumn{8}{|l|}{Phase/Mode: System Requirements}\\\hline
				\textbf{Design Function} & \textbf{Failure Modes}  & \textbf{Causes of Failure} & \textbf{Effects of Failure} & \textbf{Detection} & \textbf{Recommended Actions} & \textbf{SR} & \textbf{Ref} \\ \hline
				Data Display & Graph cannot be generated  & Not enough past 
				information from user  & No graph is displayed to the user & 
				There is no information available in user's past data & Display 
				error message stating there is not enough data to create the 
				graph & SR9 & H5-1 \\ \hline
				General System & Device loses internet connection & a. Internet 
				connection used by device is too weak & a. Unable to access 
				nutrition facts for food items & a. API calls will fail & a. 
				Display error message that informs the user that they must be 
				connected to an internet connection to use the system & SR10 & 
				H6-1 \\
				& & b. Internet shutdown on connected network & b. Same as H6-1a & b. Same as H6-1a & b. Same as H6-1a & &\\ \cline{2-8}
				& System closes unexpectedly & a. Host device shuts down (loses 
				power) & a. Loss of recently inputted data & a. Device screen 
				will turn black & a. System should save data with each new 
				input to minimize lost data & SR11 & \multirow{2}{*}{H6-2}\\
				& & b. Internal error occurs & b. Same as H6-2a & b. Application will become unresponsive & b. Same as H6-2a & &\\ \hline
				
			\end{tabular}
		}
		
		\caption{FMEA Table Part 3}
		\label{FMEAPart3}
	\end{table}
\end{landscape}

\section{Safety and Security Requirements}
\subsection{Safety Requirements}
\begin{enumerate}[{SR}1.]
	\item Utrition will return an error message when the user uploads an abnormal 
	image format that is not .png, .jpg, or .jpeg. \\
	\textbf{Rationale:}  Utrition should not crash by improper user 
	input. Users should have an opportunity to upload a new file of an appropriate format.\\	
	\textbf{Associated Hazards:} H1-1.
	
	\item Utrition will return an error message when the user uploads an image file that exceeds the maximum size. \\
	\textbf{Rationale:}  Utrition should not crash by improper user 
	input. Users should have an opportunity to upload a new file of an appropriate size.\\	
	\textbf{Associated Hazards:} H1-2.
	
	\item Utrition will return an error message when the user uploads more than three images at once. \\
	\textbf{Rationale:}  Utrition should not crash by improper user 
	input. Users should have an opportunity to upload three or fewer images to the system.\\	
	\textbf{Associated Hazards:} H1-3.
	
	\item Utrition will prompt the user if food identification cannot 
	be completed successfully. The user will be notified on the type of error 
	that occurs. \\
	\textbf{Rationale:}  Food identification may fail due to a variety of 
	reasons, and the user should be notified so they may attempt to find a  
	workaround for the issue. \\	
	\textbf{Associated Hazards:} H2-1. H2-2.
	
	\item Utrition will return an error message if the request to retrieve 
	nutritional information cannot be completed successfully. \\
	\textbf{Rationale:}  Information retrieval requests may fail due to a 
	variety of reasons, and the user should be notified of the reason why the 
	service could not be completed as expected.\\	
	\textbf{Associated Hazards:} H3-1.
	
	\item Utrition will return an error message if the nutritional information 
	of a specific item cannot be found.\\
	\textbf{Rationale:}  The user should be notified if the nutritional data of 
	their food item cannot be fetched.\\
	\textbf{Associated Hazards:} H3-2.
	
	\item Utrition will return an error message if the user's part nutritional 
	logs cannot be found.\\
	\textbf{Rationale:}  The user should be notified if their nutritional data 
	of past meals cannot be found.\\
	\textbf{Associated Hazards:} H4-1.
	
	\item Utrition will return an error message if the user's part nutritional 
	logs have been deleted.\\
	\textbf{Rationale:}  The user should be notified if their nutritional data 
	of past meals are no longer saved in the system.\\
	\textbf{Associated Hazards:} H4-2, H4-3.
	
	\item Utrition will prompt the user if past nutritional trends cannot
	be displayed successfully. The user will be notified on the type of error 
	that occurs. \\
	\textbf{Rationale:}  Failure to display past nutritional trends may fail 
	due to a variety of reasons. The user should be made aware of the issue, 
	and the underlying cause behind it.\\	
	\textbf{Associated Hazards:} H5-1.
	
	\item Utrition will prompt the user if their device is not connected to the 
	internet when attempting to access the system. \\
	\textbf{Rationale:}  The user should be notified if they are unable to 
	connect to the system so they may apply a fix to the issue. \\	
	\textbf{Associated Hazards:} H6-1.
\end{enumerate}

\subsection{Security Requirements}
\begin{enumerate}[{SR}1.] 
	\setcounter{enumi}{10}
	
	\item Utrition will periodically save user’s data during use. \\
	\textbf{Rationale:} In the event of unexpected shutdown, the user should 
	not lose all information from the last session. Periodically saving user 
	information will allow users to continue from their last step in the event 
	of an unexpected shutdown. \\	
	\textbf{Associated Hazards:} H6-2.
\end{enumerate}

\section{Roadmap}

\wss{Which safety requirements will be implemented as part of the capstone 
timeline?
	Which requirements will be implemented in the future?}

Durum Wheat Semolina is planning to implement SR1-SR10 during Utrition's capstone timeline. The safety requirements (SR1-SR10) are easy to implement and do not provide much strain to the system. They exist to guide the user through Utrition, and to aid Utrition in being an easy-to-understand application. SR11 is not required to allow the user to use Utrition. The cost of implementing and utilizing SR11 is high with the reward being low. Therefore, the implementation of this requirement, during Utrition's capstone timeline, cannot be prioritized.
	
\end{document}
