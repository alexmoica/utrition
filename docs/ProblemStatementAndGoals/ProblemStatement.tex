\documentclass{article}

\usepackage{tabularx}
\usepackage{booktabs}

\title{Problem Statement and Goals\\\progname}

\author{\authname}

\date{}

%% Comments

\usepackage{color}

\newif\ifcomments\commentstrue %displays comments
%\newif\ifcomments\commentsfalse %so that comments do not display

\ifcomments
\newcommand{\authornote}[3]{\textcolor{#1}{[#3 ---#2]}}
\newcommand{\todo}[1]{\textcolor{red}{[TODO: #1]}}
\else
\newcommand{\authornote}[3]{}
\newcommand{\todo}[1]{}
\fi

\newcommand{\wss}[1]{\authornote{blue}{SS}{#1}} 
\newcommand{\plt}[1]{\authornote{magenta}{TPLT}{#1}} %For explanation of the template
\newcommand{\an}[1]{\authornote{cyan}{Author}{#1}}

%% Common Parts

\newcommand{\progname}{Software Engineering} % PUT YOUR PROGRAM NAME HERE
\newcommand{\authname}{Team 16, Durum Wheat Semolina
	\\ Alexander Moica
	\\ Yasmine Jolly
	\\ Jeffrey Wang
	\\ Jack Theriault
	\\ Catherine Chen
	\\ Justina Srebrnjak } % AUTHOR NAMES                 

\usepackage{hyperref}
    \hypersetup{colorlinks=true, linkcolor=blue, citecolor=blue, filecolor=blue,
                urlcolor=blue, unicode=false}
    \urlstyle{same}
                                


\begin{document}

\maketitle

\begin{table}[hp]
\caption{Revision History} \label{TblRevisionHistory}
\begin{tabularx}{\textwidth}{llX}
\toprule
\textbf{Date} & \textbf{Developer(s)} & \textbf{Change}\\
\midrule
09/19/2022 & Catherine Chen, Yasmine Jolly, &Initial Document\\ 
&Jeffrey Wang, Jack Theriault, &\\
&Alex Moica, Justina Srebrnjak &\\
%Date2 & Name(s) & Description of changes\\
%... & ... & ...\\
\bottomrule
\end{tabularx}
\end{table}

\section{Problem Statement}

\wss{You should check your problem statement with the
\href{https://github.com/smiths/capTemplate/blob/main/docs/Checklists/ProbState-Checklist.pdf}
{problem statement checklist}.}
\wss{You can change the section headings, as long as you include the required information.}

\subsection{Problem}

\subsection{Inputs and Outputs}

\wss{Characterize the problem in terms of ``high level'' inputs and outputs.  
Use abstraction so that you can avoid details.}

\subsection{Stakeholders}

\subsection{Environment}

\wss{Hardware and software}

\section{Goals}

\section{Stretch Goals}

\wss{This section contains project goals that are not integral to the development of Utrition, but would be valuable extensions to develop once the aforementioned goals are met.}

\subsection{Web Hosting}

\wss{Users will be able to access the Utrition application online through a website hosted on the internet. This will increase the user base of Utrition by making Utrition more accessible and convenient to use. Once hosted online, the delay between passing an input to Utrition and receiving an output will be similar in length to the local implementation of Utrition within a small margin of error. }

\subsection{Logging Preset Meals}

\wss{Users will be able to create presets of their commonly eaten meals on Utrition and save them for future use. Once a preset is created, a user will be able to select it to log the information of the meal without needing to manually re-upload images. This feature will allow users to customize Utrition to fit their eating habits, while also reducing the amount of time it takes to log a meal and access its information.  }

\end{document}
