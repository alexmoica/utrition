\documentclass[12pt, titlepage]{article}

\usepackage{booktabs}
\usepackage{tabularx}
\usepackage{hyperref}
\usepackage{float}
\hypersetup{
	colorlinks,
	citecolor=blue,
	filecolor=black,
	linkcolor=red,
	urlcolor=blue
}
\usepackage[round]{natbib}

%% Comments

\usepackage{color}

\newif\ifcomments\commentstrue %displays comments
%\newif\ifcomments\commentsfalse %so that comments do not display

\ifcomments
\newcommand{\authornote}[3]{\textcolor{#1}{[#3 ---#2]}}
\newcommand{\todo}[1]{\textcolor{red}{[TODO: #1]}}
\else
\newcommand{\authornote}[3]{}
\newcommand{\todo}[1]{}
\fi

\newcommand{\wss}[1]{\authornote{blue}{SS}{#1}} 
\newcommand{\plt}[1]{\authornote{magenta}{TPLT}{#1}} %For explanation of the template
\newcommand{\an}[1]{\authornote{cyan}{Author}{#1}}

%% Common Parts

\newcommand{\progname}{Software Engineering} % PUT YOUR PROGRAM NAME HERE
\newcommand{\authname}{Team 16, Durum Wheat Semolina
	\\ Alexander Moica
	\\ Yasmine Jolly
	\\ Jeffrey Wang
	\\ Jack Theriault
	\\ Catherine Chen
	\\ Justina Srebrnjak } % AUTHOR NAMES                 

\usepackage{hyperref}
    \hypersetup{colorlinks=true, linkcolor=blue, citecolor=blue, filecolor=blue,
                urlcolor=blue, unicode=false}
    \urlstyle{same}
                                


\begin{document}
	
	\title{\progname{}: System Verification and Validation Plan} 
	\author{\authname}
	\date{\today}
	
	\maketitle
	
	\pagenumbering{roman}
	
	\section{Revision History}
	
	\begin{tabularx}{\textwidth}{p{3cm}p{2cm}X}
		\toprule {\bf Date} & {\bf Version} & {\bf Notes}\\
		\midrule
		11/02/2022 & 1.0 & Completed Version 1\\
		\bottomrule
	\end{tabularx}
	
	\newpage
	
	\tableofcontents
	
	\listoftables
	\wss{Remove this section if it isn't needed}
	
	\newpage
	
	\section{Symbols, Abbreviations and Acronyms}
	
	\renewcommand{\arraystretch}{1.2}
	\begin{tabular}{l l} 
		\toprule		
		\textbf{symbol} & \textbf{description}\\
		\midrule 
		SRS & Software Requirements Specification\\
		MG & Module Guide\\
		MIS & Module Interface Specification\\
		VnV & Verification and Validation\\
		SFWRENG 4G06& Software Engineering 4G06 course\\
		PR& Pull Request\\
		\bottomrule
	\end{tabular}\\
	
	\wss{symbols, abbreviations or acronyms --- you can simply reference the SRS
		\citep{SRS} tables, if appropriate}
	
	\wss{Remove this section if it isn't needed}
	
	\newpage
	
	\pagenumbering{arabic}
	
	This document depicts the Verification and Validation Plan for the Utrition project. This includes verifying both implementation and documentation. The document will depict the plans for reviewing all critical documentation along with the system and unit tests to be used for code verification.  \wss{provide an introductory blurb and roadmap of the
		Verification and Validation plan}
	
	\section{General Information}
	
	\subsection{Summary}
	
	\wss{Say what software is being tested.  Give its name and a brief overview 
	of
		its general functions.}
		The purpose behind this document is to provide the testing plan that Utrition will undergo during its development period. This plan will ensure a high quality final product. Utrition is an application that allows individuals to discover the nutritional value of the foods they are consuming and track their previously eaten meals. This software consists of three main functions that are being tested such as image processing and identification, displaying an interface for the user, and writing to a database. (COME BACK TO THESE FUNCTIONS)
	
	\subsection{Objectives} (REDO)
	
	\wss{State what is intended to be accomplished.  The objective will be 
	around
		the qualities that are most important for your project.  You might have
		something like: ``build confidence in the software correctness,''
		``demonstrate adequate usability.'' etc.  You won't list all of the 
		qualities,
		just those that are most important.}
		
		The objectives behind the design of the testing plan for Utrition is to design to code to fit various qualities to ensure the effectiveness of the final product. Firstly, the developers of Utrition would like to ensure the correctness of this application as this is a health based application. We want to ensure that the artificial intelligence can correctly identify the food that is in the image and therefore output the correct nutritional information. Additionally the Utrition application is expected to be robust and to be able to deal with very common errors that may potentially occur such as a user uploading an image of the wrong type. All tests written in this document are written to ensure that the code meets these objectives standards. 

- build confidence in image identification AI
- Ensure nutrition facts are correct
- Prevent crashing of the system with unusual inputs
	
	\subsection{Relevant Documentation}
	
	\wss{Reference relevant documentation.  This will definitely include your 
	SRS
		and your other project documents (design documents, like MG, MIS, 
		etc).  You
		can include these even before they are written, since by the time the 
		project
		is done, they will be written.}

		This document references multiple other documents that are listed below:
		
		\begin{itemize}
			\item SRS, \citet{SRS}
			\item Development Plan, \citet{DevelopmentPlan}
			\item MG, \citet{MG}
			\item MIS, \citet{MIS}
		\end{itemize}
	
	\section{Plan}
	
	\wss{Introduce this section.   You can provide a roadmap of the sections to
		come.}
	This section details Durum Wheat Semolina's approaches for verifying and validating Utrition's documentation and implementation. Specific documentation to be reviewed are the SRS, Design documents (MG and MIS), and VnV Plan. Additionally, the verification and validation of Utrition's implementation is discussed. Tools being used for verification including automated testing frameworks and linters are also detailed below.
	
	\subsection{Verification and Validation Team}
	
	\wss{Your teammates.  Maybe your supervisor.
		You shoud do more than list names.  You should say what each person's 
		role is
		for the project's verification.  A table is a good way to summarize 
		this information.}
	
	Each team member will be contributing to the verification and validation of Utrition's supporting documentation and implementation. Jack Theriault is the team's testing lead and can be deferred to for any queries regarding verification procedures. It is expected that all team members create the tests that correspond to their written code. In addition, each team member's specific role in the verification and validation process is given below.
	
	\begin{table}[H]
		\centering
		\label{Table:Testing_Roles}
		\begin{tabular}{|p{3.2cm}|p{4cm}|p{5.5cm}|}
			\hline
			\textbf{Team Member} & \textbf{Role} & \textbf{Description}\\ \hline
			Alexander Moica& Dynamic Verification Lead& This role is responsible for leading the dynamic verification for Utrition's implementation.\\ \hline
			Yasmine Jolly& Software Validation Lead & This role is responsible for leading the validation effort of Utrition.\\ \hline
			Jeffrey Wang& Static Verification Lead & This role is responsible for leading the static verification for Utrition's implementation.\\ \hline
			Jack Theriault& VnV Plan Verification Lead& This role is responsible for leading the verification effort of the VnV plan.\\ \hline
			Catherine Chen& Design Verification Lead& This role is responsible for leading the verification effort of the design documents.\\ \hline
			Justina Srebrnjak& SRS Verification Lead& This role is responsible for leading the verification effort of the SRS. \\ \hline
		\end{tabular}
		\caption{Team Member Roles for Verification and Validation Testing}
	\end{table} 
	
	\subsection{SRS Verification Plan}
	
	\wss{List any approaches you intend to use for SRS verification.  This may 
	include
		ad hoc feedback from reviewers, like your classmates, or you may plan 
		for 
		something more rigorous/systematic.}
	
	\wss{Maybe create an SRS checklist?}
	
	To verify Utrition's SRS document, two approaches will be primarily used. Firstly, ad hoc reviews by fellow classmates are done to provide Durum Wheat Semolina with feedback from an outsider perspective. The rubric for the SRS is heavily used during this process. \\
	
	Once feedback from these reviews has been collected in addition to the feedback given by SFWRENG 4G06 TAs while marking version 0, the team will walk through the SRS and make any necessary changes. Since the SRS is critical when implementing Utrition, this review process will be done by all team members to ensure total coverage of possible improvements. The team will be instructed to go through all functional and non-functional requirements to identify any ambiguities or missing requirements. Once identified, the corresponding changes will be made. This walkthrough will be led by Justina Srebrnjak who is in charge of SRS verification.

	\subsection{Design Verification Plan}
	
	\wss{Plans for design verification}
	
	\wss{The review will include reviews by your classmates}
	
	\wss{Create a checklists?}
	
	Utrition's design documents will be verified through a combination of peer reviews and documentation walkthroughs. Once version 0 of the design documents has been completed, students from SFWRENG 4G06 will perform ad hoc reviews based on the criteria given in the design rubric. These critiques will make Durum Wheat Semolina aware of any discrepancies found in the documentation.\\
	
	Secondly, documentation walkthroughs will be conducted by all team members to verify the completeness of the design documents. The primary purpose of these walkthroughs is to verify that all functional requirements listed in the SRS document are fulfilled through the proposed design. Any missing functionality will be identified through this process and amended as needed. This verification plan will be led by Catherine Chen.
	
	\subsection{Verification and Validation Plan Verification Plan}
	
	\wss{The verification and validation plan is an artifact that should also 
	be verified.}
	
	\wss{The review will include reviews by your classmates}
	
	\wss{Create a checklists?}
	
	The verification plan for the VnV plan document will consist of two techniques. Firstly, ad hoc reviews from the team's classmates in SFWRENG 4G06 will be conducted. These reviews will identify any discrepancies indicated by the VnV plan rubric. Their resulting feedback will contribute to any changes made to the document.\\
	
	In addition to peer reviews, the Utrition team will participate in a walkthrough of the document while referring to the SRS. During this review, the team will verify that every requirement found in the SRS has at least one test that will test its completeness. This review will be led by Jack Theriault. 
	
	\subsection{Implementation Verification Plan}
	
	\wss{You should at least point to the tests listed in this document and the 
	unit
		testing plan.}
	
	\wss{In this section you would also give any details of any plans for 
	static verification of
		the implementation.  Potential techniques include code walkthroughs, 
		code
		inspection, static analyzers, etc.}
	
	Verifying Utrition's implementation will be done through a variety of dynamic and static verification techniques. In terms of dynamic methods, a variety of system and unit tests will be performed on the written code. More information on these tests can be found in sections 5 and 6 of this document. These tests will be created by the entire team. More specifically, the people who work on a certain functionality of Utrition will be responsible for creating the corresponding test cases. These tests will be automated and run on the system whenever new code is pushed to the team's main branch on GitHub. This ensures that new code does not create any regressions in the current system. Alexander Moica will be the dynamic verification lead which entails monitoring new test case additions and ensuring full testing coverage of new code which includes edge cases and error cases.\\

	Static verification will be done primarily through coding standards and code inspections. When writing code, each team member will be responsible for abiding by the coding guidelines outlined in Utrition's \href{https://github.com/jeff-rey-wang/utrition/blob/dff32f8ddc662d07db9bd74e0b3705aa657dae6e/docs/DevelopmentPlan/DevelopmentPlan.pdf}{Development Plan} in section 7. Python development will use the linter Pylint to enforce PEP8 standards. The Google style guides for JavaScript, HTML, and CSS will need to be followed manually. Additionally, coding inspections will be done for every new PR that is made to the team's main branch in GitHub. Before any PR is merged, at least one team member will be required to review the incoming code. During these reviews, team members will ensure complete functionality fulfillment inline with documentation, proper coding style, efficient implementation choices, and consistent naming practices for files, functions, and variables. Jeffrey Wang will lead this static verification plan to ensure all coding inspections are completed and answer any questions regarding static verification.
	
	\subsection{Automated Testing and Verification Tools}
	
	\wss{What tools are you using for automated testing.  Likely a unit testing
		framework and maybe a profiling tool, like ValGrind.  Other possible 
		tools
		include a static analyzer, make, continuous integration tools, test 
		coverage
		tools, etc.  Explain your plans for summarizing code coverage metrics.
		Linters are another important class of tools.  For the programming 
		language
		you select, you should look at the available linters.  There may also 
		be tools
		that verify that coding standards have been respected, like flake9 for
		Python.}
	
	\wss{If you have already done this in the development plan, you can point to
		that document.}
	
	\wss{The details of this section will likely evolve as you get closer to the
		implementation.}
	
	The tools used for automated testing and verification have been previously outlined in the \href{https://github.com/jeff-rey-wang/utrition/blob/dff32f8ddc662d07db9bd74e0b3705aa657dae6e/docs/DevelopmentPlan/DevelopmentPlan.pdf}{Development Plan} for Utrition in sections 6 and 7. 
	
	\subsection{Software Validation Plan}
	
	\wss{If there is any external data that can be used for validation, you 
	should
		point to it here.  If there are no plans for validation, you should 
		state that
		here.}
	
	\wss{You might want to use review sessions with the stakeholder to check 
	that
		the requirements document captures the right requirements.  Maybe task 
		based
		inspection?}
	
	\wss{This section might reference back to the SRS verification section.}
	
	In order to validate Utrition's implementation, interviews will be scheduled with potential users. In these interviews, the interview conductor will walk through the functional requirements of the system with the interviewee. The interviewee will be asked for their feedback on the current requirements of the system. This feedback may include adding or removing specific functionality (i.e. altering requirements). Software validation will be led by Yasmine Jolly. 
	
	\section{System Test Description}
	
	\subsection{Tests for Functional Requirements}
	The below areas cover all the nine functional requirements outlined in the SRS, dividing the functional requirements into three distinct sections. The first section details user actions associated with image upload and encompasses FR1 and FR2. The second section details image processing done by the system which encompasses FR3, FR4, FR5, FR6, and FR7. Lastly, the third section outlines the user steps to view nutritional data which correlates to groupings of functional requirements.
	
	\subsubsection{Image Upload}
	This section has three tests, with each corresponding to an action the user 
	can take with Utrition, corresponding to FR1 and FR2 in the SRS. 
	Testing that users can successfully submit one or multiple images is 
	important to the core functionality of the other components of the 
	application.
	
	\paragraph{Single image upload}
	
	\begin{enumerate}
		
		\item{one-upload-1\\}
		
		Control: Manual
		
		Initial State: No image in the system
		
		Input: An image file
		
		Output: The system accepts the image upload
		
		Test Case Derivation: The system should accept an uploaded image file from the user.
		
		How test will be performed: A locally stored image will be uploaded to 
		the application through the image upload prompt.
		
		\item{one-upload-2\\}
		
		Control: Manual
		
		Initial State: No image in the system.
		
		Input: A file of extension .txt will be uploaded into the system.
		
		Output: The system will not contain the uploaded file. An error message 
		will be returned.
		
		Test Case Derivation: The system should only save the inputted image if 
		it is of the proper file extension.
		
		How test will be performed: A file of type .txt will be saved on the 
		tester's device. The tester will upload this file to the system.
		
	\end{enumerate}
	
	\paragraph{Multi-image upload}
	
	\begin{enumerate}
		
		\item{multi-upload-1\\}
		
		Control: Manual
		
		Initial State: No image in the system
		
		Input: Three image files
		
		Output: The system accepts the image uploads
		
		Test Case Derivation: The system should accept multiple uploaded image files from the user at the same time.
		
		How test will be performed: Three locally stored images will be uploaded to the application through image upload prompts. 
		
	\end{enumerate}
	
	
	\subsubsection{Image Processing}
	This section contains tests that relate to the processing of an image 
	uploaded by the user. Tests correspond to actions that will be taken by the 
	application during image processing. Relevant functional requirements 
	include FR3, FR4, FR5, FR6, and FR7.
	
	\paragraph{Image Identification}
	
	\begin{enumerate}
		
		\item{image-identification-1\\}
		
		Control: Automatic
		
		Initial State: System contains an image of extension .png type.
		
		Input: System is prompted to process the image.
		
		Output: The type of food that is captured in the image is returned as a 
		string.
		
		Test Case Derivation: The system should identify the food present in an 
		image, and return the name of the food item as a string.
		
		How test will be performed: The system identified item will be verified 
		that it matches the true item contained in the image.
		
		\item{image-identification-2\\}
		
		Control: Automatic
		
		Initial State: System contains an image of extension .jpg type.
		
		Input: System is prompted to process the image.
		
		Output: The type of food that is captured in the image is returned as a 
		string.
		
		Test Case Derivation: The system should identify the food present in an 
		image, and return the name of the food item as a string.
		
		How test will be performed: The system identified item will be verified 
		that it matches the true item contained in the image.
		
		\item{image-identification-3\\}
		
		Control: Automatic
		
		Initial State: System contains an image of extension .jpeg type.
		
		Input: System is prompted to process the image.
		
		Output: The type of food that is captured in the image is returned as a 
		string.
		
		Test Case Derivation: The system should identify the food present in an 
		image, and return the name of the food item as a string.
		
		How test will be performed: The system identified item will be verified 
		that it matches the true item contained in the image.
		
	\end{enumerate}
	
	\paragraph{Call and Fetch API Response}
	
	\begin{enumerate}
		
		\item{api-1\\}
		
		Control: Automatic
		
		Initial State: System on standby.
		
		Input: Food item name as a string.
		
		Output: JSON containing nutrition facts for the inputted food item.
		
		Test Case Derivation: A request will be made to the Nutritionix API 
		with the food item name, which will return a response body containing 
		the food item's nutrition facts.
		
		How test will be performed: The API response will be verified that the 
		contents are as expected.
		
	\end{enumerate}
	
	\paragraph{Logging Data}
	
	\begin{enumerate}
		
		\item{log-data-1\\}
		
		Control: Automatic
		
		Initial State: System contains nutritional data of a food item.
		
		Input: Request to log the data to the database.
		
		Output: Database is updated as the system writes the data to the 
		database.
		
		Test Case Derivation: Nutritional data of a food will be saved for 
		future reference. This is done by logging the data to the database.
		
		How test will be performed: After the system receives the request to 
		update the database, the database will be verified that it is updated 
		with the new data.
		
	\end{enumerate}
	
	\paragraph{Reading Data}
	
	\begin{enumerate}
		
		\item{read-data-1\\}
		
		Control: Automatic
		
		Initial State: Database contains nutritional data of a food item.
		
		Input: Request to fetch the data of a particular food item.
		
		Output: Nutritional data of the requested food item.
		
		Test Case Derivation: The system shall be able to fetch previously 
		recorded data.
		
		How test will be performed: After the system receives the request to 
		fetch the data of a food item, the returned data will be verified that 
		it matches with the saved data in the database.
		
		\item{read-data-2\\}
		
		Control: Automatic
		
		Initial State: Database contains nutritional data of 5 different dates.
		
		Input: Request to fetch the data from the last 3 most recent dates.
		
		Output: A list of the nutritional data from the last 3 most recent 
		dates.
		
		Test Case Derivation: The system shall be able to fetch previously 
		recorded data. The system shall be able to fetch data from a range of 
		dates.
		
		How test will be performed: The nutritional data from the 3 most recent 
		dates will be verified if it matches the data in the database.
		
		\item{read-data-3\\}
		
		Control: Automatic
		
		Initial State: Database contains nutritional data.
		
		Input: Request to fetch the data of a particular food item that is not 
		logged in the database.
		
		Output: No data is returned.
		
		Test Case Derivation: The system shall be able to fetch previously 
		recorded data. This is an edge case for if the system is requested to 
		read data from the database that does not exist. In this case, no data 
		will be returned.
		
		How test will be performed: The output will be verified to be empty.
	\end{enumerate}
	
	\subsubsection{Nutritional Facts Display}
	This section has three tests, with each corresponding to an action the user can take with Utrition. Each test corresponds to (FR3, FR4, FR5, FR6, FR8), (FR7, FR8), and (FR7, FR8, FR9) respectively in the SRS. These tests will test the actions the user can take after uploading an image, which are comprised of viewing past and present nutritional data in textual and graphical formats.
	
	
	
	\paragraph{Viewing the nutritional data of an uploaded food image}
	
	\begin{enumerate}
		
		\item{current-nutrition-1\\}
		
		Control: Automatic
		
		Initial State: Image(s) exists in system
		
		Input: Input signal from the image upload process
		
		Output: The nutritional data of the food
		
		Test Case Derivation: Primary black-box test for our system as a whole. The system will take the uploaded image(s), analyse them for the food they display, cross-reference them with a nutrition database, and display the information to the user.
		
		How test will be performed: The program will automatically test that the resulting nutrition data of a pre-uploaded image file matches that of the nutrition database entry for the food.
		
	\end{enumerate}
	
	\paragraph{Viewing textual nutritional data of logged foods}
	
	\begin{enumerate}
		
		\item{past-nutrition-text-1\\}
		
		Control: Automatic
		
		Initial State: Logged foods exist in system
		
		Input: Request to view nutritional data of logged foods as text
		
		Output: The nutritional data of past logged foods in a textual format
		
		Test Case Derivation: The system will keep a record of all analysed foods for the user, allowing the user to request to see the nutritional data of their past uploads. This data can be displayed in a human readable textual format.
		
		How test will be performed: The program will automatically test that a file of pre-logged foods can be processed to return their corresponding nutritional data as text.
		
	\end{enumerate}
	
	\paragraph{Viewing graphical nutritional data of logged foods}
	
	\begin{enumerate}
		
		\item{past-nutrition-graph-1\\}
		
		Control: Automatic
		
		Initial State: Logged foods exist in system
		
		Input: Request to view nutritional data of logged foods as graph
		
		Output: The nutritional data of past logged foods in a graphical format
		
		Test Case Derivation: The system will keep a record of all analysed foods for the user, allowing the user to request to see the nutritional data of their past uploads. This data can be displayed in a graphical format.
		
		How test will be performed: The program will automatically test that a file of pre-logged foods can be processed to return their corresponding nutritional data in the form of a graph.
		
	\end{enumerate}
	\subsection{Traceability Between Test Cases and Requirements}
	
	\newpage
	\begin{table}[H]
	
	\begin{tabular}{|p{3.5cm}|p{6.5cm}|p{4.5cm}|}

	\hline
	\textbf{Requirement \#} & \textbf{Description}                                                                                                                                                                  & \textbf{Test ID(s)}                                                                                          \\ \hline
	FR1                     & The user must have the ability to upload a 
	digital image of a standard image type to the 
	system.                                                                     
	                 &
	\begin{tabular}[c]{@{}l@{}}one-upload-1\\ 
	one-upload-2\end{tabular}                                                  
	                                            \\
	 \hline
	FR2                     & The user shall have the ability to upload multiple digital images of standard image types to the system.                                                                              & multi-upload-1                                                                                               \\ \hline
	FR3                     & The system will be able to identify the type of 
	food that is captured in an 
	image.                                                                      
	                              &
	 \begin{tabular}[c]{@{}l@{}}image-identification-1\\ 
	 image-identification-2\\ 
	 image-identification-3\\current-nutrition-1\end{tabular}                   
	        
		                                                                 \\ 
		\hline
	FR4                     & The system will be able to make an API call to an 
	external database of nutrition facts for a variety of foods, including 
	their macro-nutrients, micro-nutrients, and caloric details. & 
	\begin{tabular}[c]{@{}l@{}}api-1\\ 
		current-nutrition-1\end{tabular}                                        
		                                                    \\ \hline
	FR5                     & The system will be able to retrieve the nutrition 
	facts for a specific 
	food.                                                                       
	                                   &
	 \begin{tabular}[c]{@{}l@{}}api-1\\ 
		read-data-1\\ 
		current-nutrition-1\end{tabular}                                        
		                                                    \\
		 \hline
	FR6                     & The system will log the nutritional data of a 
	food.                                                                       
	                                                            &  
	\begin{tabular}[c]{@{}l@{}}api-1\\ 
		log-data-1\\ 
		current-nutrition-1\end{tabular}                                        
		                                               \\ \hline
	FR7                     & The user will be able to access the nutritional 
	data of previously logged 
	foods.                                                                      
	                                &
	 \begin{tabular}[c]{@{}l@{}}read-data-1\\ read-data-2\\ read-data-3\\ 
	past-nutrition-text-1\\ 
	past-nutrition-graph-1\end{tabular}                       \\ \hline
	FR8                     & The system will display the nutritional information of a food to the user.                                                                                                            & \begin{tabular}[c]{@{}l@{}}current-nutrition-1\\ past-nutrition-text-1\\ past-nutrition-graph-1\end{tabular} \\ \hline
	FR9                     & The system will display the history of logged nutritional data in a graph.                                                                                                            & past-nutrition-graph-1                                                                                       \\ \hline
	
	\end{tabular}
	\caption {Traceability between System Tests and Functional Requirements}
	\end{table}
	
	\subsection{Tests for Nonfunctional Requirements}
	
	\wss{The nonfunctional requirements for accuracy will likely just reference 
	the
		appropriate functional tests from above.  The test cases should mention
		reporting the relative error for these tests.  Not all projects will
		necessarily have nonfunctional requirements related to accuracy}
	
	\wss{Tests related to usability could include conducting a usability test 
	and
		survey.  The survey will be in the Appendix.}
	
	\wss{Static tests, review, inspections, and walkthroughs, will not follow 
	the
		format for the tests given below.}

	
	\subsubsection{Look and Feel Testing}
	
	%\paragraph{Title for Test}
	
	\begin{enumerate}
		\item{NFR1\\} 
		
		Type: Functional, Dynamic, and Manual.
		
		Initial State: A developer from Durum Wheat Semolina has launched the Utrition app.
		
		Input/Condition: The developer measures the distance between the displayed user interface components.
		
		Output/Result: The distance between user interface components surpasses 20 pixels.
		
		How test will be performed: Each developer launches the Utrition app and opens the \href{https://www.rapidtables.com/web/tools/pixel-ruler.html}{Pixel Measuring Tool} . The developer will use the tool by viewing a page on Utrition, then clicking “Print Screen'' on their keyboard. Then, the developer pastes the screenshot into the tool. The developer clicks the two spots he would like to measure distance from. The developer measures the distances between the following components:	
		
		\begin{enumerate}
			\item[$-$] On the main menu:
			\begin{itemize}
				\item Main Menu Button
				\item Upload Image Button
				\item Past Nutritional Data Button
				\item Utrition Logo
				\item The Edge of the Application
			\end{itemize}
			\item[$-$] When uploading an image:
			\begin{itemize}
				\item Utrition Logo
				\item “Add More” Button
				\item Loading Bar
				\item Loading Text
				\item Main Menu Button
				\item The Edge of the Application	
			\end{itemize}
			\item[$-$] When viewing nutritional data:
			\begin{itemize}
				\item Utrition Logo
				\item Nutrition Data Text
				\item Associated Nutrition Symbol
				\item Continue Button
				\item Main Menu Button
				\item Photo of Identified Food
				\item Name of Identified Food
				\item The Edge of the Application	
			\end{itemize}
			\item[$-$] When viewing past nutritional data:
			\begin{itemize}
				\item Utrition Logo
				\item Nutrition Data Text
				\item Associated Nutrition Symbol
				\item Next Page Button
				\item Previous Page Button
				\item Main Menu Button
				\item Column and Row Headers
				\item The Edge of the Application
			\end{itemize}
			\item[$-$] When viewing past nutritional data chart:
			\begin{itemize}
				\item Utrition Logo
				\item Chart Entry Dates
				\item The Chart
				\item Nutrition Data Text
				\item Associated Nutrition Symbol
				\item Next Page Button
				\item Previous Page Button
				\item Back Button
				\item Main Menu Button
				\item The Edge of the Application	
			\end{itemize}
			
		\end{enumerate}
		
	\end{enumerate}
	
	\subsubsection{Usability and Humanity Testing}
	
	%\paragraph{Title for Test}
	
	\begin{enumerate}
		
		\item{NFR2\\}
		
		Type: Functional, Dynamic, and Manual.
		
		Initial State: The user is viewing a page of the user interface (Upload Image, View Nutritional Data, View Past Nutritional Data, View Past Nutritional Data Chart)
		
		Input/Condition: The user clicks on the “Menu” button located at the bottom of the user interface.
		
		Output/Result: The user will be brought to the main menu screen.
		
		How test will be performed: A developer of Durum Wheat Semolina will access every page listed in “Initial State” and will attempt to reach the main menu in 2 or less clicks from the respective page.
		
		
		\item{NFR3\\}
		
		Type: Functional, Dynamic, and Manual.
		
		Initial State: The user has previously entered 5 items in Utrition, and is now viewing the main menu.
		
		Input: The user clicks on the “View Past Nutritional Data” button located at the right of the user interface.
		
		Output: The user views a list displaying all 5 food items with their respective details in: food name, calories, fat, sodium, proteins, carbohydrates, sugars, and date entered.
		
		How test will be performed: A developer of Durum Wheat Semolina will open Utrition on their personal device and access the main menu. The developer will upload an image of different food items 5 times, with a minimum 1 minute time difference between each input. The uploaded images of the food items will be randomly selected by the developer in the utrition/test/testPhotos directory. The developer closes Utrition and reopens it. The developer clicks on the “View Past Nutritional Data” button, and views the list displaying the 5 different food items and their respective details.
		
		\item{NFR4\\}
		
		Type: Functional, Dynamic, and Manual.
		
		Initial State: The user has previously entered 5 items in Utrition, and has just clicked on the “View Past Nutritional Data” button.
		
		Input/Condition: The user clicks on the “View Past Nutritional Data Chart” button located at the top of the user interface.
		
		Output/Result: The user views a chart displaying all 5 food items with their respective details in: food name, calories, proteins, carbohydrates, sugars, and date entered.
		
		How test will be performed: A developer of Durum Wheat Semolina will open Utrition on their personal device and access the main menu. The developer will upload an image of different food items 5 times, with a minimum 1 minute time difference between each input. The uploaded images of the food items will be randomly selected by the developer in the utrition/test/testPhotos directory. The developer closes Utrition and reopens it. The developer clicks on the “View Past Nutritional Data” button, and then clicks on the “View Past Nutritional Data Chart” button. The developer views a chart displaying 5 different food items and their respective details.
		
		\item{NFR5\\}
		
		Type: Functional, Dynamic, and Manual.
		
		Initial State: Utrition’s main menu is opened on the user’s personal device.
		
		Input/Condition: The user clicks on the “Upload Image” button located at the middle of the main menu and uploads a photo of food.
		
		Output/Result: The user views the food’s nutritional information.
		
		How test will be performed: Each developer of Durum Wheat Semolina will load Utrition onto their personal device. Each developer will ask 2 people aged 14 or older to upload and retrieve the nutritional data of the food item “pineapple.jpg” found in the utrition/test/testPhotos directory. The developer will track the time it takes for each test.
		
		\item{NFR6\\}
		
		Type: Functional, Dynamic, and Manual.
		
		Initial State: The user loads into Utrition’s main menu.
		
		Input/Condition: The user views every page of the user interface (Upload Image, View Nutritional Data, View Past Nutritional Data, View Past Nutritional Data Chart).
		
		Output/Result: The user will see the respective symbol associated with every mention of calories, fat, sodium carbohydrates, sugar, and protein.
		
		How test will be performed: On each screen, including the main menu, a developer on Durum Wheat Semolina will check if there is a symbol associated with every mention of calories, fat, sodium carbohydrates, sugar, and protein. The developer will open Utrition on their personal device. The developer will click on the “Upload Image” button, and then the developer will upload a random image of food found in the testPhotos directory. The developer will view the nutritional information and then click on the “Menu” button. The developer clicks on the “View Past Nutritional Data” button, and then the “View Past Nutritional Data Chart” button.
		
		\item{NFR7\\}
		
		Type: Functional, Dynamic, and Manual.
		
		Initial State: The user has previously entered 2 items in Utrition, and is now viewing the main menu.
		
		Input/Condition: The user views every page of the user interface (Upload Image, View Nutritional Data, View Past Nutritional Data, View Past Nutritional Data Chart).
		
		Output/Result: The user does not see any backend calculations.
		
		How test will be performed: : On each screen, a developer on Durum Wheat Semolina will check if any backend calculations are displayed to the user. The developer will open Utrition on their personal device and access the main menu. The developer will upload an image of different food items 2 times, with a minimum 1 minute time difference between the inputs. The uploaded images of the food items will be randomly selected by the developer in the testPhotos directory. The developer closes Utrition and reopens it. The developer will click on the “Upload Image” button, and then the developer will upload an image of food. The developer will view the nutritional information and then click on the “Menu” button. The developer clicks on the “View Past Nutritional Data'' button, and then the “View Past Nutritional Data Chart” button.
		
		\item{NFR8\\}
		
		Type: Functional, Dynamic, and Manual.
		
		Initial State: The user loads into Utrition’s main menu with a stable internet connection.
		
		Input/Condition: The user views every page of the user interface (Upload Image, View Nutritional Data, View Past Nutritional Data, View Past Nutritional Data Chart).
		
		Output/Result: 
		\begin{itemize}
			\item Utrition is available to be used by users at all times since it is a web app.
			\item The user does not hear Utrition generate any sound.
		\end{itemize}
		
		How test will be performed: On each screen, a developer on Durum Wheat Semolina will check if a sound plays. The developer will open Utrition on their personal device. The developer will click on the “Upload Image” button, and then the developer will upload an image of food. The uploaded images of the food item will be randomly selected by the developer in the testPhotos directory The developer will view the nutritional information and then click on the “Menu” button. The developer clicks on the “View Past Nutritional Data” button, and then the “View Past Nutritional Data Chart” button.
		
		\item{NFR9\\}
		
		Type: Static and Manual.
		
		Initial State: A developer on Durum Wheat Semolina views Utrition’s github page. 
		
		Input/Condition: The developer checks every single directory if there is an audio file.
		
		Output/Result: The developer removes all audio files found in Utrition’s github.
		How test will be performed: The developer loads into Utrition’s github page (https://github.com/jeff-rey-wang/utrition/) and views the files in every single directory. For each directory the developer will check for the following file types:
		
		\begin{itemize}
			\item MP3
			\item AAC
			\item Ogg Vorbis
			\item FLAC
			\item ALAC
			\item WAV
			\item AIFF
			\item DSD
			\item PCM
		\end{itemize}
	\end{enumerate}
	
	\subsubsection{Performance Testing}
	
	%\paragraph{Title for Test}
	
	\begin{enumerate}
		
		\item{NFR10\\} 
		
		Type: Functional, Dynamic, and Manual.
		
		Initial State: The user has previously entered 10 items in Utrition, and is now viewing the main menu.
		
		Input/Condition: The user views every page of the user interface (Upload Image, View Nutritional Data, View Past Nutritional Data, View Past Nutritional Data Chart).
		
		Output/Result: The user will be able to access every page of the user interface in 2 seconds or less.
		
		How test will be performed: A developer on Durum Wheat Semolina will measure the time it takes for each new screen to load, excluding the screen that identifies each food item, using a stopwatch. The developer will open Utrition on their personal device and access the main menu. The developer will upload an image of different food items 10 times, with a minimum 1 minute time difference between the inputs. The uploaded images of the food items will be randomly selected by the developer in the utrition/test/testPhotos directory. The developer closes Utrition and reopens it. The developer will click on the “Upload Image” button, and then the developer will upload an image of food. The developer will view the nutritional information and then click on the “Menu” button. The developer clicks on the “View Past Nutritional Data'' button, and then the “View Past Nutritional Data Chart” button. The developer clicks on the “Menu” button.
		
		\item{NFR11\\} 
		
		Type: Functional, Dynamic, and Manual.
		
		Initial State: The user loads into Utrition’s main menu, and has clicked on the “Upload Image” button.
		
		Input/Condition: The user uploads 3 images simultaneously.
		
		Output/Result: 
		\begin{itemize}
			\item The user is able to view the identification for all 3 food items in 10 or less seconds.
			\item The user is able to view the nutritional information for all 3 food items in 5 or less seconds.
		\end{itemize}
		
		How test will be performed: A developer on Durum Wheat Semolina will 
		open Utrition and click on the “Upload Image” button. The developer 
		will upload a random image of a food item found in the testPhotos 
		directory, and then click on the “Add More” button. The developer 
		uploads 2 more random images of different food items and clicks to view 
		the foods’ nutritional information. The developer measures the amount 
		of time it takes for Utrition to notify the user the name of the 
		identified food items. The developer measures the amount of time it 
		takes for the system to change from the food identification interface 
		to the nutritional information interface.
		
		\item{nutrition-display-performance-1\\}
		
		Type: Automatic
		
		Initial State: Nutritional data of a food item is contained in the 
		system.
		
		Input/Condition: Request to display the nutritional information of a 
		food item.
		
		Output/Result: System will generate the HTML body to be displayed on 
		the user 
		interface.
		
		How test will be performed: The test will measure if this interaction 
		was completed in 5 seconds or less.
		
		\item{NFR12\\} 
		
		Type: Functional, Dynamic, and Manual.
		
		Initial State: The user loads into Utrition’s main menu, and has clicked on the “Upload Image” 
		button.
		
		Input/Condition: The user uploads an image of food.
		
		Output/Result: The food is correctly identified and the correlated nutritional facts are correct.
		How test will be performed: A developer on Durum Wheat Semolina will open Utrition and click on the “Upload Image” button. The developer will upload the images of the following foods separately (images found in testPhotos directory):
		\begin{itemize}
			\item Uncooked Pork
			\item Corn
			\item Lettuce
			\item Beef
			\item Penne Pasta
			\item Rice
			\item Milk
			\item Butter
			\item Wheat Bread
			\item Orange Juice
		\end{itemize} 
		
		The developer will check the accuracy of each identified food item and 
		their respective nutritional facts.
		
		\item{image-identification-accuracy-1\\}
		
		Type: Automatic
		
		Initial State: System contains a set of images; all with foods, and of 
		extension .jpg, .png, or .jpeg.
		
		Input/Condition: System prompted to process image, for each image 
		within the provided set.
		
		Output/Result: For each image, the system should identify the food 
		present in an image, and return the name of the food item as a string.
		
		How test will be performed: The test will check if among all the 
		provided images, more than 90\% of the images have been assessed 
		correctly.
		
		\item{identification-accuracy-2\\}
		
		Type: Automatic
		
		Initial State: System contains a set of images; some with and others 
		without foods, and of extension .jpg, .png, or 
		.jpeg.
		
		Input/Condition: System prompted to process image, for each image 
		within the provided set.
		
		Output/Result: The system should identify the food present in an image, 
		and return the name of the food item as a string, and prompt the user 
		if no food is detected.
		
		How test will be performed: The test will check if among all the 
		provided images, more than 70\% of the images have been assessed 
		correctly.
		
		\item{NFR13\\} 
		
		Type: Functional, Dynamic, and Manual.
		
		Initial State: The user loads into Utrition’s main menu, and has clicked on the “Upload Image” button.
		
		Input/Condition: The user uploads 4 photos of different food items.
		
		Output/Result: The user is notified that they are not able to proceed with viewing the identified foods or their nutritional information.
		
		How test will be performed: A developer on Durum Wheat Semolina will open Utrition and click on the “Upload Image” button. The developer will upload an image of a random food item found in the testPhotos directory, and then click on the “Add More” button. The developer uploads 3 more random images of different food. The developer is notified that they cannot proceed with viewing the nutritional information unless they remove 1 image.
		
		\item{NFR14\\} 
		
		Type: Static and Manual.
		
		Initial State: A developer on Durum Wheat Semolina opens their internet browser.
		
		Input/Condition: The developer is able to access \href{https://github.com/jeff-rey-wang/utrition}{Utrition’s github page} and download the repository onto their personal device.
		
		Output/Result: Utrition is able to be downloaded by any user.
		
		How test will be performed: The developer will access Utrition’s github on their personal device’s web browser. The developer will click on “Code”, and then “Download ZIP”. The developer will verify that Utrition has been downloaded onto their device.
		
		\item{NFR15\\} 
		
		Type: Functional, Dynamic, and Manual.
		
		Initial State: The user loads into Utrition’s main menu, and has clicked on the “Upload Image” button.
		
		Input/Condition: The user uploads a very large photo for Utrition to identify.
		
		Output/Result: The user is notified that they are not able to proceed with viewing the identified foods or their nutritional information.
		
		How test will be performed: A developer on Durum Wheat Semolina will open Utrition and click on the “Upload Image” button. The developer will upload “art.png” found in the testPhotos directory. The developer is notified that they cannot proceed with viewing the nutritional information unless they decrease the uploaded image’s file size.
		
		\item{NFR16\\} 
		
		Type: Functional, Dynamic, and Manual.
		
		Initial State: The user loads into Utrition’s main menu, and has clicked on the “Upload Image” button.
		
		Input/Condition: The user uploads an unintelligible image for Utrition to identify.
		
		Output/Result: The user is notified that Utrition is not confident in the system’s identification. The system will suggest to upload another image.
		
		How test will be performed: A developer on Durum Wheat Semolina will 
		open Utrition and click on the “Upload Image” button. The developer 
		will upload “desk.jpg” found in the testPhotos directory. The developer 
		is notified that the machine learning algorithm is not confident in the 
		identification of the food, and that they should upload another image.
		
		\item{failed-identification-1\\}
		
		Type: Automatic
		
		Initial State: System is identifying food objects within the user's 
		uploaded image(s).
		
		Input/Condition: Uploaded image.
		
		Output/Result: Error message is displayed prompting user of a failed 
		identification.
		
		How test will be performed: Test will check if an error message element 
		has been rendered.
		
		\item{NFR17\\}
		Type: Functional, Dynamic, and Manual.
		
		Initial State: The user loads into Utrition’s main menu, and has clicked on the “Upload Image” button without connecting to the internet.
		
		Input/Condition: The user uploads an image for Utrition to identify.
		
		Output/Result: The user is notified that the system is unable to retrieve nutritional information.
		
		How test will be performed: A developer on Durum Wheat Semolina will open Utrition and click on the “Upload Image” button. The developer will upload a random working image found in the testPhotos directory. The developer is notified that they are not connected to the internet. If the developer has not completed these steps in 20 seconds, an error message appears at the top of their screen displaying: “You are not connected to the internet!”. The developer will be able to continue with the test.
		
		\item{NFR18\\} 
		Type: Functional, Dynamic, and Manual.
		
		Initial State: The user loads into Utrition’s main menu, and has clicked on the “Upload Image” button.
		
		Input/Condition: The user uploads an image for Utrition to identify.
		
		Output/Result: The user is notified that the system has identified the food item, but could not find the food’s nutritional data.
		
		How test will be performed: A developer on Durum Wheat Semolina will open Utrition and click on the “Upload Image” button. The developer will upload “dragonfruit.jpg” found in the testPhotos directory. The developer is notified that they have uploaded an image of a “Dragon Fruit”, but the nutritional data cannot be found.
		
		\item{NFR19\\} 
		
		Type: Functional, Dynamic, and Manual.
		
		Initial State: The user views the main menu in Utrition without previously inputting any food items.
		
		Input/Condition: The user clicks on the “View Past Nutritional Data” button, and then clicks on the “View Past Nutritional Data Chart” button.
		
		Output/Result: The user views an empty list with column entries: food name, calories, proteins, carbohydrates, sugars, and date entered. The user will see an error message detailing that there are no previous nutritional logs. The user will then see an empty canvas for the chart, and text detailing that there are no previous nutritional logs.
		
		How test will be performed: A developer of Durum Wheat Semolina will open Utrition on their personal device and access the main menu. The developer clicks on the “View Past Nutritional Data” button, and views the list displaying an empty list with the column entries listed in “Output”. Above this list, the developer reads an error message displaying “There are no previous food items recorded.”. The developer clicks on the “View Past Nutritional Data Chart” button located at the top of the user interface. The developer is prompted by an error message displaying “There are no previous food items recorded.”
		
		\item{NFR20\\}
		Type: Functional, Dynamic, and Manual.
		
		Initial State: A user loads into Utrition’s main menu, and has clicked on the “Upload Image” button.
		
		Input/Condition: The user uploads a PDF document as an image.
		
		Output/Result: The user is notified that they are not able to proceed with viewing the identified foods or their nutritional information.
		
		How test will be performed: A developer on Durum Wheat Semolina will open Utrition and click on the “Upload Image” button. The developer will upload the image “waffle.pdf” to Utrition from the testPhotos directory. The developer is notified that they cannot proceed with viewing the nutritional information, since Utrition does not support “.pdf” file types.
		
	\end{enumerate}
	
	\subsubsection{Operational and Environmental Testing}
	
	%\paragraph{Title for Test}
	
	\begin{enumerate}
		
		\item{NFR21\\} 
		
		Type: Static and Manual.
		
		Initial State: A user aged 14 or older has read Utrition’s “README.md” located at the bottom of the github page.
		
		Input/Condition: The user installs Utrition on their personal device
		
		Output/Result: The intended demographic is able to install Utrition on their computer.
		
		How test will be performed: Each developer on Durum Wheat Semolina will send Utrition’s github to 5 of their peers. The developers will send the following message: “Hello! Whenever you have time available, could you please attempt to install my software? The installation guide can be found under “README.md” at the bottom of this page: \href{https://github.com/jeff-rey-wang/utrition/}{https://github.com/jeff-rey-wang/utrition/}. Please let me know if you were or were not able to install the program with the installation guide. Thanks!”
		
		\item{NFR22\\} 
		
		Type: Static and Manual.
		
		Initial State: A user visits Utrition’s github page.
		
		Input/Condition: The user checks that the latest change to Utrition’s github is not after April 30th, 2022.
		
		Output/Result: Utrition will not have any updates after the final release.
		
		How test will be performed: A developer on Durum Wheat Semolina will visit Utrition’s github page bi-yearly to ensure no updates are made to Utrition’s github.
		
	\end{enumerate}
	
	\subsubsection{Maintainability and Support Testing}
	
	%\paragraph{Title for Test}
	
	\begin{enumerate}
		
		\item{NFR23\\} 
		
		Type: Functional, Dynamic, and Manual.
		
		Initial State: The developer is in the utrition/src/database directory.
		
		Input/Condition: The developer switches the database file found in the database directory for another database.
		
		Output/Result: Backend developers are able to upload new data to Utrition’s database.
		
		How test will be performed: The developer travels to the utrition/src/database directory on their device. The developer replaces the database found in the directory with “testdatabase.jpg”, which can be found in the testPhotos directory. The developer launches Utrition on their device to see that the system still runs.	
		
		\item{NFR24\\} 
		
		Type: Functional, Dynamic, and Manual.
		
		Initial State: The user has installed Utrition onto their Windows, macOS, or Linux device.
		
		Input/Condition: The user is able to upload an image of food and view all pages of the user interface.
		
		Output/Result: All of Utrition’s functionality can be accessed by Windows, macOS, and Linux devices
		
		How test will be performed: Developers of Durum Wheat Semolina will install Utrition on 2 of each Windows, macOS, and Linux device. The developer will open Utrition on the device, and then clicks on the “Upload Image” button. The developer will upload a random image of a food item found in the testPhotos directory, and then click on the “Add More” button. The developer uploads 2 more random images of different food. The developer continues to see their foods’ nutritional information. The developer clicks on the “Menu” button. The developer clicks on the “View Past Nutritional Data” button, and then clicks on the “View Past Nutritional Data Chart” button.
		
	\end{enumerate}
	
	\subsubsection{Security Testing}
	
	%\paragraph{Title for Test}
	
	\begin{enumerate}
		
		\item{NFR25\\} 
		
		Type: Functional, Dynamic, and Manual.
		
		Initial State: The user is viewing the nutritional facts for a food item they just uploaded
		
		Input/Condition: After idling for 3 minutes, the user closes Utrition and reopens 
		it.
		
		Output/Result: The user will open Utrition and will view the nutritional facts page they were previously on.
		
		How test will be performed: A developer on Durum Wheat Semolina will open Utrition and will click on the “Upload Image” button. The developer will upload “waffle.jpg” found in the testPhotos directory. The developer views the nutritional facts of a waffle, and then idles for 3 minutes. After the elapsed time, the developer closes their instance of Utrition and opens it again. The developer will view the nutritional facts of a waffle instead of the main menu.
		
		\item{NFR26\\} 
		
		Type: Static and Manual.
		
		Initial State: A Durum Wheat Semolina developer visits Utrition’s github page.
		
		Input/Condition: The developer checks the userdata directory for any files.
		
		Output/Result: The userdata folder will be empty on github, so Utrition will only contain personalized data.
		
		How test will be performed: The developer loads into \href{https://github.com/jeff-rey-wang/utrition/}{Utrition’s github page} and goes into the utrition/src/userdata directory. The developer ensures that no userdata is preloaded into Utrition’s github.
		
		
	\end{enumerate}
	
	\subsubsection{Cultural Testing}
	
	%\paragraph{Title for Test}
	
	\begin{enumerate}
		\item{NFR27\\} 
		
		Type: Functional, Dynamic, and Manual.
		
		Initial State: Utrition’s main menu is being displayed to a user.
		
		Input/Condition: The user is told by a Durum Wheat Semolina developer to find culturally insensitive material for a reward of 5 canadian dollars.
		
		Output/Result: The user will be questioned after 5 minutes of searching if they have found any culturally insensitive material while using Utrition.
		
		How test will be performed: Each developer will approach 5 peers with Utrition’s main menu open on the developer’s device. The developer will give full control over to their peers, and will tell their peers to search for culturally insensitive material for \$5. After 5 minutes are up, the developer will ask their peers if they have found any culturally insensitive material.
		
	\end{enumerate}
	
	\subsubsection{Legal Testing}
	
	%\paragraph{Title for Test}
	
	\begin{enumerate}
		\item{NFR28\\}
		
		Type: Static and Manual.
		
		Initial State: A developer of Durum Wheat Semolina views Utrition’s github repository.
		
		Input/Condition: The developer inspects all code to ensure Utrition is following Canada’s Anti-Spam Legislation Requirements for Installing Computer Programs, Canada’s Guide for Publishing Open Source Code, Google’s HTML/CSS Style Guide, and Google’s guide for material design.
		
		Output/Result: Utrition is in compliance with all the standards laid out by the SRS document.
		
		How test will be performed: The developer goes into the utrition/src directory, and then opens the link to one of the following guides:
		\begin{itemize}
			\item \href{https://crtc.gc.ca/eng/internet/install.htm}{Canada’s Anti-Spam Legislation Requirements for Installing Computer Programs}
			\item \href{https://www.canada.ca/en/government/system/digital-government/digital-government-innovations/open-source-software/guide-for-publishing-open-source-code.html}{Guide for Publishing Open Source Code}
			\item \href{https://google.github.io/styleguide/jsguide.html}{Google JavaScript Style Guide}
			\item \href{https://google.github.io/styleguide/htmlcssguide.html}{Google HTML/CSS Style Guide}
			\item \href{https://material.io/design}{Material Design Guide}
		\end{itemize}
		The developer will go into each module found in the utrition/src directory, and will inspect the code to ensure that each rule outlined in Canada’s or Google’s guide is strictly adhered to. The developer is expected to review the code using the guide in chronological order.
		
		\item{NFR29\\}
		
		Type: Functional, Dynamic, and Manual.
		
		Initial State: A Durum Wheat Semolina developer has completed a module for Utrition, and has Pylint installed on their device.
		
		Input/Condition: The developer runs Pylint on their device.
		
		Output/Result: Pylint will notify the developer if there needs to be changes to the code to ensure Utrition adheres to the PEP8 Python coding conventions.
		
		How test will be performed: The developer will open their command prompt and travel to the directory with the finished module (utrition/src) by using “cd”. The developer runs Pylint on the finished module by typing “pylint finishedmodule.py” in their command prompt. The name “finishedmodule.py” is a placeholder for any future modules that Utrition will implement in the future. The command prompt will notify the developer to make any necessary changes.
		
		
	\end{enumerate}
	
	\subsection{Traceability Between Test Cases and Requirements}
	
	\wss{Provide a table that shows which test cases are supporting which
		requirements.}
	
	\begin{table}[H]
		\caption{Non-Functional Traceability Matrix}
		\begin{tabularx}{\linewidth}{|l|X|l|}
			\hline
			{\bf Requirement \#} & {\bf Description of Fit Criterion} & {\bf Test ID(s)}\\
			\hline
			LF1 & The distance between user interface components must exceed 20 pixels. & NFR1 \\
			\hline
			UH1 & The user will be able to navigate to the main menu from any page in 2 clicks. & NFR2 \\
			\hline
			UH2 & The user must be able to view every past inputted food item. & NFR3, NFR4 \\
			\hline
			UH3 & 90\% of a test panel of the target demographic should be able to upload their food and retrieve the corresponding nutritional facts in under 2 minutes. & NFR5 \\
			\hline
			UH4 & Every clickable object will have a symbol associated with it. Calories, fat, sodium, carbohydrates, sugar, and protein will have a symbol associated next to it.
			& NFR6 \\
			\hline
			UH5 & All backend calculations, such as identifying the uploaded food and retrieving the nutritional information, will not be displayed to the user. & NFR7 \\
			\hline
			UH6 & There will be no sound used by the Utrition web application.  & NFR8, NFR9 \\
			\hline
			PR1 & The time between the user’s click and displaying the new UI will not exceed 2 seconds. & NFR10 \\
			\hline
			PR2 & After uploading a picture, Utrition will always display the 
			identified food item within 10 seconds in a test with multiple 
			images. & NFR11 \\
			\hline
		\end{tabularx}
		\label{tab:Non-Functional Traceability}
	\end{table}
	\begin{table}[H]
		\begin{tabularx}{\textwidth}{|l|X|l|}
			\hline
			{\bf Requirement \#} & {\bf Description of Fit Criterion} & {\bf Test ID(s)}\\
			\hline
			PR3 & After identifying the correct food item, Utrition will always 
			display the nutritional facts within 5 seconds in a test with 
			multiple images. & \begin{tabular}[c]{@{}l@{}}
				NFR11,\\nutrition-\\display-\\performance-1\end{tabular}\\
			\hline
			PR4 & Error message should prompt user upon upload of an abnormal 
			image. & NFR20\\
			\hline
			PR5 & Error message should prompt user upon upload of an image 
			larger than 30MB. & NFR15\\
			\hline
			PR6 & Error message should prompt user upon upload of more than 3 
			images. & NFR13\\
			\hline
			PR7 & Error message should prompt user upon failure to identify 
			any food items in the uploaded image. & 
			\begin{tabular}[c]{@{}l@{}}NFR16,\\failed-\\identification-1 
			\end{tabular}\\
			\hline
			PR8 & Error message should prompt user to try again upon failure 
			to request nutritional info regarding food items in the uploaded image. & NFR17\\
			\hline
			PR9 & Error message should notify user that the nutritional 
			information pertaining to their food item does not exist. & NFR18\\
			\hline
			PR10 & Error message should notify user upon failure to access 
			user's past nutritional info. & NFR19\\
			\hline
			PR11 & Error message should notify user if user's past nutritional 
			info has been deleted. & NFR19\\
			\hline
			PR12 & Error message should prompt user if user's past nutritional 
			info cannot be depicted on a chart. & NFR19\\
			\hline
			PR13 & Request will time out after 20 seconds, and user will be 
			prompted that their device is not connected to the internet. & NFR17\\
			\hline
		\end{tabularx}
	\end{table}
	
	\begin{table}[H]
		%identification-accuracy-2
		\begin{tabularx}{\textwidth}{|l|X|l|}
			\hline
			{\bf Requirement \#} & {\bf Description of Fit Criterion} & {\bf Test ID(s)}\\
			\hline
			PR14 & In a test with multiple images, Utrition will identify and 
			display the correct food with an average accuracy of 70\%. & 
			\begin{tabular}[c]{@{}l@{}}NFR12\\ 
						identification-\\accuracy-2 \end{tabular}\\
			\hline
			PR15 & In a test with multiple images, Utrition will display the 
			correct nutritional information for an identified food 90\% of the 
			time. & \begin{tabular}[c]{@{}l@{}}NFR12\\ 
						identification-\\accuracy-1 \end{tabular}\\
			\hline
			PR16 & Since Utrition is a local web app, the user will always be allowed to use Utrition unless their device or internet connection is down. & NFR12\\
			\hline
			PR17 & The system will not allow more than 3 photos to be uploaded at once. & NFR11, NFR13\\
			\hline
			PR18 & Utrition is a local web app and does not depend on a server. & NFR14\\
			\hline
			PR19 & Utrition will be available to use on GitHub for an indefinite amount of time. & NFR14\\
			\hline
			OE1 & Users will be able to find and pull the repository on to their personal device. & NFR14\\
			\hline
			OE2 & 95\% of a test panel with the intended demographic will be able to install Utrition after reading the installation guide. & NFR21\\
			\hline
			OE3 & Utrition’s GitHub repository will receive no updates after final implementation. & NFR22\\
			\hline
			MS1 & Any backend developer is able to successfully upload new data to Utrition’s machine learning model at any time during development.  & NFR23\\
			\hline
		\end{tabularx}
	\end{table}
	
	\begin{table}[H]
		\begin{tabularx}{\textwidth}{|l|X|l|}
			\hline
			{\bf Requirement \#} & {\bf Description of Fit Criterion} & {\bf Test ID(s)}\\
			\hline
			MS2 & All users will be able to view the support manuals on how to install and use Utrition. & NFR21\\
			\hline
			MS3 & 100\% of the tested Windows, macOS, and Linux devices should be able to use all functionality provided for Utrition. & NFR24\\
			\hline
			SR1 & Anyone will be able to pull the open-source project from GitHub. & NFR14, NFR21 \\
			\hline
			SR2 & Utrition will save user data ever 3 minutes throughout the 
			application runtime. & NFR25\\
			\hline
			SR3 & Utrition will not contain personalized data from one user on another user's system. & NFR26\\
			\hline
			CP1 & 95\% of the test panel agrees that Utrition does not provide any culturally insensitive content to the user. & NFR27\\
			\hline
			LR1 & No malicious software will be present in Utrition’s GitHub repository. & NFR28\\
			\hline
			LR2 & Durum Wheat Semolina will not commit infractions when following Canada’s guide. & NFR28\\
			\hline
			LR3 & Pylint will be used throughout development to ensure every pull request contains properly formatted code. & NFR29\\
			\hline
			LR4 & Before every pull request, code will be reviewed to ensure no violations of the standards occur.  & NFR28\\
			\hline
			LR5 & Utrition's user interface will be reviewed to ensure all fonts and colours adhere to the style guide.  & NFR28\\
			\hline
		\end{tabularx}
	\end{table}
	
	\section{Unit Test Description}
	
	\wss{Reference your MIS (detailed design document) and explain your overall
		philosophy for test case selection.}  
	\wss{This section should not be filled in until after the MIS (detailed 
	design
		document) has been completed.}
	
	\subsection{Unit Testing Scope}
	
	\wss{What modules are outside of the scope.  If there are modules that are
		developed by someone else, then you would say here if you aren't 
		planning on
		verifying them.  There may also be modules that are part of your 
		software, but
		have a lower priority for verification than others.  If this is the 
		case,
		explain your rationale for the ranking of module importance.}
	
	\subsection{Tests for Functional Requirements}
	
	\wss{Subsets of the tests may be in related, so this section is divided into
		different areas.  If there are no identifiable subsets for the tests, this
		level of document structure can be removed.}
	
	\wss{Include a blurb here to explain why the subsections below
		cover the requirements.  References to the SRS would be good.}

	
	
	\wss{It would be nice to have a blurb here to explain why the subsections below
		cover the requirements.  References to the SRS would be good.  If a section
		covers tests for input constraints, you should reference the data constraints
		table in the SRS.}
	

	
	\subsection{Tests for Nonfunctional Requirements}
	
	\wss{If there is a module that needs to be independently assessed for
		performance, those test cases can go here.  In some projects, planning 
		for
		nonfunctional tests of units will not be that relevant.}
	
	\wss{These tests may involve collecting performance data from previously
		mentioned functional tests.}
	
	\subsection{Traceability Between Test Cases and Modules}
	
	
	\wss{Provide evidence that all of the modules have been considered.}
	
	\bibliographystyle{plainnat}
	
	\bibliography{../../refs/References}
	
	\newpage
	
	\section{Appendix}
	
	\subsection{Symbolic Parameters}
	
	N/A
	
	\subsection{Usability Survey Questions}
	
	\wss{This is a section that would be appropriate for some projects.}
	N/A
	
	\newpage{}
	\section*{Appendix --- Reflection}
	
	\subsection*{Skills}
	
	\subsubsection*{Usage of Pytest framework}
	Utrition's testing process will include unit tests, integration tests, and functional tests as a part of the verification and validation process. Pytest is a framework that is used for writing simple and scalable test cases for databases, APIs, and user interfaces. This framework can be used for anything from simple unit tests to complex functional tests. To master the use of Pytest, a team member can study the Pytest documentation, reference guides, and refer to the Pytest support community posts when necessary. Following tutorials and how-to guides can also prove to be helpful when learning how to use this framework.  
	
	\subsubsection*{CI/CD with automatic test cases in GitHub}
	
	During the development phase, new code should be continuously tested in 
	order to mitigate the introduction of bugs. GitHub Actions is an effective 
	tool that can be leveraged in order to automate testing every time new 
	code is added to the repository. By automating test cases, unit tests can 
	be executed whenever new pull requests are open to ensure new additions 
	have not broken any existing functionality. To implement this, team members 
	can consult online articles on the use of these CI/CD tools, or watch a 
	video walk-through on how to add automated testing to an existing GitHub 
	project.
	
	\subsubsection*{Usage of Jest framework} 
	Developers of Durum Wheat Semolina must be able to test the frontend development of Utrition which is made using JavaScript, HTML, and CSS. Jest is a testing framework that can write tests specifically for JavaScript, HTML, and CSS that can be run from the command line The team has decided to use this framework to ensure the correctness and usability of the code for Utrition. This tool will be able to automatically catch errors in the code so that developers do not need to manually go over the code themselves. To master using the Jest framework, team members can watch online tutorials or read Jest supporting documentation.
	
	\subsubsection*{Knowledge of and performing effective static code inspections}
	This skill describes the ability to understand the necessary components for a static code inspection and be able to perform one effectively. These inspections are done for all new code attempting to merge into the main branch of the team's GitHub through a PR. When static code inspections are not done properly, sloppy code will enter the main project which can cause errors as implementation progresses. Some approaches for learning what should be included in a code inspection and how to perform said inspection is by reading documentation on what should be done during a code inspection and asking more experienced team members of Utrition for guidance.
	
	\subsubsection*{Installing and using Pylint}
	When coding in Python, we do not have a visible compiler to catch syntax errors and bugs prior to running our program as we might have in Java or Haskell. Pylint is required for catching errors like this, as well as for ensuring consistent and syntactically correct code that follows established coding practices. Pylint can be installed through a \textbf{pip install pylint} command in the terminal, and can be used through the command \textbf{pylint filename.py} to evaluate a file. Pylint can be set up to ignore or only use certain coding rules in its extensive coding practice ruleset, making it customizable to our own specific uses. This skill can be developed by reading the documentation of Pylint and its syntax rules to see which ones are relevant to Utrition. Practicing coding with Pylint can also help acclimatize a team member and help develop good habits for using Pylint before pushing changes.
	
	\subsubsection*{Knowledge of Google Style Guides for JavaScript, HTML, and CSS}
	Frontend developers of Durum Wheat Semolina need to understand and implement the Google style guides for frontend technologies in order for all Utrition code to be consistent. By following these style guides, developers do not need to take extra time attempting to decipher unorganized code. Once the developers understand the Google style guides, they will be able to do static code reviews of the written modules. Therefore, they will now be equipped to complete the NFR28 test for the validation and verification of the project. One method for learning the style guides is by approaching the task like a school test. The person learning should take notes on the guide, and to quiz themselves at random intervals. Another way of acquiring the knowledge of the style guides is to use online video resources to help teach them the material.
	
	\subsection*{Teammate Specific Learning Approaches}
	\subsubsection*{Alexander Moica}
	Alexander would like to write code with greater consistency in its syntax and that follows established coding standards, even those he does not know about. To help him develop these skills and habits, he will incorporate Pylint into his existing projects and create personal coding projects that make use of Pylint from the very start. Alexander will learn to master using Pylint through practice since he learns fastest by making mistakes and remembering them in the future.
	
	\subsubsection*{Yasmine Jolly}
	Yasmine has had no previous experience working with the Jest testing framework and requires this knowledge for the testing of Utrition. She will be watching video tutorials on how to use and write effective test cases in Jest. This method of learning was chosen because Yasmine finds it effective to learn by watching examples. With videos, she can pause and rewind steps that she did not fully understand.
	
	\subsubsection*{Jeffrey Wang}
	
	Jeffrey has sparsely worked with automated testing frameworks during his 
	previous work experience, and seen the ways it has helped to prevent the 
	introduction of bugs. As a result, he is interested in leveraging CI/CD 
	tools to automate test cases, and set up mitigations to prevent the 
	introduction of bugs during the development of Utrition. In order to 
	complete this successfully, Jeffrey plans to read online tutorials about 
	the potential ways to automate test cases using GitHub Actions. He has chosen this strategy because the tutorials will walk him through the process step-by-step which he finds most effective for learning.
	
	\subsubsection*{Jack Theriault}
	Jack has chosen to develop his knowledge in the Google style guides. Jack has made this choice because he has never used any frontend coding technologies such as JavaScript, HTML, or CSS. Since he is using these languages for the first time, it is best that he develops proper coding style while working in the new languages. To become familiar with the style guides, Jack will read through the guides and quiz himself at regular intervals. He has chosen this learning strategy because he usually learns this way during school and has found it effective.
	
	\subsubsection*{Catherine Chen}
	
	Catherine has had limited experience when writing Python tests using Pytest. She'd like to develop her scalable test writing skills, which include unit tests and functional tests. Her goal is to become familiar with the Pytest framework in order to achieve her goals. Catherine has chosen to read Pytest documentation to learn this skill because of how easily she can access supporting documentation is online.
	
	\subsubsection*{Justina Srebrnjak}
	Justina has identified that she struggles with performing static code inspections on other team members' code. More specifically, she is confused about what should be inspected during these reviews (ex. code functionality, potential errors, coding conventions). To master this skill, Justina will seek guidance from fellow teammates who have performed many code inspections in the past (from co-op experiences). To learn most effectively, she will watch one of her teammates perform a static code inspection of another team member's work. She has chosen this strategy because she finds it helpful to watch an example be completed where she can take notes on important steps in the process. With these notes, she will replicate and refer to them when she performs her own reviews.
	
\end{document}
