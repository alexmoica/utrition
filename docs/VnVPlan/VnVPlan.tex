\documentclass[12pt, titlepage]{article}

\usepackage{booktabs}
\usepackage{tabularx}
\usepackage{hyperref}
\usepackage{float}
\hypersetup{
	colorlinks,
	citecolor=blue,
	filecolor=black,
	linkcolor=red,
	urlcolor=blue
}
\usepackage[round]{natbib}

%% Comments

\usepackage{color}

\newif\ifcomments\commentstrue %displays comments
%\newif\ifcomments\commentsfalse %so that comments do not display

\ifcomments
\newcommand{\authornote}[3]{\textcolor{#1}{[#3 ---#2]}}
\newcommand{\todo}[1]{\textcolor{red}{[TODO: #1]}}
\else
\newcommand{\authornote}[3]{}
\newcommand{\todo}[1]{}
\fi

\newcommand{\wss}[1]{\authornote{blue}{SS}{#1}} 
\newcommand{\plt}[1]{\authornote{magenta}{TPLT}{#1}} %For explanation of the template
\newcommand{\an}[1]{\authornote{cyan}{Author}{#1}}

%% Common Parts

\newcommand{\progname}{Software Engineering} % PUT YOUR PROGRAM NAME HERE
\newcommand{\authname}{Team 16, Durum Wheat Semolina
	\\ Alexander Moica
	\\ Yasmine Jolly
	\\ Jeffrey Wang
	\\ Jack Theriault
	\\ Catherine Chen
	\\ Justina Srebrnjak } % AUTHOR NAMES                 

\usepackage{hyperref}
    \hypersetup{colorlinks=true, linkcolor=blue, citecolor=blue, filecolor=blue,
                urlcolor=blue, unicode=false}
    \urlstyle{same}
                                


\begin{document}
	
	\title{Project Title: System Verification and Validation Plan for 
	\progname{}} 
	\author{\authname}
	\date{\today}
	
	\maketitle
	
	\pagenumbering{roman}
	
	\section{Revision History}
	
	\begin{tabularx}{\textwidth}{p{3cm}p{2cm}X}
		\toprule {\bf Date} & {\bf Version} & {\bf Notes}\\
		\midrule
		Date 1 & 1.0 & Notes\\
		Date 2 & 1.1 & Notes\\
		\bottomrule
	\end{tabularx}
	
	\newpage
	
	\tableofcontents
	
	\listoftables
	\wss{Remove this section if it isn't needed}
	
	\listoffigures
	\wss{Remove this section if it isn't needed}
	
	\newpage
	
	\section{Symbols, Abbreviations and Acronyms}
	
	\renewcommand{\arraystretch}{1.2}
	\begin{tabular}{l l} 
		\toprule		
		\textbf{symbol} & \textbf{description}\\
		\midrule 
		T & Test\\
		SRS & Software Requirements Specification\\
		MG & Module Guide\\
		MIS & Management Information System\\
		VnV & Verification and Validation\\
		SFWRENG 4G06& Software Engineering 4G06 course\\
		PR& Pull Request\\
		\bottomrule
	\end{tabular}\\
	
	\wss{symbols, abbreviations or acronyms --- you can simply reference the SRS
		\citep{SRS} tables, if appropriate}
	
	\wss{Remove this section if it isn't needed}
	
	\newpage
	
	\pagenumbering{arabic}
	
	This document ... \wss{provide an introductory blurb and roadmap of the
		Verification and Validation plan}
	
	\section{General Information}
	
	\subsection{Summary}
	
	\wss{Say what software is being tested.  Give its name and a brief overview 
	of
		its general functions.}
		The purpose behind this document is to provide clarity of the testing plan that the Utrition application will go through to ensure the quality of the final product. The document has been separated into the various types of testing that will be performed to outline what needs to be completed by the developers in the testing stage. The software being tested includes the Utrition application. This software consists of three main functions that are being tested such as image processing and identification, displaying an interface for the user, and writing to a database.
	
	\subsection{Objectives}
	
	\wss{State what is intended to be accomplished.  The objective will be 
	around
		the qualities that are most important for your project.  You might have
		something like: ``build confidence in the software correctness,''
		``demonstrate adequate usability.'' etc.  You won't list all of the 
		qualities,
		just those that are most important.}
		
		As mentioned in previous documents such as the SRS and the Hazard Analysis document there are requirements that are essential for the essence of the application which the proof of concept will therefore be focused on implementing. The proof of concept of the utrition application will consist a medium to high fidelity prototype that consists of various key requirements and the successful evasion of the potential hazards. There should additionally be a basic front-end implementation that will not be complete visually but will be able to perform every task specified in the requirements. 

		Requirements: Requirement FR3 that states, successful identification of foods captured in the user uploaded image, which should be proven for a subset of the set of food that the artificial intelligence will learn. Additionally, the requirement FR4 states, the system will be able to make an API call to an external database of nutrition facts, for the same subset of foods as listed earlier.  

		Significant risks: The prototype should also be able to detect and respond to the most common errors as listed in the failure mode table such as H1-1 where the user may be, uploading an image of the wrong type.  
	
	\subsection{Relevant Documentation}
	
	\wss{Reference relevant documentation.  This will definitely include your 
	SRS
		and your other project documents (design documents, like MG, MIS, 
		etc).  You
		can include these even before they are written, since by the time the 
		project
		is done, they will be written.}

		This document references mutiple other documents which have been listed below:
	
	\citet{SRS}
	\citet{DevelopmentPlan}
	\citet{MG}
	\citet{MIS}

	
	
	\section{Plan}
	
	\wss{Introduce this section.   You can provide a roadmap of the sections to
		come.}
	This section details Durum Wheat Semolina's approaches for verifying and validating Utrition's documentation and implementation. Specific documentation to be reviewed are the SRS, Design documents (MG and MIS), and VnV Plan. Additionally, the verification and validation of Utrition's implementation is discussed. Tools being used for verification including automated testing frameworks and linters are also detailed below.
	
	\subsection{Verification and Validation Team}
	
	\wss{Your teammates.  Maybe your supervisor.
		You shoud do more than list names.  You should say what each person's 
		role is
		for the project's verification.  A table is a good way to summarize 
		this information.}
	
	Each team member will be contributing to the verification and validation of Utrition's supporting documentation and implementation. Jack Theriault is the team's testing lead and can be deferred to for any queries regarding testing procedures. It is expected that all team members create the tests that correspond to their written code. In addition, each team member's specific role in the testing process is given below.
	
	\begin{table}[H]
		\centering
		\label{Table:Testing_Roles}
		\begin{tabular}{|p{3.2cm}|p{4cm}|p{5.5cm}|}
			\hline
			\textbf{Team Member} & \textbf{Role} & \textbf{Description}\\ \hline
			Alexander Moica& Dynamic Verification Lead& This role is responsible for leading the dynamic verification for Utrition's implementation.\\ \hline
			Yasmine Jolly& Software Validation Lead & This role is responsible for leading the validation effort of Utrition.\\ \hline
			Jeffrey Wang& Static Verification Lead & This role is responsible for leading the static verification for Utrition's implementation.\\ \hline
			Jack Theriault& VnV Plan Verification Lead& This role is responsible for leading the verification effort of the VnV plan.\\ \hline
			Catherine Chen& Design Verification Lead& This role is responsible for leading the verification effort of the design documents.\\ \hline
			Justina Srebrnjak& SRS Verification Lead& This role is responsible for leading the verification effort of the SRS. \\ \hline
		\end{tabular}
		\caption{Team Member Roles for Verification and Validation Testing}
	\end{table} 
	
	\subsection{SRS Verification Plan}
	
	\wss{List any approaches you intend to use for SRS verification.  This may 
	include
		ad hoc feedback from reviewers, like your classmates, or you may plan 
		for 
		something more rigorous/systematic.}
	
	\wss{Maybe create an SRS checklist?}
	
	To verify Utrition's SRS document, two approaches will be primarily used. Firstly, ad hoc reviews by fellow classmates are done to provide Durum Wheat Semolina with feedback from an outsider perspective. The rubric for the SRS is heavily used during this process. \\
	
	Once feedback from these reviews has been collected in addition to the feedback given by SFWRENG 4G06 TAs while marking version 0, the team will walk through the SRS and make any necessary changes. Since the SRS is critical when implementing Utrition, this review process will be done by all team members to ensure total coverage of possible improvements. The team will be instructed to go through all functional and non-functional requirements to identify any ambiguities or missing requirements. Once identified, the corresponding changes will be made. This walkthrough will be lead by Justina Srebrnjak who is in charge of SRS verification.

	\subsection{Design Verification Plan}
	
	\wss{Plans for design verification}
	
	\wss{The review will include reviews by your classmates}
	
	\wss{Create a checklists?}
	
	Utrition's design documents will be verified through a combination of peer reviews and documentation walkthroughs. Once version 0 of the design documents have been completed, students from SFWRENG 4G06 will perform ad hoc reviews based on the criteria given in the design rubric. These critiques will make Durum Wheat Semolina aware of any discrepencies found in the documentation.\\
	
	Secondly, documentation walkthroughs will be conducted by all team members to verify the completeness of the design documents. The primary purpose of these walkthroughs is to verify that all functional requirements listed in the SRS document are fulfilled through the proposed design. Any missing functionality will be identified through this process and amended as needed. This verification plan will be lead by Catherine Chen.
	
	\subsection{Verification and Validation Plan Verification Plan}
	
	\wss{The verification and validation plan is an artifact that should also 
	be verified.}
	
	\wss{The review will include reviews by your classmates}
	
	\wss{Create a checklists?}
	
	The verification plan for the VnV Plan document will consist of two techniques. Firstly, ad hoc reviews from the team's classmates in SFWRENG 4G06 will be conducted. These reviews will identify any discrepencies indicated by the VnV plan rubric. Their resulting feedback will contribute to any changes made to the document.\\
	
	In addition to peer reviews, the Utrition team will participate in a walkthrough of the document while referring to the SRS. During this review, the team will verify that every requirement found in the SRS has at least one test that will test its completeness. This review will be lead by Jack Theriault. 
	
	\subsection{Implementation Verification Plan}
	
	\wss{You should at least point to the tests listed in this document and the 
	unit
		testing plan.}
	
	\wss{In this section you would also give any details of any plans for 
	static verification of
		the implementation.  Potential techniques include code walkthroughs, 
		code
		inspection, static analyzers, etc.}
	
	Verifying Utrition's implementation will be done through a variety of dynamic and static verification techniques. In terms of dynamic methods, a variety of system and unit tests will be performed on the written code. More information on these tests can be found in sections 5 and 6 of this document. These tests will be created by the entire team. More specifically, the people who work on a certain functionality of Utrition will be responsible for creating the corresponding test cases. These tests will be automated and run on the system whenever new code is pushed to the team's master branch on GitHub. This ensures that new code does not create any regressions in the current system. Alexander Moica will be the dynamic verification lead which entails monitoring new test case additions and ensuring full testing coverage of new code which includes edge cases and error cases.\\
	
	Static verification will be done primarily through coding standards and code inspections. When writing code, each team member will be responsible for abiding by the coding guidelines outlined in Utrition's \href{https://github.com/jeff-rey-wang/utrition/blob/dff32f8ddc662d07db9bd74e0b3705aa657dae6e/docs/DevelopmentPlan/DevelopmentPlan.pdf}{Development Plan} in section 7. Python development will use the linter Pylint to enforce PEP8 standards. The Google style guides for JavaScript, HTML, and CSS will need to be followed manually. Additionally, coding inspections will be done for every new PR that is made to the team's master branch in GitHub. Before any PR is merged, at least one team member will be required to review the incoming code. During these reviews, team members will ensure complete functionality fulfillment inline with documentation, proper coding style, efficient implementation practices, and consistent naming practices for files, functions, and varibles. Jeffrey Wang will lead this static verification plan to ensure all coding inspections are completed and answer any questions regarding static verification.
	
	\subsection{Automated Testing and Verification Tools}
	
	\wss{What tools are you using for automated testing.  Likely a unit testing
		framework and maybe a profiling tool, like ValGrind.  Other possible 
		tools
		include a static analyzer, make, continuous integration tools, test 
		coverage
		tools, etc.  Explain your plans for summarizing code coverage metrics.
		Linters are another important class of tools.  For the programming 
		language
		you select, you should look at the available linters.  There may also 
		be tools
		that verify that coding standards have been respected, like flake9 for
		Python.}
	
	\wss{If you have already done this in the development plan, you can point to
		that document.}
	
	\wss{The details of this section will likely evolve as you get closer to the
		implementation.}
	
	The tools used for automated testing and verification have been previously outlined in the \href{https://github.com/jeff-rey-wang/utrition/blob/dff32f8ddc662d07db9bd74e0b3705aa657dae6e/docs/DevelopmentPlan/DevelopmentPlan.pdf}{Development Plan} for Utrition in sections 6 and 7. 
	
	\subsection{Software Validation Plan}
	
	\wss{If there is any external data that can be used for validation, you 
	should
		point to it here.  If there are no plans for validation, you should 
		state that
		here.}
	
	\wss{You might want to use review sessions with the stakeholder to check 
	that
		the requirements document captures the right requirements.  Maybe task 
		based
		inspection?}
	
	\wss{This section might reference back to the SRS verification section.}
	
	In order to validate Utrition's implementation, interviews will be scheduled with potential users. In these interviews, the interview conducter will walk through the functional requirements of the system with the interviewee. The interviewee will be asked for their feedback on the current requirements of the system. This feedback may include adding or removing specific functionality (i.e. altering requirements). Software validation will be lead by Yasmine Jolly. 
	
	\section{System Test Description}
	
	\subsection{Tests for Functional Requirements}
	The below areas cover all of the nine functional requirements outlined in the SRS, dividing the functional requirements into two distinct sections. The first section details user actions before the image processing steps and encompasses FR1 and FR2. The second section details user actions after the image processing steps and encompasses FR3, FR4, FR5, FR6, FR7, FR8, and FR9.
	
	\subsubsection{Prior to Image Processing}
	This section has two tests, with each corresponding to an action the user can take with our application, and likewise to FR1 and FR2 in the SRS. Testing that users can successfully submit one or multiple images is important to the core functionality of the other components of the application.
	
	\paragraph{Single image upload}
	
	\begin{enumerate}
		
		\item{one-upload-1\\}
		
		Control: Manual
		
		Initial State: No image in the system
		
		Input: An image file
		
		Output: The system accepts the image upload
		
		Test Case Derivation: The system should accept an uploaded image file from the user.
		
		How test will be performed: A locally stored image will be uploaded to the application through the image upload prompt. 
		
	\end{enumerate}
	
	\paragraph{Multi-image upload}
	
	\begin{enumerate}
		
		\item{multi-upload-1\\}
		
		Control: Manual
		
		Initial State: No image in the system
		
		Input: Three image files
		
		Output: The system accepts the image uploads
		
		Test Case Derivation: The system should accept multiple uploaded image files from the user at the same time.
		
		How test will be performed: Three locally stored images will be uploaded to the application through image upload prompts. 
		
	\end{enumerate}
	
	\subsubsection{Subsequent to Image Processing}
	This section has three tests, with each corresponding to an action the user can take with our application, and likewise to (FR3, FR4, FR5, FR6, FR8), (FR7, FR8), and (FR7, FR8, FR9) respectively in the SRS. These tests will test the actions the user can take after uploading an image, comprising of viewing past and present nutritional data in textual and graphical formats.
	
	\paragraph{Viewing the nutritional data of an uploaded food image}
	
	\begin{enumerate}
		
		\item{current-nutrition-1\\}
		
		Control: Automatic
		
		Initial State: Image(s) exists in system
		
		Input: Input signal from the image upload process
		
		Output: The nutritional data of the food
		
		Test Case Derivation: Primary black-box test for our system as a whole. The system will take the uploaded image(s), analyse them for the food they display, cross-reference them with a nutrition database, and display the information to the user.
		
		How test will be performed: The program will automatically test that the resulting nutrition data of a pre-uploaded image file matches that of the nutrition database entry for the food.
		
	\end{enumerate}
	
	\paragraph{Viewing textual nutritional data of logged foods}
	
	\begin{enumerate}
		
		\item{past-nutrition-text-1\\}
		
		Control: Automatic
		
		Initial State: Logged foods exist in system
		
		Input: Request to view nutritional data of logged foods as text
		
		Output: The nutritional data of past logged foods in a textual format
		
		Test Case Derivation: The system will keep a record of all analysed foods for the user, allowing the user to request to see the nutritional data of their past uploads. This data can be displayed in a human readable textual format.
		
		How test will be performed: The program will automatically test that a file of pre-logged foods can be processed to return their corresponding nutritional data as text.
		
	\end{enumerate}
	
	\paragraph{Viewing graphical nutritional data of logged foods}
	
	\begin{enumerate}
		
		\item{past-nutrition-graph-1\\}
		
		Control: Automatic
		
		Initial State: Logged foods exist in system
		
		Input: Request to view nutritional data of logged foods as graph
		
		Output: The nutritional data of past logged foods in a graphical format
		
		Test Case Derivation: The system will keep a record of all analysed foods for the user, allowing the user to request to see the nutritional data of their past uploads. This data can be displayed in a graphical format.
		
		How test will be performed: The program will automatically test that a file of pre-logged foods can be processed to return their corresponding nutritional data in the form of a graph.
		
	\end{enumerate}
	
	\subsection{Traceability Between Test Cases and Requirements}
	
	\begin{table}[]
	
	\begin{tabular}{|p{3.5cm}|p{6.5cm}|p{4.5cm}|}

	\hline
	\textbf{Requirement \#} & \textbf{Description}                                                                                                                                                                  & \textbf{Test ID(s)}                                                                                          \\ \hline
	FR1                     & The user must have the ability to upload a digital image of a standard image type to the system.                                                                                      & one-upload-1                                                                                                 \\ \hline
	FR2                     & The user shall have the ability to upload multiple digital images of standard image types to the system.                                                                              & multi-upload-1                                                                                               \\ \hline
	FR3                     & The system will be able to identify the type of food that is captured in an image.                                                                                                    & current-nutrition-1                                                                                          \\ \hline
	FR4                     & The system will be able to make an API call to an external database of nutrition facts for a variety of foods, including their macro-nutrients, micro-nutrients, and caloric details. & current-nutrition-1                                                                                          \\ \hline
	FR5                     & The system will be able to retrieve the nutrition facts for a specific food.                                                                                                          & current-nutrition-1                                                                                          \\ \hline
	FR6                     & The system will log the nutritional data of a food.                                                                                                                                   & current-nutrition-1                                                                                          \\ \hline
	FR7                     & The user will be able to access the nutritional data of previously logged foods.                                                                                                      & \begin{tabular}[c]{@{}l@{}}past-nutrition-text-1\\ past-nutrition-graph-1\end{tabular}                       \\ \hline
	FR8                     & The system will display the nutritional information of a food to the user.                                                                                                            & \begin{tabular}[c]{@{}l@{}}current-nutrition-1\\ past-nutrition-text-1\\ past-nutrition-graph-1\end{tabular} \\ \hline
	FR9                     & The system will display the history of logged nutritional data in a graph.                                                                                                            & past-nutrition-graph-1                                                                                       \\ \hline
	
	\end{tabular}
	
	\end{table}
	
	\subsection{Tests for Nonfunctional Requirements}
	
	\wss{The nonfunctional requirements for accuracy will likely just reference 
	the
		appropriate functional tests from above.  The test cases should mention
		reporting the relative error for these tests.  Not all projects will
		necessarily have nonfunctional requirements related to accuracy}
	
	\wss{Tests related to usability could include conducting a usability test 
	and
		survey.  The survey will be in the Appendix.}
	
	\wss{Static tests, review, inspections, and walkthroughs, will not follow 
	the
		format for the tests given below.}
	
	\subsubsection{Area of Testing1}
	
	\paragraph{Title for Test}
	
	\begin{enumerate}
		
		\item{test-id1\\}
		
		Type: Functional, Dynamic, Manual, Static etc.
		
		Initial State: 
		
		Input/Condition: 
		
		Output/Result: 
		
		How test will be performed: 
		
		\item{test-id2\\}
		
		Type: Functional, Dynamic, Manual, Static etc.
		
		Initial State: 
		
		Input: 
		
		Output: 
		
		How test will be performed: 
		
	\end{enumerate}
	
	\subsubsection{Area of Testing2}
	
	...
	
	\subsection{Traceability Between Test Cases and Requirements}
	
	\wss{Provide a table that shows which test cases are supporting which
		requirements.}
	
	\section{Unit Test Description}
	
	\wss{Reference your MIS (detailed design document) and explain your overall
		philosophy for test case selection.}  
	\wss{This section should not be filled in until after the MIS (detailed 
	design
		document) has been completed.}
	
	\subsection{Unit Testing Scope}
	
	\wss{What modules are outside of the scope.  If there are modules that are
		developed by someone else, then you would say here if you aren't 
		planning on
		verifying them.  There may also be modules that are part of your 
		software, but
		have a lower priority for verification than others.  If this is the 
		case,
		explain your rationale for the ranking of module importance.}
	
	\subsection{Tests for Functional Requirements}
	
	\wss{Subsets of the tests may be in related, so this section is divided into
		different areas.  If there are no identifiable subsets for the tests, this
		level of document structure can be removed.}
	
	\wss{Include a blurb here to explain why the subsections below
		cover the requirements.  References to the SRS would be good.}
	
	The functional requirements of the system will be broken down into general phases of the system's functionality: Image Processing, API Calls, Reading and Writing to Database, and Displaying to User Interface. These subsections cover the functional requirements outlined in the SRS document.
	
	\subsubsection{Image Processing}
	
	\wss{It would be nice to have a blurb here to explain why the subsections below
		cover the requirements.  References to the SRS would be good.  If a section
		covers tests for input constraints, you should reference the data constraints
		table in the SRS.}
	
	The following set of tests correlate FR1, FR2, and FR3 in the SRS document. Manual testing will be used, in which the tester will upload images according to the test plans to validate the proper response to the input. Automated testing will be used to verify the system's image identification process.
	
	\paragraph{Image Upload}
	
	\begin{enumerate}
		
		\item{image-upload-1\\}
		
		Control: Manual
		
		Initial State: No image in the system.
		
		Input: An image file of extension .png will be uploaded into the system.
		
		Output: The system will contain the uploaded image of type .png.
		
		Test Case Derivation: The system should save the inputted image, so that further processing can be done on the image.
		
		How test will be performed: An image of type .png will be saved on the tester's device. The tester will upload this image to the system.
		
		\item{image-upload-2\\}
		
		Control: Manual
		
		Initial State: No image in the system.
		
		Input: An image file of extension .jpg will be uploaded into the system.
		
		Output: The system will contain the uploaded image of type .jpg.
		
		Test Case Derivation: The system should save the inputted image, so that further processing can be done on the image.
		
		How test will be performed: An image of type .jpg will be saved on the tester's device. The tester will upload this image to the system.
		
		\item{image-upload-3\\}
		
		Control: Manual
		
		Initial State: No image in the system.
		
		Input: An image file of extension .jpeg will be uploaded into the system.
		
		Output: The system will contain the uploaded image of type .jpeg.
		
		Test Case Derivation: The system should save the inputted image, so that further processing can be done on the image.
		
		How test will be performed: An image of type .jpeg will be saved on the tester's device. The tester will upload this image to the system.
		
		\item{image-upload-4\\}
		
		Control: Manual
		
		Initial State: No image in the system.
		
		Input: A file of extension .txt will be uploaded into the system.
		
		Output: The system will not contain the uploaded file. An error message will be returned.
		
		Test Case Derivation: The system should only save the inputted image if it is of the proper file extension.
		
		How test will be performed: A file of type .txt will be saved on the tester's device. The tester will upload this file to the system.
		
	\end{enumerate}
	
	\paragraph{Image Identification}
	
	\begin{enumerate}
		
		\item{image-identification-1\\}
		
		Control: Automatic
		
		Initial State: System contains an image of extension .png type.
		
		Input: System is prompted to process the image.
		
		Output: The type of food that is captured in the image is returned as a string.
		
		Test Case Derivation: The system should identify the food present in an image, and return the name of the food item as a string.
		
		How test will be performed: The system identified item will be verified that it matches the true item contained in the image.
		
		\item{image-identification-2\\}
		
		Control: Automatic
		
		Initial State: System contains an image of extension .jpg type.
		
		Input: System is prompted to process the image.
		
		Output: The type of food that is captured in the image is returned as a string.
		
		Test Case Derivation: The system should identify the food present in an image, and return the name of the food item as a string.
		
		How test will be performed: The system identified item will be verified that it matches the true item contained in the image.
		
		\item{image-identification-3\\}
		
		Control: Automatic
		
		Initial State: System contains an image of extension .jpeg type.
		
		Input: System is prompted to process the image.
		
		Output: The type of food that is captured in the image is returned as a string.
		
		Test Case Derivation: The system should identify the food present in an image, and return the name of the food item as a string.
		
		How test will be performed: The system identified item will be verified that it matches the true item contained in the image.
		
	\end{enumerate}
	
	\subsubsection{API Calls}
	
	These tests cover  FR4 and FR5 in the SRS document. Automated unit tests will be used to verify that the system is able to send requests to the external API.
	
	\paragraph{Call and Fetch API Response}
	
	\begin{enumerate}
		
		\item{api-1\\}
		
		Control: Automatic
		
		Initial State: System on standby.
		
		Input: Food item name as a string.
		
		Output: JSON containing nutirition facts for the inputted food item.
		
		Test Case Derivation: A request will be made to the Nutritionix API with the food item name, which will return a response body containing the food item's nutrition facts.
		
		How test will be performed: The API response will be verified that the contents are as expected.
		
	\end{enumerate}
	
	\subsubsection{Reading and Writing to Database}
	
	The following tests cover  FR6 and FR7 in the SRS document. Automated unit tests will be used to verify that the system is log and read from the database.
	
	\paragraph{Logging Data}
	
	\begin{enumerate}
		
		\item{log-data-1\\}
		
		Control: Automatic
		
		Initial State: System contains nutritional data of a food item.
		
		Input: Request to log the data to the database.
		
		Output: Database is updated as the system writes the data to the database.
		
		Test Case Derivation: Nutritional data of a food will be saved for future reference. This is done by logging the data to the database.
		
		How test will be performed: After the system receives the request to update the database, the database will be verified that it is updated with the new data.
		
	\end{enumerate}
	
	\paragraph{Reading Data}
	
	\begin{enumerate}
		
		\item{read-data-1\\}
		
		Control: Automatic
		
		Initial State: Database contains nutritional data of a food item.
		
		Input: Request to fetch the data of a particular food item.
		
		Output: Nutritional data of the requested food item.
		
		Test Case Derivation: The system shall be able to fetch previously recorded data.
		
		How test will be performed: After the system receives the request to fetch the data of a food item, the returned data will be verified that it matches with the saved data in the database.
		
		\item{read-data-2\\}
		
		Control: Automatic
		
		Initial State: Database contains nutritional data of 5 different dates.
		
		Input: Request to fetch the data from the last 3 most recent dates.
		
		Output: A list of the nutritional data from the last 3 most recent dates.
		
		Test Case Derivation: The system shall be able to fetch previously recorded data. The system shall be able to fetch data from a range of dates.
		
		How test will be performed: The nutritional data from the 3 most recent dates will be verified if it matches the data in the database.
		
		\item{read-data-3\\}
		
		Control: Automatic
		
		Initial State: Database contains nutritional data.
		
		Input: Request to fetch the data of a particular food item that is not logged in the database.
		
		Output: No data is returned.
		
		Test Case Derivation: The system shall be able to fetch previously recorded data. This is an edge case for if the system is requested to read data from the database that does not exist. In this case, no data will be returned.
		
		How test will be performed: The output will be verified to be empty.
		
	\end{enumerate}
	
	\subsubsection{Displaying to User Interface}
	
	The following tests cover  FR8 and FR9 in the SRS document. Automated unit tests will be used to validate the generated visual display.
	
	\paragraph{Visualize Data}
	
	\begin{enumerate}
		
		\item{visualize-1\\}
		
		Control: Automatic
		
		Initial State: Nutritional data of a food item is contained in the system.
		
		Input: Request to display the nutritional information of a food item.
		
		Output: System will generate the HTML body to be displayed on the user interface.
		
		Test Case Derivation: The nutritional data of a particular food item shall be displayed on the user interface. The system will generate the frontend script to display this data.
		
		How test will be performed: The generated HTML body will be verified that it matches the expected output.
		
		\item{visualize-2\\}
		
		Control: Automatic
		
		Initial State: Database contains nutritional data of 5 different dates.
		
		Input: Request to display the past 5 dates of nutritional data in a graph.
		
		Output: System will generate the HTML body to be displayed on the user interface.
		
		Test Case Derivation: The history of logged nutritional data shall be displayed in a graph on the user interface. The system will generate the frontend script to display this data.
		
		How test will be performed: The generated HTML body will be verified that it matches the expected output.
		
	\end{enumerate}
	
	\subsection{Tests for Nonfunctional Requirements}
	
	\wss{If there is a module that needs to be independently assessed for
		performance, those test cases can go here.  In some projects, planning 
		for
		nonfunctional tests of units will not be that relevant.}
	
	\wss{These tests may involve collecting performance data from previously
		mentioned functional tests.}
	
	The nonfunctional requirements of the system will be broken down in the 
	same manner as done so in section 6.2 (Tests for Functional Requirements).
	
	\subsubsection{Image Processing}
	
	\paragraph{Image Upload}
	
	\begin{enumerate}
		
		\item{bad-image-1\\}
		
		Type: Automatic \wss{Functional, Dynamic, Manual, Automatic, Static 
		etc. Most will
			be automatic}
		
		Initial State: User is about to upload an image to the system.
		
		Input/Condition: Uploaded image is of an abnormal format.

		Output/Result: Error message is displayed prompting user of their 
		mistake.
		
		How test will be performed: Test will check if an error message element 
		has been rendered.
		
		\item{bad-image-2\\}
		
		Type: Automatic
		
		Initial State: User is about to upload an image to the system.
		
		Input/Condition: Uploaded image exceeds the maximum size.
		
		Output/Result: Error message is displayed prompting user of their 
		mistake.
		
		How test will be performed: Test will check if an error message element 
		has been rendered.
		
		
		\item{bad-image-3\\}

		Type: Automatic
		
		Initial State: User is about to upload images to the system.
		
		Input/Condition: More than three images are attempting to be uploaded.
		
		Output/Result: Error message is displayed prompting user of their 
		mistake.
		
		How test will be performed: Test will check if an error message element 
		has been rendered.
	\end{enumerate}

	\paragraph{Image Identification}
	
	\begin{enumerate}
		
		\item{failed-identification-1\\}
		
		Type: Automatic
		
		Initial State: System is identifying food objects within the user's 
		uploaded image(s).
		
		Input/Condition: Uploaded image.
		
		Output/Result: Error message is displayed prompting user of a failed 
		identification.
		
		How test will be performed: Test will check if an error message element 
		has been rendered.
		
		\item{image-identification-accuracy-1\\}
		
		Type: Automatic
		
		Initial State: System contains a set of images; all with foods, and of 
		extension .jpg, .png, or .jpeg, in accordance with scenarios outline in 
		image-identification-1,2,3.
		
		Input/Condition: System prompted to process image, for each image 
		within the provided set.
		
		Output/Result: For each image, the system should identify the food 
		present in an image, and return the name of the food item as a string.
		
		How test will be performed: The test will check if among all the 
		provided images, more than 90\% of the images have been assessed 
		correctly.
		
		\item{identification-accuracy-2\\}
		
		Type: Automatic
		
		Initial State: System contains a set of images; some with and others 
		without foods, and of extension .jpg, .png, or 
		.jpeg, in accordance with scenarios outline in 
		image-identification-1,2,3.
		
		Input/Condition: System prompted to process image, for each image 
		within the provided set.
		
		Output/Result: The system should identify the food present in an image, 
		and return the name of the food item as a string, and prompt the user 
		if no food is detected.
		
		How test will be performed: The test will check if among all the 
		provided images, more than 70\% of the images have been assessed 
		correctly.
		
		\item{identification-performance-2\\}
		
		Type: Automatic
		
		Initial State: System contains a set of images; some with and others 
		without foods, and of extension .jpg, .png, or 
		.jpeg, in accordance with scenarios outline in 
		image-identification-1,2,3.
		
		Input/Condition: System prompted to process image, for each image 
		within the provided set.
		
		Output/Result: The system should identify the food present in an image, 
		and return the name of the food item as a string, and prompt the user 
		if no food is detected.
		
		How test will be performed: The test will check if among all the 
		provided images, all requests were completed under 10 seconds each.
	\end{enumerate}
	\subsubsection{Displaying to User Interface}
	
	\paragraph{Visualize Data}
	\begin{enumerate}
		\item{visualization-performance-1\\}
		
		Type: Automatic
		
		Initial State: Nutritional data of a food item is contained in the 
		system.
		
		Input/Condition: Request to display the nutritional information of a 
		food item.
		
		Output/Result: System will generate the HTML body to be displayed on 
		the user 
		interface.
		
		How test will be performed: The test will measure if this interaction 
		was completed in 5 seconds or less.
		
		\item{visualize-performance-2\\}
		
		Control: Automatic
		
		Initial State: Database contains nutritional data of 5 different dates.
		
		Input/Condition: Request to display the past 5 dates of nutritional 
		data in a graph.
		
		Output/Result: System will generate the HTML body to be displayed on 
		the user interface.
		
		How test will be performed: The test will measure if this interaction 
		was completed in 5 seconds or less.
		
	\end{enumerate}
	
	\subsection{Traceability Between Test Cases and Modules}
	
	\wss{Provide evidence that all of the modules have been considered.}
	
	\bibliographystyle{plainnat}
	
	\bibliography{../../refs/References}
	
	\newpage
	
	\section{Appendix}
	
	This is where you can place additional information.
	
	\subsection{Symbolic Parameters}
	
	The definition of the test cases will call for SYMBOLIC\_CONSTANTS.
	Their values are defined in this section for easy maintenance.
	
	\subsection{Usability Survey Questions?}
	
	\wss{This is a section that would be appropriate for some projects.}
	
	\newpage{}
	\section*{Appendix --- Reflection}
	
	\subsection*{Skills}
	\subsubsection{CI/CD with automatic test cases in GitHub}
	
	During the development phase, new code should be continuously tested in 
	order to mitigate the introduction of bugs. GitHub Actions is an effective 
	tool that can be leveraged in order to automate the testing every time new 
	code is added to the repository. By automating test cases, unit tests can 
	be executed whenever new pull requests are open to ensure new additions 
	have not broken any existing functionality. To implement this, team members 
	can consult online articles on the use of these CI/CD tools, or watch a 
	video walk-through on how to add automated testing to an existing GitHub 
	project.
	
	\subsubsection*{Knowledge of and performing effective static code inspections}
	This skill describes the ability to understand the necessary components for a static code inspection and be able to perform one effectively. These inspections are done for all new code attempting to merge into the master branch of the team's GitHub through a PR. When static code inspections are not done properly, sloppy code will enter the main project which can cause errors as implementation progresses. Some approaches for learning what should be included in a code inspection and how to perform said inspection is by reading documentation on what should be done during a code inspection and asking more experience team members of Utrition for guidance.
	
	\subsubsection*{Installing and using Pylint}
	When coding in Python, we do not have a visible compiler to catch syntax errors and bugs prior to running our program as we might have in Java or Haskell. Pylint is required for catching errors like this, as well as for ensuring consistent and syntactically-correct code following established coding practices. Pylint can be installed through a \textbf{pip install pylint} command in the terminal, and can be used through the command \textbf{pylint filename.py} to evaluate a file. Pylint can be set up to ignore or only use certain coding rules in its extensive coding practice rule-set, making it customizable to our own specific uses. This skill can be developed by reading the documentation of Pylint and its syntax rules to see which ones are relevant to your own project. Practicing coding with Pylint can also help acclimatize you and help develop good habits for using Pylint before pushing changes.
	
	\subsection*{Teammate Specific Learning Approaches}
	\subsubsection*{Alexander Moica}
	Alexander would like to write code with greater consistency in its syntax as well as code that follows established coding standards, even those that he does not know about. To help him develop these skills and habits, he will incorporate Pylint into his existing projects and create personal coding projects that make use of Pylint from the very start.
	
	\subsubsection*{Yasmine Jolly}
	
	\subsubsection*{Jeffrey Wang}
	
	Jeffrey has sparsely worked with automated testing frameworks during his 
	previous work experience, and seen the ways it has helped to prevent the 
	introduction of bugs. As a result, he is interested in leveraging CI/CD 
	tools to automate test cases, and set up mitigations to prevent the 
	introduction of bugs during the development of Utrition. In order to 
	complete this successfully, Jeffrey plans to read online tutorials about 
	the potential ways to automate test cases using GitHub Actions.
	
	\subsubsection*{Jack Theriault}
	
	\subsubsection*{Catherine Chen}
	
	\subsubsection*{Justina Srebrnjak}
	Justina has identified that she struggles with performing static code inspections on other team members' code. More specifically, she is confused about what should be inspected during these reviews (ex. code functionality, potential errors, coding conventions). To master this skill, Justina will seek guidance from fellow teammates who have performed many code inspections in the past (from co-op experiences). To most effectively learn, she will watch one of her teammates perform a static code inspection of another team member's work. She has chosen this strategy because she finds it helpful to watch an example be completed where she can take notes on important steps in the process. With these notes, she will replicate and refer to them when she performs her own reviews.
	
\end{document}
