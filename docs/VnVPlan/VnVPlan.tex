\documentclass[12pt, titlepage]{article}

\usepackage{booktabs}
\usepackage{tabularx}
\usepackage{hyperref}
\usepackage{float}
\hypersetup{
	colorlinks,
	citecolor=blue,
	filecolor=black,
	linkcolor=red,
	urlcolor=blue
}
\usepackage[round]{natbib}

%% Comments

\usepackage{color}

\newif\ifcomments\commentstrue %displays comments
%\newif\ifcomments\commentsfalse %so that comments do not display

\ifcomments
\newcommand{\authornote}[3]{\textcolor{#1}{[#3 ---#2]}}
\newcommand{\todo}[1]{\textcolor{red}{[TODO: #1]}}
\else
\newcommand{\authornote}[3]{}
\newcommand{\todo}[1]{}
\fi

\newcommand{\wss}[1]{\authornote{blue}{SS}{#1}} 
\newcommand{\plt}[1]{\authornote{magenta}{TPLT}{#1}} %For explanation of the template
\newcommand{\an}[1]{\authornote{cyan}{Author}{#1}}

%% Common Parts

\newcommand{\progname}{Software Engineering} % PUT YOUR PROGRAM NAME HERE
\newcommand{\authname}{Team 16, Durum Wheat Semolina
	\\ Alexander Moica
	\\ Yasmine Jolly
	\\ Jeffrey Wang
	\\ Jack Theriault
	\\ Catherine Chen
	\\ Justina Srebrnjak } % AUTHOR NAMES                 

\usepackage{hyperref}
    \hypersetup{colorlinks=true, linkcolor=blue, citecolor=blue, filecolor=blue,
                urlcolor=blue, unicode=false}
    \urlstyle{same}
                                


\begin{document}
	
	\title{\progname{}: System Verification and Validation Plan} 
	\author{\authname}
	\date{\today}
	
	\maketitle
	
	\pagenumbering{roman}
	
	\section{Revision History}
	
	\begin{tabularx}{\textwidth}{p{3cm}p{2cm}X}
		\toprule {\bf Date} & {\bf Version} & {\bf Notes}\\
		\midrule
		11/02/2022 & 1.0 & Completed Version 1\\
		07/03/2023 & 2.0 & Added New Tests To Account For New FRs\\
		03/04/2023 & 3.0 & Added New Tests To Account For New FRs, Updated General Information Section, Updated Traceability Matrix\\
		\bottomrule
	\end{tabularx}
	
	\newpage
	
	\tableofcontents
	
	\listoftables
	\wss{Remove this section if it isn't needed}
	
	\newpage
	
	\section{Symbols, Abbreviations and Acronyms}
	
	\renewcommand{\arraystretch}{1.2}
	\begin{tabular}{l l} 
		\toprule		
		\textbf{symbol} & \textbf{description}\\
		\midrule 
		App & Application\\
		CT & Cultural Test\\
		LF & Look and Feel Test\\
		LT & Legal Test\\
		MG & Module Guide\\
		MIS & Module Interface Specification\\
		MS & Maintainability and Support\\
		OE & Operational and Environmental Test\\
		PR& Pull Request\\
		PT& Performance Test\\
		SFWRENG 4G06& Software Engineering 4G06 course\\
		SRS & Software Requirements Specification\\
		ST & Security Test\\
		UH & Usability and Humanity Test\\
		UI & User Interface\\
		VnV & Verification and Validation\\
		\bottomrule
	\end{tabular}\\
	
	\wss{symbols, abbreviations or acronyms --- you can simply reference the SRS
		\citep{SRS} tables, if appropriate}
	
	\wss{Remove this section if it isn't needed}
	
	\newpage
	
	\pagenumbering{arabic}
	
	This document depicts the Verification and Validation Plan for the Utrition project. This includes verifying both implementation and documentation. The document will depict the plans for reviewing all critical documentation along with the system and unit tests to be used for code verification.  \wss{provide an introductory blurb and roadmap of the
		Verification and Validation plan}
	
	\section{General Information}
	
	\subsection{Summary}
	
	\wss{Say what software is being tested.  Give its name and a brief overview 
	of
		its general functions.}
		The purpose behind this document is to provide the testing plan that Utrition will undergo during its development period. This plan will ensure a high-quality final product. Utrition is an application that allows individuals to discover the nutritional value of the foods they are consuming, track their previously eaten meals, and provide dietary insight on given user information. This software consists of eight main functions that are being tested such as image upload, image processing, displaying nutritional information, text upload, voice upload, profile page display, settings page display, and the taskbar for navigation.
	
	\subsection{Objectives}
	
	\wss{State what is intended to be accomplished.  The objective will be 
	around
		the qualities that are most important for your project.  You might have
		something like: ``build confidence in the software correctness,''
		``demonstrate adequate usability.'' etc.  You won't list all of the 
		qualities,
		just those that are most important.}
		
		The objectives for Utrition's VnV plan is to confirm that the implementation and documentation is as effective as possible to produce the best final product. Firstly, the developers of Utrition would like to ensure the correctness of the application's internal processes. The team must ensure that the artificial intelligence can correctly identify the food that is depicted in the uploaded image. The team must also ensure that text and voice upload methods are properly made, so that unusual inputs can still be understood by the system. The output the nutritional information must also be correct. Since the application is health-based, all nutritional facts must be correct since users will be using this information to influence their eating habits. Additionally, the Utrition application must be robust in order to deal with common and uncommon errors that may potentially occur. One such error is the user uploading an image of the wrong type. All tests in this document are written to ensure that the code meets these objectives. 
	
	\subsection{Relevant Documentation}
	
	\wss{Reference relevant documentation.  This will definitely include your 
	SRS
		and your other project documents (design documents, like MG, MIS, 
		etc).  You
		can include these even before they are written, since by the time the 
		project
		is done, they will be written.}

		This document references multiple other documents that are listed below:
		
		\begin{itemize}
			\item SRS, \citet{SRS}
			\item Development Plan, \citet{DevelopmentPlan}
			\item MG, \citet{MG}
			\item MIS, \citet{MIS}
		\end{itemize}
	
	\section{Plan}
	
	\wss{Introduce this section.   You can provide a roadmap of the sections to
		come.}
	This section details Durum Wheat Semolina's approaches for verifying and validating Utrition's documentation and implementation. Specific documentation to be reviewed are the SRS, Design documents (MG and MIS), and VnV Plan. Additionally, the verification and validation of Utrition's implementation is discussed. Tools being used for verification including automated testing frameworks and linters are also detailed below.
	
	\subsection{Verification and Validation Team}
	
	\wss{Your teammates.  Maybe your supervisor.
		You shoud do more than list names.  You should say what each person's 
		role is
		for the project's verification.  A table is a good way to summarize 
		this information.}
	
	Each team member will be contributing to the verification and validation of Utrition's supporting documentation and implementation. Jack Theriault is the team's testing lead and can be deferred to for any queries regarding verification procedures. It is expected that all team members create the tests that correspond to their written code. In addition, each team member's specific role in the verification and validation process is given below.
	
	\begin{table}[H]
		\centering
		\label{Table:Testing_Roles}
		\begin{tabular}{|p{3.2cm}|p{4cm}|p{5.5cm}|}
			\hline
			\textbf{Team Member} & \textbf{Role} & \textbf{Description}\\ \hline
			Alexander Moica& Dynamic Verification Lead& This role is responsible for leading the dynamic verification for Utrition's implementation.\\ \hline
			Yasmine Jolly& Software Validation Lead & This role is responsible for leading the validation effort of Utrition.\\ \hline
			Jeffrey Wang& Static Verification Lead & This role is responsible for leading the static verification for Utrition's implementation.\\ \hline
			Jack Theriault& VnV Plan Verification Lead& This role is responsible for leading the verification effort of the VnV plan.\\ \hline
			Catherine Chen& Design Verification Lead& This role is responsible for leading the verification effort of the design documents.\\ \hline
			Justina Srebrnjak& SRS Verification Lead& This role is responsible for leading the verification effort of the SRS. \\ \hline
		\end{tabular}
		\caption{Team Member Roles for Verification and Validation Testing}
	\end{table} 
	
	\subsection{SRS Verification Plan}
	
	\wss{List any approaches you intend to use for SRS verification.  This may 
	include
		ad hoc feedback from reviewers, like your classmates, or you may plan 
		for 
		something more rigorous/systematic.}
	
	\wss{Maybe create an SRS checklist?}
	
	To verify Utrition's SRS document, two approaches will be primarily used. Firstly, ad hoc reviews by fellow classmates are done to provide Durum Wheat Semolina with feedback from an outsider perspective. The rubric for the SRS is heavily used during this process. \\
	
	Once feedback from these reviews has been collected in addition to the feedback given by SFWRENG 4G06 TAs while marking version 0, the team will walk through the SRS and make any necessary changes. Since the SRS is critical when implementing Utrition, this review process will be done by all team members to ensure total coverage of possible improvements. The team will be instructed to go through all functional and non-functional requirements to identify any ambiguities or missing requirements. Once identified, the corresponding changes will be made. This walkthrough will be led by Justina Srebrnjak who is in charge of SRS verification.

	\subsection{Design Verification Plan}
	
	\wss{Plans for design verification}
	
	\wss{The review will include reviews by your classmates}
	
	\wss{Create a checklists?}
	
	Utrition's design documents will be verified through a combination of peer reviews and documentation walkthroughs. Once version 0 of the design documents has been completed, students from SFWRENG 4G06 will perform ad hoc reviews based on the criteria given in the design rubric. These critiques will make Durum Wheat Semolina aware of any discrepancies found in the documentation.\\
	
	Secondly, documentation walkthroughs will be conducted by all team members to verify the completeness of the design documents. The primary purpose of these walkthroughs is to verify that all functional requirements listed in the SRS document are fulfilled through the proposed design. Any missing functionality will be identified through this process and amended as needed. This verification plan will be led by Catherine Chen.
	
	\subsection{Verification and Validation Plan Verification Plan}
	
	\wss{The verification and validation plan is an artifact that should also 
	be verified.}
	
	\wss{The review will include reviews by your classmates}
	
	\wss{Create a checklists?}
	
	The verification plan for the VnV plan document will consist of two techniques. Firstly, ad hoc reviews from the team's classmates in SFWRENG 4G06 will be conducted. These reviews will identify any discrepancies indicated by the VnV plan rubric. Their resulting feedback will contribute to any changes made to the document.\\
	
	In addition to peer reviews, the Utrition team will participate in a walkthrough of the document while referring to the SRS. During this review, the team will verify that every requirement found in the SRS has at least one test that will test its completeness. This review will be led by Jack Theriault. 
	
	\subsection{Implementation Verification Plan}
	
	\wss{You should at least point to the tests listed in this document and the 
	unit
		testing plan.}
	
	\wss{In this section you would also give any details of any plans for 
	static verification of
		the implementation.  Potential techniques include code walkthroughs, 
		code
		inspection, static analyzers, etc.}
	
	Verifying Utrition's implementation will be done through a variety of dynamic and static verification techniques. In terms of dynamic methods, a variety of system and unit tests will be performed on the written code. More information on these tests can be found in sections 5 and 6 of this document. These tests will be created by the entire team. More specifically, the people who work on a certain functionality of Utrition will be responsible for creating the corresponding test cases. These tests will be automated and run on the system whenever new code is pushed to the team's main branch on GitHub. This ensures that new code does not create any regressions in the current system. Alexander Moica will be the dynamic verification lead which entails monitoring new test case additions and ensuring full testing coverage of new code which includes edge cases and error cases.\\

	Static verification will be done primarily through coding standards and code inspections. When writing code, each team member will be responsible for abiding by the coding guidelines outlined in Utrition's \href{https://github.com/jeff-rey-wang/utrition/blob/dff32f8ddc662d07db9bd74e0b3705aa657dae6e/docs/DevelopmentPlan/DevelopmentPlan.pdf}{Development Plan} in section 7. Python development will use the linter Pylint to enforce PEP8 standards. The Google style guides for JavaScript, HTML, and CSS will need to be followed manually. Additionally, coding inspections will be done for every new PR that is made to the team's main branch in GitHub. Before any PR is merged, at least one team member will be required to review the incoming code. During these reviews, team members will ensure complete functionality fulfillment inline with documentation, proper coding style, efficient implementation choices, and consistent naming practices for files, functions, and variables. Jeffrey Wang will lead this static verification plan to ensure all coding inspections are completed and answer any questions regarding static verification.
	
	\subsection{Automated Testing and Verification Tools}
	
	\wss{What tools are you using for automated testing.  Likely a unit testing
		framework and maybe a profiling tool, like ValGrind.  Other possible 
		tools
		include a static analyzer, make, continuous integration tools, test 
		coverage
		tools, etc.  Explain your plans for summarizing code coverage metrics.
		Linters are another important class of tools.  For the programming 
		language
		you select, you should look at the available linters.  There may also 
		be tools
		that verify that coding standards have been respected, like flake9 for
		Python.}
	
	\wss{If you have already done this in the development plan, you can point to
		that document.}
	
	\wss{The details of this section will likely evolve as you get closer to the
		implementation.}
	
	The tools used for automated testing and verification have been previously outlined in the \href{https://github.com/jeff-rey-wang/utrition/blob/dff32f8ddc662d07db9bd74e0b3705aa657dae6e/docs/DevelopmentPlan/DevelopmentPlan.pdf}{Development Plan} for Utrition in sections 6 and 7. 
	
	\subsection{Software Validation Plan}
	
	\wss{If there is any external data that can be used for validation, you 
	should
		point to it here.  If there are no plans for validation, you should 
		state that
		here.}
	
	\wss{You might want to use review sessions with the stakeholder to check 
	that
		the requirements document captures the right requirements.  Maybe task 
		based
		inspection?}
	
	\wss{This section might reference back to the SRS verification section.}
	
	In order to validate Utrition's implementation, interviews will be scheduled with potential users. In these interviews, the interview conductor will walk through the functional requirements of the system with the interviewee. The interviewee will be asked for their feedback on the current requirements of the system. This feedback may include adding or removing specific functionality (i.e. altering requirements). Software validation will be led by Yasmine Jolly. 
	
	\section{System Test Description}
	
	\subsection{Tests for Functional Requirements}
	The below areas cover all the twenty functional requirements outlined in the SRS, dividing the functional requirements into seven distinct sections. The first section details user actions associated with image upload and encompasses FR1, FR2, FR3, FR4, FR7, and FR15. The second section details image processing done by the system which encompasses FR3, FR4, FR5, FR6, and FR7. The third section outlines the user steps to view nutritional data which correlates to groupings of functional requirements. The fourth section details the ability for the user to upload food items through text and encompasses FR3, FR4, FR7, FR8, FR9, FR14, and FR15. The fifth section details the ability for the user to upload food items through speech and encompasses FR3, FR4, FR7, FR10, FR11, FR12, FR13, FR14, and FR15. The sixth section showcases what must be displayed when a user is viewing their profile and encompasses FR5, FR6, FR7, FR16, FR17, FR18, FR19, FR21, FR27, and FR28. The seventh section details Utrition's ability to save the user's information to display relevant facts about their well-being and diet. The seventh section encompasses FR22, FR23, FR24, FR25, and FR26. The last section details tests to ensure the navigation task bar is always present and allows users to freely travel between pages. The last section encompasses FR20.
	
	\subsubsection{Image Upload}
	This section has four tests, with each corresponding to an image upload action the user can take with Utrition. The tests deal with FR1, FR2, FR3, FR4, FR7, and FR15 in the SRS. 
	Testing that users can successfully submit one image is 
	important to the core functionality of the other components of the 
	application.
	
	\paragraph{Single image upload}
	
	\begin{enumerate}
		
		\item{one-upload-1\\}
		
		Control: Functional, Dynamic, and Manual.
		
		Initial State: User is on the ``Upload" page with ``Image Upload" selected.
		
		Input: A User clicks on ``Select Image" and selects an the image file ``apple.jpeg" found in the utrition/test/testPhotos directory. User clicks on ``Text Upload" and then clicks on ``Image Upload" and then clicks on the ``Submit" button.
		
		Output: The system accepts the image upload and returns the nutritional information for an apple. 
		
		Test Case Derivation: User should be able to view the nutritional information of an apple on their screen.
		
		How test will be performed: The tester will open Utrition and will click on ``Upload" found in the taskbar. The tester will click on ``Image Upload" and then ``Select Image". The tester selects ``apple.jpeg" found in the utrition/test/testPhotos directory. The tester clicks on the ``Image Upload" button and then clicks on the ``Submit" button.
		
		\item{one-upload-2\\}
		
		Control: Functional, Dynamic, and Manual.
		
		Initial State: No image in the system. User is on the ``Upload" page with ``Image Upload" selected.
		
		Input: A User clicks on the ``Submit" button.
		
		Output: No nutritional information will be returned to the user.
		
		Test Case Derivation: The system should not break when the user does not submit an image.
		
		How test will be performed: The tester will open Utrition and clicks on ``Upload" found in the taskbar. The tester clicks on ``Image Upload" and then the ``Submit" button.
		
		\item{one-upload-auto-1\\}
		
		Control: Functional, Dynamic, Automatic.
		
		Initial State: The system contains a generated JPG image of red pixels with dimensions 128 x 128.
		
        Input: The generated image is passed as an argument to interface.open(), with the result saved to the variable 'result'.

		Output: The system accepts the image upload.
		
		Test Case Derivation: The system should accept an uploaded image file from the user.
		
		How test will be performed: The generated image is passed to interface.open, which is the starting point for the machine learning image classification pipeline. If the test returns a value it means the pipeline executed successfully for the passed file.
		
		\item{one-upload-auto-2\\}
		
		Control: Functional, Dynamic, Automatic.
		
		Initial State: The system contains a generated txt file with text ``test content".
		
		Input: The generated text file is passed as an argument to interface.open().
		
		Output: The system will not contain the uploaded file. An error message 
		will be returned.
		
		Test Case Derivation: The system should only save the inputted image if 
		it is of the proper file extension.
		
		How test will be performed: The generated text file is passed to interface.open, which is the starting point for the machine learning image classification pipeline. If the test throws an UnidentifiedImageError that means that the format of the data passed is not valid for image classification.

	\end{enumerate}
	
	\subsubsection{Image Processing}
	This section contains tests that relate to the processing of an image 
	uploaded by the user. Tests correspond to actions that will be taken by the 
	application during image processing. Relevant functional requirements 
	include FR3, FR4, FR5, FR6, and FR7.
	
	\paragraph{Image Identification}
	
	\begin{enumerate}
		
		\item{image-identification-1\\}
		
		Control: Functional and Automatic.
		
		Initial State: System contains an image of extension .png type.
		
		Input: System is prompted to process the image.
		
		Output: The type of food that is captured in the image is returned as a 
		string.
		
		Test Case Derivation: The system should identify the food present in an 
		image, and return the name of the food item as a string.
		
		How test will be performed: The system identified item will be verified 
		that it matches the true item contained in the image.
		
		\item{image-identification-2\\}
		
		Control: Functional and Automatic.
		
		Initial State: System contains an image of extension .jpg type.
		
		Input: System is prompted to process the image.
		
		Output: The type of food that is captured in the image is returned as a 
		string.
		
		Test Case Derivation: The system should identify the food present in an 
		image, and return the name of the food item as a string.
		
		How test will be performed: The system identified item will be verified 
		that it matches the true item contained in the image.
		
		\item{image-identification-3\\}
		
		Control: Functional and Automatic.
		
		Initial State: System contains an image of extension .jpeg type.
		
		Input: System is prompted to process the image.
		
		Output: The type of food that is captured in the image is returned as a 
		string.
		
		Test Case Derivation: The system should identify the food present in an 
		image, and return the name of the food item as a string.
		
		How test will be performed: The system identified item will be verified 
		that it matches the true item contained in the image.
		
	\end{enumerate}
	
	\paragraph{Call and Fetch API Response}
	
	\begin{enumerate}
		
		\item{api-1\\}
		
		Control: Functional and Automatic.
		
		Initial State: System on standby.
		
		Input: Food item name as a string.
		
		Output: JSON containing nutrition facts for the inputted food item.
		
		Test Case Derivation: A request will be made to the Nutritionix API 
		with the food item name, which will return a response body containing 
		the food item's nutrition facts.
		
		How test will be performed: The API response will be verified that the 
		contents are as expected.
		
		\item{axios-call-1\\}
		
		Control: Functional and Automatic.
		
		Initial State: The text upload functionality has not yet been used.
		
		Input: The tester will attempt to make an axios call
		
		Output: The tester will receive an http request back from the backend.
		
		Test Case Derivation: The testing framework should reply with a not null value to ensure that there has been a successful call made to the backend
		
		How test will be performed: Tester will create mocked text upload function to test to see if the front end can successfully make and receive http calls.
		
	\end{enumerate}
	
	\paragraph{Logging Data}
	
	\begin{enumerate}
		
		\item{log-data-read-data-1\\}
		
		Control: Functional, Unit and Automatic.
		
		Initial State: The system contains a list of one object of sample JSON data.
		
		Input: The sample data is logged to a CSV file and then a request to read the CSV file is made.
		
		Output: The output will be of length 2. The value at the second index of the list will equal the sample data when converted from a list format to a JSON object format. 
		
		Test Case Derivation: Nutritional data of a food will be saved for 
		future reference. This is done by logging the data to the database.
		
		How test will be performed: The sample data is sent to our CSV logger function to create an entry for the data as well as its header row detailing the keys of the key value JSON pairings. To account for this header row, after we read the CSV file line by line, we check that the length of the returning list is 2.
		
	\end{enumerate}
	
	\subsubsection{Nutritional Facts Display}
	This section has three tests, with each corresponding to an action the user can take with Utrition. Each test corresponds to (FR3, FR4, FR5, FR6, FR8), (FR7, FR8), and (FR7, FR8, FR9) respectively in the SRS. These tests will test the actions the user can take after uploading an image, which are comprised of viewing past and present nutritional data in textual and graphical formats.
	
	
	
	\paragraph{Viewing the nutritional data of an uploaded food image}
	
	\begin{enumerate}
		
		\item{current-nutrition-1\\}
		
		Control: Functional, Dynamic, Unit, and Automatic
		
		Initial State: Image(s) exists in system
		
		Input: Input signal from the image upload process
		
		Output: The nutritional data of the food
		
		Test Case Derivation: Primary black-box test for our system as a whole. The system will take the uploaded image(s), analyse them for the food they display, cross-reference them with a nutrition database, and display the information to the user.
		
		How test will be performed: The program will automatically test that the resulting nutrition data of a pre-uploaded image file matches that of the nutrition database entry for the food.
		
	\end{enumerate}
	
	\paragraph{Viewing textual nutritional data of logged foods}
	
	\begin{enumerate}
		
		\item{past-nutrition-text-read-data-2\\}
		
		Control: Functional, Dynamic, Unit, and Automatic
		
		Initial State: Logged foods exist in system
		
		Input: Request to view nutritional data of logged foods as text
		
		Output: The nutritional data of past logged foods in a textual format
		
		Test Case Derivation: The system will keep a record of all analyzed foods for the user, allowing the user to request to see the nutritional data of their past uploads. This data can be displayed in a human readable textual format.
		
		How test will be performed: The program will automatically test that a file of pre-logged foods can be processed to return their corresponding nutritional data as text.
			
	\end{enumerate}

	\paragraph{Viewing webpages}

	\begin{enumerate}
		\item{display-elements-1\\}
		
		Control: Functional, Unit and Automatic.
		
		Initial State: Tester will be located on a page of the Utrition application.
		
		Input: The tester will attempt to render page to see if necessary elements exist.
		
		Output: The framework replies with if elements exist.
		
		Test Case Derivation: The testing framework should reply that the text and image does exist upon the user opening the Utrition application.
		
		How test will be performed: the tester will mock the Utrition application home page screen and will check to see if the elements exist for the user.
		
		\item{correct-path-1\\}
		
		Control: Functional, Unit and Automatic.
		
		Initial State: Tester will start on home page
		
		Input: Tester will click hyperlink to go to a different page
		
		Output: The tester will be taken to the new page specified in hyperlink
		
		Test Case Derivation: The testing framework should reply with the newly rendered page
		
		How test will be performed: The tester will attempt to mock a user input event clicking onto the specified hyperlink area by rendering a mocked version of the home page. Upon clicking the hyperlink area of the profile page the testing framework should reply with a rendered profile page.
	\end{enumerate}

	\subsubsection{Text Upload}
	This section has twelve tests, with each corresponding to an text upload action the user can take with Utrition. The tests deal with FR3, FR4, FR7, FR8, FR9, FR14, and FR15 in the SRS. Testing that users can successfully submit text inputs to identify food items is 
	important to the core functionality of the other components of the 
	application.
	
	\begin{enumerate}
	\item{text-upload-1\\}
	
	Control: Functional, Dynamic, and Manual.
	
	Initial State: User is on the ``Upload" page with ``Text Upload" selected.
	
	Input: User inputs ``cup of yogurt" into the text upload box and then clicks on the ``Submit" button
	
	Output: The nutritional facts for a cup of yogurt is displayed to the user.
	
	Test Case Derivation: The basic usage of the text upload functional requirement should work properly.
	
	How test will be performed: The tester will open Utrition and click on ``Upload" found in the taskbar. The tester will type ``cup of yogurt" into the text upload box and press the ``Submit" button.
	
	\item{text-upload-2\\}
	
	Control: Functional, Dynamic, and Manual.
	
	Initial State: User is on the ``Upload" page with ``Text Upload" selected.
	
	Input: User inputs ``i had and ate 3 pieces of shumai" into the text upload box and then clicks on the ``Submit" button
	
	Output: The nutritional facts for 3 pieces of shumai is displayed to the user.
	
	Test Case Derivation: Ethnic foods should be available for the user to input. User should be able to use complex phrases instead of just food items.
	
	How test will be performed: The tester will open Utrition and click on ``Upload" found in the taskbar. The tester will type ``i had and ate 3 pieces of shumai" into the text upload box and press the ``Submit" button.
	
	\item{text-upload-3\\}
	
	Control: Functional, Dynamic, and Manual.
	
	Initial State: User is on the ``Upload" page with ``Text Upload" selected.
	
	Input: User inputs ``avacadoe" into the text upload box and then clicks on the ``Submit" button
	
	Output: No nutritional facts will be displayed to the user.
	
	Test Case Derivation: Typos should not break the Utrition system.
	
	How test will be performed: The tester will open Utrition and click on ``Upload" found in the taskbar. The tester will type ``avacadoe" into the text upload box and press the ``Submit" button.
	
	\item{text-upload-4\\}
	
	Control: Functional, Dynamic, and Manual.
	
	Initial State: User is on the ``Upload" page with ``Text Upload" selected.
	
	Input: User inputs nothing into the text upload box and then clicks on the ``Submit" button
	
	Output: No nutritional facts will be displayed to the user.
	
	Test Case Derivation: Empty inputs should not break the Utrition system.
	
	How test will be performed: The tester will open Utrition and clicks on ``Upload" found in the taskbar. The tester will type nothing into the text upload box and will press the ``Submit" button.
	
	\item{text-upload-5\\}
	
	Control: Functional, Dynamic, and Manual.
	
	Initial State: User is on the ``Upload" page with ``Text Upload" selected.
	
	Input: User inputs ``celeree" into the text upload box and then clicks on the ``Submit" button
	
	Output: No nutritional facts will be displayed to the user.
	
	Test Case Derivation: Typos should not bring any results to the user.
	
	How test will be performed: The tester will open Utrition and clicks on ``Upload" found in the taskbar. The tester will type ``celeree" into the text upload box and will press the ``Submit" button.
	
	\item{text-upload-6\\}
	
	Control: Functional, Dynamic, and Manual.
	
	Initial State: User is on the ``Upload" page with ``Text Upload" selected.
	
	Input: User inputs ``a cup of lucky charms" into the text upload box and then clicks on the ``Submit" button
	
	Output: The nutritional facts for a cup of Lucky Charms is displayed to the user.
	
	Test Case Derivation: Popular food brands should be available to the user to input.
	
	How test will be performed: The tester will open Utrition and will click on ``Upload" found in the taskbar. The tester will type ``a cup of lucky charms" into the text upload box and will press the ``Submit" button.
	
	\item{text-upload-7\\}
	
	Control: Functional, Dynamic, and Manual.
	
	Initial State: User is on the ``Upload" page with ``Text Upload" selected.
	
	Input: User inputs ``yuhddasdihijdsahdaihdhi chicken shawarma wrap asgyudy1hu13h123hej" into the text upload box and then clicks on the ``Voice Upload" button. User clicks on the ``Text Upload" button and clicks ``Submit"
	
	Output: The nutritional facts for a chicken shawarma wrap is displayed to the user.
	
	Test Case Derivation: The system should be able to decipher food items in a flood of gibberish.
	
	How test will be performed: The tester will open Utrition and will click on ``Upload" found in the taskbar. The tester will type ``yuhddasdihijdsahdaihdhi chicken shawarma wrap asgyudy1hu13h123hej" into the text upload box and will press the ``Voice Upload" button. The tester then clicks on the ``Text Upload" button and then the ``Submit" button.
	
	\item{text-upload-8\\}
	
	Control: Functional, Dynamic, and Manual.
	
	Initial State: User is on the ``Upload" page with ``Text Upload" selected.
	
	Input: User inputs ``half of Fries Burger King" into the text upload box and then clicks on the ``Submit" button
	
	Output: The nutritional facts for half of a Burger King's fries is displayed to the user.
	
	Test Case Derivation: The user should be able to use normal text writing to declare the serving size of food items they consume.
	
	How test will be performed: The tester will open Utrition and will click on ``Upload" found in the taskbar. The tester will type ``half of Fries Burger King
	" into the text upload box and will press the ``Submit" button.
	
	\paragraph{Multi-Text Upload}
	\paragraph{Multi-Text Upload}
	
	\item{multi-text-upload-1\\}
	
	Control: Functional, Dynamic, and Manual.
	
	Initial State: User is on the ``Upload" page with ``Text Upload" selected.
	
	Input: User inputs ``apple" and ``banana" into the text upload box and then presses the ``Enter" key on their keyboard.
	
	Output: The nutritional facts for an apple and a banana is displayed to the user.
	
	Test Case Derivation: The user should be allowed to use the ``Enter" key on their keyboard to submit a response.
	
	How test will be performed: The tester will open Utrition and click on ``Upload" found in the taskbar. The tester will type ``apple banana" into the text upload box and then press enter. 
	
	\item{multi-text-upload-2\\}
	
	Control: Functional, Dynamic, and Manual.
	
	Initial State: User is on the ``Upload" page with ``Text Upload" selected.
	
	Input: User inputs ``3 kit kat bars, 100g of salt, and 2 cucumbers" into the text upload box and then clicks on the ``Submit" button.
	
	Output: The nutritional facts for 3 Kit Kat bars, 100g of salt, and 2 cucumbers is displayed to the user.
	
	Test Case Derivation: User should be allowed to input multiple food items, and be able to specify serving sizes for food items (without requiring all food items to have their serving sizes declared).
	
	How test will be performed: The tester will open Utrition and will click on ``Upload" found in the taskbar. The tester will type ``3 kit kat bars, 100g of salt, and 2 cucumbers" into the text upload box and then presses the ``Submit" button.
	
	\item{multi-text-upload-3\\}
	
	Control: Functional, Dynamic, and Manual.
	
	Initial State: User is on the ``Upload" page with ``Text Upload" selected.
	
	Input: User inputs ``i ate two pieces of chicken breast... also had 100g of a martini. . . . . . . . . 2 Cheeseburger!!!!!11" into the text upload box and then clicks on the ``Submit" button.
	
	Output: The nutritional facts for 2 pieces of chicken breast, 100g of a martini, and 2 cheeseburgers is displayed to the user.
	
	Test Case Derivation: The system should be able to detect multiple food items even when surrounded by incorrect use of grammar.
	
	How test will be performed: The tester will open Utrition and will click on ``Upload" found in the taskbar. The tester will type ``i ate two pieces of chicken breast... also had 100g of a martini. . . . . . . . . 2 Cheeseburger!!!!!11" into the text upload box and then will press the ``Submit" button.
	
	\item{multi-text-upload-4\\}
	
	Control: Functional, Dynamic, and Manual.
	
	Initial State: User is on the ``Upload" page with ``Text Upload" selected.
	
	Input: User inputs ``ubdhsjsbubahasjvdhasbhdabhsdbh asbdh ahbsdbajsdbhashbdb 1 slice of cheese b2b23b2b3b2b3b2b 2 grapefruits aaahhhhh 100g of apple sauce ajjdjj2j2j3j23j who goes there? yup its me! aaa 1 slice of white bread celery" into the text upload box and then clicks on the ``Submit" button.
	
	Output: The nutritional facts for 1 slice of cheese and 2 grapefruits, 100g of apple sauce, 1 slice of white bread, and 1 celery is displayed to the user.
	
	Test Case Derivation: The system should be able to detect multiple food items even when surrounded by gibberish and random symbols.
	
	How test will be performed: The tester opened Utrition and clicked on ``Upload" found in the taskbar. The tester typed ``ubdhsjsbubahasjvdhasbhdabhsdbh asbdh ahbsdbajsdbhashbdb 1 slice of cheese b2b23b2b3b2b3b2b 2 grapefruits aaahhhhh 100g of apple sauce ajjdjj2j2j3j23j who goes there? yup its me! aaa 1 slice of white bread celery" into the text upload box and then pressed the ``Submit" button.

	\end{enumerate}

	\subsubsection{Voice Upload}
	This section has fifteen tests, with each corresponding to an voice upload action the user can take with Utrition. The tests deal with FR3, FR4, FR7, FR10, FR11, FR12, FR13, FR14, and FR15 in the SRS. 
	Testing that users can successfully submit speech inputs is 
	important to the core functionality of the other components of the 
	application.
	
	\begin{enumerate}
	\item{voice-upload-1\\}
	
	Control: Functional, Dynamic, and Manual.
	
	Initial State: User is on the ``Upload" page with ``Voice Upload" selected.
	
	Input: User clicks on the ``Start Talking" button and says out loud ``I ate a slice of apple pie". User clicks on the ``Submit" button.
	
	Output: The nutritional facts for a slice of apple pie is displayed to the user.
	
	Test Case Derivation: The user should be able to input a normal spoken phrase to search up a food item.
	
	How test will be performed: The tester will open Utrition and clicks on ``Upload" found in the taskbar. The tester will click on the ``Voice Upload" button and then will click on the ``Start Talking" button. The tester verbally says ``I ate a slice of apple pie." and presses the ``Submit" button.
	
	\item{voice-upload-2\\}
	
	Control: Functional, Dynamic, and Manual.
	
	Initial State: User is on the ``Upload" page with ``Voice Upload" selected.
	
	Input: User clicks on the ``Start Talking" button and says out loud ``I ate two pieces of Kimchi period king yes". User clicks on the ``Stop Talking" button and then clicks on the ``Submit" Button.
	
	Output: The nutritional facts for 2 pieces of Kimchi is displayed to the user.
	
	Test Case Derivation: The voice detection system should be able to detect ethnic foods. The system should not be confused by random words that aren't food.
	
	How test will be performed: The tester will open Utrition and clicks on ``Upload" found in the taskbar. The tester will click on the ``Voice Upload" button and then clicks on the ``Start Talking" button. The tester verbally says ``I ate two pieces of Kimchi period king yes" and presses the ``Stop Talking" button. Then the tester will press the ``Submit" button.
	
	\item{voice-upload-3\\}
	
	Control: Functional, Dynamic, and Manual.
	
	Initial State: User has just completed test case voice-upload-2.
	
	Input: User clicks on the ``Start Talking` button and says out loud ``three chicken wings". User clicks on the ``Stop Talking" button and then clicks on the ``Submit" Button.
	
	Output: The nutritional facts for 2 pieces of Kimchi and 3 chicken wings is displayed to the user.
	
	Test Case Derivation: Since the user does not reset the input, the previous input should still remain in the system.
	
	How test will be performed: After the tester completes the steps laid out in test case voice-upload-2, the tester will click on the ``Start Talking" button. The tester verbally says ``three chicken wings" and presses the ``Stop Talking" button. Then the tester will press the ``Submit" button.
	
	\item{voice-upload-4\\}
	
	Control: Functional, Dynamic, and Manual.
	
	Initial State: User has just completed test case voice-upload-3.
	
	Input: User clicks on the ``Start Talking" button and then the ``Reset" button.
	
	Output: The ``Upload" page resets and the user sees no previous uploaded entries. The user is able to click the ``Stop Talking" button.
	
	Test Case Derivation: The system should allow the user to reset their input in the middle of giving an input.
	
	How test will be performed: After the tester completed the steps laid out in test case voice-upload-3, the tester then clicked on the ``Start Talking" button and then clicked on the ``Reset" button. The tester clicked on the ``Stop Talking" button again, and the button is properly updated to ``Start Talking".
	
	\item{voice-upload-5\\}
	
	Control: Functional, Dynamic, and Manual.
	
	Initial State: User is on the ``Upload" page with ``Voice Upload" selected.
	
	Input: User clicks on the ``Reset" button.
	
	Output: The ``Upload" page resets. The user is met with the normal ``Voice Upload" page.
	
	Test Case Derivation: The system should allow the user to reset their input without giving an input.
	
	How test will be performed: The tester will open Utrition and will click on ``Upload" found in the taskbar. The tester will click on the ``Voice Upload" button and then clicks on the ``Reset" button. 
	
	\item{voice-upload-6\\}
	
	Control: Functional, Dynamic, and Manual.
	
	Initial State: User is on the ``Upload" page and has typed ``apple" in the text box found in ``Text Upload".
	
	Input: User clicks on ``Voice Upload" and then User clicks on the ``Reset" button.
	
	Output: The ``Upload" page resets. The user is no longer able to see ``apple" in the text box.
	
	Test Case Derivation: The ``Reset" button should only reset the input in the ``Voice Upload" section.
	
	How test will be performed: The tester will open Utrition and click on ``Upload" found in the taskbar. The tester types ``apple" in the text box. The tester clicks on the ``Voice Upload" button and then clicks on the ``Reset" button. The tester clicks on the ``Text Upload" button and no longer sees ``apple" in the text box.
	
	\item{voice-upload-7\\}
	
	Control: Functional, Dynamic, and Manual.
	
	Initial State: User is on the ``Upload" page with ``Voice Upload" selected.
	
	Input: User clicks on the ``Submit" button.
	
	Output: No nutritional facts will be displayed to the user.
	
	Test Case Derivation: The system should not break when submitting nothing on the ``Voice Upload" section.
	
	How test will be performed: The tester will open Utrition and will click on ``Upload" found in the taskbar. The tester clicks on the ``Voice Upload" button and then clicks on the ``Submit" button.
	
	\item{voice-upload-8\\}
	
	Control: Functional, Dynamic, and Manual.
	
	Initial State: User is on the ``Upload" page with ``Voice Upload" selected.
	
	Input: User clicks on the ``Start Talking" button and says out loud ``Hey now, you're a rock star. Get the show on, get paid". User clicks on the ``Stop Talking" button and then clicks on the ``Submit" Button.
	
	Output: No nutritional facts will be displayed to the user.
	
	Test Case Derivation: No food items should be identified and displayed to the user when there is none in the input.
	
	How test will be performed: The tester will open Utrition and clicks on ``Upload" found in the taskbar. The tester clicks on the ``Voice Upload" button and then clicks on the ``Start Talking" button. The tester verbally says ``Hey now, you're a rock star. Get the show on, get paid" and presses the ``Stop Talking" button. Then the tester presses the ``Submit" button.
	
	\item{voice-upload-9\\}
	
	Control: Functional, Dynamic, and Manual.
	
	Initial State: User has just completed test case voice-upload-8.
	
	Input: User clicks on the ``Start Talking" button and says out loud ``a can of sprite". User clicks on the ``Stop Talking" button and then clicks on the ``Submit" Button.
	
	Output: The nutritional facts for a can of sprite is displayed to the user.
	
	Test Case Derivation: Without resetting and after a submitted sequence with no food items, the system should be able to identify inputted food items.
	
	How test will be performed: After the tester completes the steps laid out in test case voice-upload-9, the tester will click on the ``Start Talking" button. The tester verbally says ``a can of sprite" and presses the ``Stop Talking" button. Then the tester will press the ``Submit" button.
	
	\item{voice-upload-10\\}
	
	Control: Functional, Dynamic, and Manual.
	
	Initial State: User is on the ``Upload" page with ``Voice Upload" selected.
	
	Input: User clicks on the ``Start Talking" button and says out loud ``Ahhh! Don't scare me like that. a pound of white rice. so by the way im vegan lol". User clicks on the ``Stop Talking" button and then clicks on the ``Submit" Button.
	
	Output: The nutritional facts for a pound of white rice is displayed to the user.
	
	Test Case Derivation: A food item sneaked into a coherent sentence should be able to be identified by the system. 
	
	How test will be performed: The tester will open Utrition and will click on ``Upload" found in the taskbar. The tester clicks on the ``Voice Upload" button and then clicks on the ``Start Talking" button. The tester verbally says ``Ahhh! Don't scare me like that. a pound of white rice. so by the way im vegan lol" and presses the ``Stop Talking" button. Then the tester presses the ``Submit" button.
	
	\item{voice-upload-11\\}
	
	Control: Functional, Dynamic, and Manual.
	
	Initial State: User is on the ``Upload" page with ``Voice Upload" selected.
	
	Input: User clicks on the ``Start Talking" button and says out loud ``I ate 2 eggs benedicts". User clicks on ``Stop Talking" and then clicks on ``Text Upload". User clicks on ``Voice Upload" and then the ``Submit" button.
	
	Output: The nutritional facts for 2 eggs benedicts is displayed to the user.
	
	Test Case Derivation: Unique food items containing more than two words in a phrase should be identified by the system.
	
	How test will be performed: The tester will open Utrition and clicks on ``Upload" found in the taskbar. The tester will click on the ``Voice Upload" button and then clicks on the ``Start Talking" button. The tester verbally says ``I ate 2 eggs benedicts" and presses the ``Stop Talking" button. Then the tester presses on the ``Text Upload" button and then the ``Voice Upload" button. The tester presses the ``Submit" button.
	
	\item{voice-upload-12\\}
	
	Control: Functional, Dynamic, and Manual.
	
	Initial State: User is on the ``Upload" page with ``Voice Upload" selected.
	
	Input: User clicks on the ``Start Talking" button and says out loud ``2 oranges". User clicks on ``Text Upload". User says out loud ``a slice of an apple" User clicks on ``Voice Upload" and then the ``Submit" button.
	
	Output: No nutritional facts will be displayed to the user.
	
	Test Case Derivation: Once the user clicks away from the ``Voice Upload" section, the page should reset.
	
	How test will be performed: The tester will open Utrition and will click on ``Upload" found in the taskbar. The tester clicks on the ``Voice Upload" button and then clicks on the ``Start Talking" button. The tester verbally says ``2 oranges" and then the tester presses on the ``Text Upload" button. The tester says ``a slice of an apple" and then clicks on the ``Voice Upload" button. The tester presses the ``Submit" button.
		\paragraph{Multi-Voice Upload}
		\paragraph{Multi-Voice Upload}
	\item{multi-voice-upload-1\\}
	
	Control: Functional, Dynamic, and Manual.
	
	Initial State: User is on the ``Upload" page with ``Voice Upload" selected.
	
	Input: User clicks on the ``Start Talking" button and says out loud ``Today I ate 3 spoons of peanut butter, I know im a mess. I also had 3 whole baguettes with a side of 100 G of salsa. Then to top it all off I ate 1 carrot cake." User clicks the ``Submit" button.
	
	Output: The nutritional facts for 3 spoons of peanut butter, 3 whole baguettes, 100g of salsa, and 1 whole carrot cake is displayed to the user.
	
	Test Case Derivation: User should be able to say a complex sentence with multiple food items. User should be able to specify serving sizes of foods if they choose to do so.
	
	How test will be performed: The tester will open Utrition and will click on ``Upload" found in the taskbar. The tester clicks on the ``Voice Upload" button and then clicks on the ``Start Talking" button. The tester verbally says ``Today I ate 3 spoons of peanut butter, I know im a mess. I also had 3 whole baguettes with a side of 100 G of salsa. Then to top it all off I ate 1 carrot cake". The tester presses the ``Submit" button.
	
	\item{multi-voice-upload-2\\}
	
	Control: Functional, Dynamic, and Manual.
	
	Initial State: User is on the ``Upload" page with ``Voice Upload" selected.
	
	Input: User clicks on the ``Start Talking" button and says out loud ``100 grams of hummus, 2 bowls of mac and cheese". The user pauses for 5 seconds and then says ``2 teaspoons of vegetable oil". User clicks the ``Submit" button.
	
	Output: The nutritional facts for 100g of hummus, 2 bowls of mac and cheese, and 2 teaspoons of vegetable oil is displayed to the user.
	
	Test Case Derivation: The user should be able to list out the food items instead of speaking a normal phrase to the system. The user should be able to pause during their input.
	
	How test will be performed: The tester will open Utrition and clicks on ``Upload" found in the taskbar. The tester clicks on the ``Voice Upload" button and then clicks on the ``Start Talking" button. The tester verbally says ``100 grams of hummus, 2 bowls of mac and cheese". The tester pauses for 5 seconds and says ``2 teaspoons of vegetable oil". The tester presses the ``Submit" button.
	
	\item{multi-voice-upload-3\\}
	
	Control: Functional, Dynamic, and Manual.
	
	Initial State: User is on the ``Upload" page with ``Voice Upload" selected.
	
	Input: User clicks on the ``Start Talking" button and says out loud ``ausdhjasdihjashduih (gibberish) 2 grams of Pina Colada ausdhjasdihjashduih (gibberish) penguin 1 pineapple ausdhjasdihjashduih (gibberish)". User clicks the ``Submit" button.
	
	Output: The nutritional facts for 2g of Pina Colada and 1 whole pineapple is displayed to the user.
	
	Test Case Derivation: The system should be able to identify the multiple food items surrounded by gibberish.
	
	How test will be performed: The tester will open Utrition and clicks on ``Upload" found in the taskbar. The tester clicks on the ``Voice Upload" button and then clicks on the ``Start Talking" button. The tester verbally says ``ausdhjasdihjashduih (gibberish) 2 grams of Pina Colada ausdhjasdihjashduih (gibberish) penguin 1 pineapple ausdhjasdihjashduih (gibberish)". The tester presses the ``Submit" button.
	
\end{enumerate}
	
	\subsubsection{Profile Page}
	This section has thirteen tests, with each corresponding to the user viewing their nutritional logs and profile on Utrition. The tests deal with FR5, FR6, FR7, FR16, FR17, FR18, FR19, FR21, FR27, and FR28 in the SRS. Testing that users can successfully log and view their past inputs ensures the main functionality of the Utrition application.
	
	\begin{enumerate}
		\item{profile-1\\}
		
		Control: Functional, Dynamic, and Manual.
		
		Initial State: User has completed test case ``text-upload-1".
		
		Input: User clicks on ``Profile" found in the taskbar.
		
		Output: The user is able to see that their most logged food is yogurt, their total calories consumed for today is 142.88 (calories in 1 cup of yogurt), the nutritional values of yogurt next to the current date, and a graph depicting calories versus date.
		
		Test Case Derivation: The system should properly display the profile page with only one food item inputted into the system.
		
		How test will be performed: After the tester completes the steps laid out in test case text-upload-1, the tester will click on ``Profile" found in the taskbar.
		
		\item{profile-2\\}
		
		Control: Functional, Dynamic, and Manual.
		
		Initial State: User has completed test case ``text-upload-1" three times.
		
		Input: User clicks on ``Profile" found in the taskbar.
		
		Output: The user is able to see that their most logged food is yogurt, their total calories consumed for today is 428.64, the nutritional values of yogurt next to the current date is shown three times, the total calories of three cups of yogurt is summed on the right table, and a graph depicting calories versus date.
		
		Test Case Derivation: The system should properly display the profile page with three of the same food item inputted into the system.
		
		How test will be performed: After the tester completes the steps laid out in test case text-upload-1 three times, the tester will click on ``Profile" found in the taskbar.
		
		\item{profile-3\\}
		
		Control: Functional, Dynamic, and Manual.
		
		Initial State: User has completed test case ``profile-2".
		
		Input: User completes test case ``multi-text-upload-3" and then clicks on ``Profile" found in the taskbar.
		
		Output: The user is able to see that their most logged food is yogurt, their total calories consumed for today is 2137.12, the nutritional values of 2 chicken breasts next to the current date, the nutritional value of 100g of a martini next to the current date, the nutritional value of 2 cheeseburgers next to the current date, and the nutritional value of a cup of yogurt next to the current date (in that order). The user is able to see the total calories of 3 cups of yogurt, 2 chicken breasts, 100g of a martini, and 2 cheeseburgers is summed on the right table. The user is able to see a graph depicting calories versus date.
		
		Test Case Derivation: The system should properly display the profile page with a multitude of different food items inputted into the system.
		
		How test will be performed: After the tester completes the steps laid out in test case profile-2, the tester completes test case ``multi-text-upload-3". The tester will then click on ``Profile" found in the taskbar.
		
		\item{profile-4\\}
		
		Control: Functional, Dynamic, and Manual.
		
		Initial State: User has completed test case ``profile-3".
		
		Input: User clicks on ``Upload" found in the taskbar. User clicks on ``Voice Upload" and then ``Start Talking". User says ``2 cheeseburgers random random random how was your day?" and then clicks on ``Stop Talking". User clicks on the ``Submit" button, and clicks on ``Profile" found in the taskbar.
		
		Output: The user is able to see that their most logged food is yogurt, their total calories consumed for today is 3207.74, the nutritional value of 2 cheeseburgers next to the current date, the nutritional values of 2 chicken breasts next to the current date, the nutritional value of 100g of a martini next to the current date, and the nutritional value of 2 cheeseburgers next to the current date (in that order). The user is able to see the total calories of 3 cups of yogurt, 2 chicken breasts, 100g of a martini, and 4 cheeseburgers is summed on the right table. The user is able to see a graph depicting calories versus date.
		
		Test Case Derivation: The system should properly display the user's most logged food being yogurt, even with 2 cheeseburgers being implemented. This is because yogurt is logged more than once.
		
		How test will be performed: After the tester completes the steps laid out in test case profile-3, the tester will click on ``Upload" in the taskbar. The tester clicks on ``Voice Upload" and then ``Start Talking". Tester says ``2 cheeseburgers random random random how was your day?" and then clicks on ``Stop Talking". Tester clicks on the ``Submit" button. The tester clicks on ``Profile" found in the taskbar.
		
		\item{profile-5\\}
		
		Control: Functional, Dynamic, and Manual.
		
		Initial State: User has completed test case ``profile-4" and is viewing the ``Upload" page
		
		Input: User clicks on ``Text Upload" and then types ``cheeseburger and cheeseburger" . User clicks on the ``Submit" button, and clicks on ``Profile" found in the taskbar.
		
		Output: The user is able to see that their most logged food is yogurt, their total calories consumed for today is 4278.36, the nutritional value of 1 cheeseburger next to the current date, the nutritional values of 1 cheeseburger next to the current date, the nutritional value of 2 cheeseburgers next to the current date, and the nutritional value of 1 chicken breast next to the current date (in that order). The user is able to see the total calories of 3 cups of yogurt, 2 chicken breasts, 100g of a martini, and 6 cheeseburgers is summed on the right table. The user is able to see a graph depicting calories versus date.
		
		Test Case Derivation: The system should properly display the user's most logged food being yogurt. The system should properly display the cheeseburger's nutritional information even when given in different serving sizes and formats.
		
		How test will be performed: After the tester completes the steps laid out in test case profile-4, the tester clicks on ``Upload" found in the taskbar. The tester clicks on ``Text Upload" and then types ``cheeseburger and cheeseburger". Tester clicks on the ``Submit" button. The tester clicks on ``Profile" found in the taskbar.
		
		\item{profile-6\\}
		
		Control: Functional, Dynamic, and Manual.
		
		Initial State: User has completed test case ``profile-5" and is viewing the ``Profile" page.
		
		Input: User clicks on ``Look at next 4 entries".
		
		Output: User is able to see the nutritional information for: 100g of a martini, 2 cheeseburgers, a cup of yogurt, and another cup of yogurt all next to the current date.
		
		Test Case Derivation: The system should allow the user to look at the nutritional information of their previously entered entries.
		
		How test will be performed: After the tester completes the steps laid out in test case profile-5, the tester will click on the ``Look at next 4 entries" button.
		
		\item{profile-7\\}
		
		Control: Functional, Dynamic, and Manual.
		
		Initial State: User has completed test case ``profile-6" and is viewing the ``Profile" page.
		
		Input: User clicks on ``Look at next 4 entries".
		
		Output: User is able to see the nutritional information for 1 cup of yogurt.
		
		Test Case Derivation: The system should allow the user to look at the nutritional information of their previously entered entry.
		
		How test will be performed: After the tester completes the steps laid out in test case profile-6, the tester will click on the ``Look at next 4 entries" button.
		
		\item{profile-8\\}
		
		Control: Functional, Dynamic, and Manual.
		
		Initial State: User has completed test case ``profile-7" and is viewing the ``Profile" page.
		
		Input: User clicks on ``Look at previous 4 entries".
		
		Output: User is able to see the nutritional information for: 100g of a martini, 2 cheeseburgers, a cup of yogurt, and another cup of yogurt all next to the current date.
		
		Test Case Derivation: The system should allow the user to look at the nutritional information of their most recently entered entries.
		
		How test will be performed: After the tester completes the steps laid out in test case profile-7, the tester will click on the ``Look at previous 4 entries" button.
		
		\item{profile-9\\}
		
		Control: Functional, Dynamic, and Manual.
		
		Initial State: User has inputted a cup of yogurt into Utrition for 15 days and is now viewing the ``Profile" page
		
		Input: User clicks on the ``Previous Week" button.
		
		Output: The table is updated to show the previous 7 days in which a cup of yogurt was inputted into Utrition. 
		
		Test Case Derivation: The system should allow the user to look at the weekly summary of their previously entered entries. Even if the entries are the same.
		
		How test will be performed: The tester will go into the directory ``utrition$\backslash$src$\backslash$utrition$\backslash$utrition-backend" and open the ``nutrition\_log.csv" file. The tester will fake data for yogurt by making sure each column is filled out with appropriate data (Time having format d/m/Y H:M:S, Name must be a string, serving unit must be a string, and the rest filled with floats), and then copying and pasting the data for 14 other rows (15 in total). Each row will have a different date associated with it. The tester saves the ``nutrition\_log.csv" changes, and refreshes the ``Profile" page. The tester clicks on the ``Previous Week" button.
		
		\item{profile-10\\}
		
		Control: Functional, Dynamic, and Manual.
		
		Initial State: User has completed test case ``profile-9" and is viewing the ``Profile" page.
		
		Input: User clicks on the ``Previous Week" button.
		
		Output: The table is updated to show the first day the user inputted a food item (yogurt) into Utrition.
		
		Test Case Derivation: The system should allow the user to look at the weekly summary of their previously entered entry. Even if the previous entries are the same.
		
		How test will be performed: After the tester completes the steps laid out in test case profile-9, the tester will click on the ``Previous Week" button.
		
		\item{profile-11\\}
		
		Control: Functional, Dynamic, and Manual.
		
		Initial State: Food items were inputted into Utrition on 15 different days. In increasing order, the following food items were inputted: Apple, Banana, Celery, Orange, Grapefruit, Chicken, Cheeseburger, Pizza, Sprite, Cashew, Cucumber, Pickle, Poutine, Sour Cream, and Salsa.
		
		Input: User clicks on the ``Previous Week" button twice. Then, the user clicks on the ``Next Week" button once.
		
		Output: The table is updated to show the following food items with their attached date and calorie information: Pizza, Cheeseburger, Chicken, Grapefruit, Orange, Celery, and Banana.
		
		Test Case Derivation: The system should allow the user to look at the weekly summary of their previously entered entries. Both button functionalities should work properly.
		
		How test will be performed: The tester will go into the directory ``utrition$\backslash$src$\backslash$utrition$\backslash$utrition-backend" and will open the ``nutrition\_log.csv" file. The tester fakes data for the food items by making sure each column was filled out with appropriate data (Time having format d/m/Y H:M:S, serving unit must be a string, and the rest filled with floats), and then copying and pasting the data for 14 other rows (15 in total). Each row will have a different date associated with it. The tester adds in the ``Name" column the following 15 food items in increasing order by date: Apple, Banana, Celery, Orange, Grapefruit, Chicken, Cheeseburger, Pizza, Sprite, Cashew, Cucumber, Pickle, Poutine, Sour Cream, and Salsa. The tester saves the ``nutrition\_log.csv" changes, and refreshes the ``Profile" page. The tester clicks on the ``Previous Week" button twice, and the ``Next Week" button once.
		
		\item{profile-12\\}
		
		Control: Functional, Dynamic, and Manual.
		
		Initial State:  User has completed test case ``profile-11" but has now added an extra row of salsa having the same data as the previous salsa entry.
		
		Input: User clicks on the ``Previous Week" button twice. Then, the user clicks on the ``Next Week" button twice.
		
		Output: The table is updated to show the following food items with their attached date and calorie information: Salsa and Salsa, Sour Cream, Poutine, Pickle, Cucumber, Cashew, and Sprite.
		
		Test Case Derivation: The system should allow the user to look at the weekly summary of their previously entered entries. Even if there are multiple entries on the same day.
		
		How test will be performed: After the tester completes the steps laid out in test case profile-11, the tester will go into the directory ``utrition$\backslash$src$\backslash$utrition$\backslash$utrition-backend" and will open the ``nutrition\_log.csv" file. The tester copies and pastes the Salsa row once more at the bottom of the excel file. The tester will save the ``nutrition\_log.csv" changes, and refreshes the ``Profile" page. The tester clicks on the ``Previous Week" button twice, and the ``Next Week" button twice.
		
		\item{profile-13\\}
		
		Control: Functional, Dynamic, and Manual.
		
		Initial State: User has no entries saved on their device and is viewing the ``Profile" page.
		
		Input: User clicks on ``Profile" found in the taskbar.
		
		Output: The user is able to see the following headers with no nutritional information near them: Your most logged food, Your total calories consumed for today, PAST MEALS, and My Caloric Intake.
		
		Test Case Derivation: The system should not break if no entries are found on the user's personal device.
		
		How test will be performed: The tester will open Utrition and click on ``Profile" found in the taskbar. The tester will click on ``Profile" one more time.
		
		\item{profile-14\\}
		
		Control: Functional, Dynamic, and Manual.
		
		Initial State: User has completed test case ``profile-5"
		
		Input: User clicks on the ``Delete" button for the inputted item of ``2 cheeseburgers" and confirms the deletion for the food item.
		
		Output: The user sees the food items: ``1 cheeseburger", ``1 cheeseburger", ``2 chicken breasts", and ``100g of martini" under the header ``PAST MEALS".
		
		Test Case Derivation: User should be allowed to delete food items. After deletion of food item, the list should be properly updated.
		
		How test will be performed: After the tester completes the steps laid out in test case profile-5, the tester will click on the ``Delete" button found beside the inputted item of ``2 cheeseburgers". The tester will click on the ``Yes" button.
		
		\item{profile-15\\}
		
		Control: Functional, Dynamic, and Manual.
		
		Initial State: User has completed test case ``profile-14"
		
		Input: User clicks on the ``Delete" button for the inputted item of ``1 cheeseburger" and denies the deletion for the food item.
		
		Output: The user sees the food items: ``1 cheeseburger", ``1 cheeseburger", ``2 chicken breasts", and ``100g of martini" under the header ``PAST MEALS".
		
		Test Case Derivation: After the tester completes the steps laid out in test case profile-14, the tester will click on the ``Delete" button found beside the inputted item of ``1 cheeseburgers". The tester will click on the ``No" button.
		
		How test will be performed: The tester will open Utrition and click on ``Profile" found in the taskbar. The tester will click on ``Profile" one more time.
		
	\end{enumerate}
	
	\subsubsection{Settings Page}
	This section has four tests that deal with the settings page and the appropriate update to the profile page after receiving the user's statistics.  The tests deal with FR22, FR23, FR24, FR25, and FR26 in the SRS. Testing to ensure the user settings page works is vital in order to stand out from the other food tracking applications.
	
	\begin{enumerate}
		\item{settings-page-1\\}
		
		Control: Functional, Dynamic, and Manual.
		
		Initial State: User is viewing the ``Edit Your Statistics" page.
		
		Input: User adds the following input to the statistics page and then clicks the ``Save" button:
		
		\begin{itemize}
			\item Birth Sex: ``Male"
			\item Weight: ``80kg"
			\item Height: ``180cm"
			\item Age: ``22"
			\item Activity Level: ``Moderately Active"
		\end{itemize}
		
		Output: The user views the ``Your User Settings" page with the updated values.
		
		Test Case Derivation: User should be able to put in their user statistics to learn more about healthy eating habits.
		
		How test will be performed: Tester will click on ``Profile" found in the taskbar. Tester clicks on the ``Edit Profile Statisctics" button. Tester clicks on the ``Edit Settings" button. Tester clicks on the text input boxes and inputs the test values as seen in ``Input". Tester selects the appropriate gender and activity level as seen in ``Input". Tester clicks on ``Save".
		
		\item{settings-page-2\\}
		
		Control: Functional, Dynamic, and Manual.
		
		Initial State: User has previously completed test case ``text-upload-1" and has just completed test case ``settings-page-1".
		
		Input: User clicks on ``Profile" found in the taskbar and clicks on the ``How close am I to being classified differently?" button.
		
		Output: The user is able to see the nutritional information of yogurt on the left side of the page. On the right side of the page, the user is able to see that the most logged food is yogurt, they have logged 142.88 calories out of 2822.49 calories, their BMI is 24.69, they are normal weight, a BMI table showing the ranges for appropriate weight, a caloric intake summary chart showing yogurt, and a graph showing the appropriate date and calories.
		
		Test Case Derivation: User statistics must properly reflect in the profile page, and provide the user information that is relevant to their statistics.
		
		How test will be performed: Tester completes ``text-upload-1" and then completes ``settings-page-1". Then, the tester will click on ``Profile" found in the taskbar, and then clicks on the ``How close am I to being classified differently?" button.
		
		\item{settings-page-3\\}
		
		Control: Functional, Dynamic, and Manual.
		
		Initial State: User has completed test case ``settings-page-2"
		
		Input: User clicks on ``Edit Settings" button, adds the following input to the statistics page, and then clicks the ``Save" button:
		
		\begin{itemize}
			\item Birth Sex: ``Female"
			\item Weight: ``200lbs"
			\item Height: ``5ft 5in"
			\item Age: ``19"
			\item Activity Level: ``Sedentary"
		\end{itemize}
		
		Output: The user views the ``Your User Settings" page with the updated values.
		
		Test Case Derivation: User should be able to edit their statistics.
		
		How test will be performed: After the tester completes test case ``settings-page-2", tester will click on ``Profile" found in the taskbar. Tester clicks on the ``Edit Profile Statisctics" button. Tester clicks on the ``Edit Settings" button. Tester clicks on the text input boxes and inputs the test values as seen in ``Input". Tester selects the appropriate gender and activity level as seen in ``Input". Tester clicks on ``Save".
		
		\item{settings-page-4\\}
		
		Control: Functional, Dynamic, and Manual.
		
		Initial State: User has completed test case ``settings-page-3".
		
		Input: User clicks on ``Profile" found in the taskbar and clicks on the ``How close am I to being classified differently?" button.
		
		Output: The user is able to see the nutritional information of yogurt on the left side of the page. On the right side of the page, the user is able to see that the most logged food is yogurt, they have logged 142.88 calories out of 2020.41 calories, their BMI is 33.28, they are obese, a BMI table showing the ranges for appropriate weight, a caloric intake summary chart showing yogurt, and a graph showing the appropriate date and calories.
		
		Test Case Derivation: Modified statistics and the information regarding it should properly reflect in the profile page.
		
		How test will be performed: After tester completes ``settings-page-3", tester will click on ``Profile" found in the taskbar, and then clicks on the ``How close am I to being classified differently?" button.
		
	\end{enumerate}
	
	\subsubsection{Taskbar}
	This section has six tests, with each corresponding to the actions the user can take with the navigation taskbar in Utrition. The tests deal with FR20 in the SRS. Testing that users can successfully navigate throughout the Utrition application is important to ensure freedom of use for the application.
	
	\begin{enumerate}
		\item{upload-page-1\\}
		
		Control: Functional, Dynamic, and Manual.
		
		Initial State: User is on the ``Home" page.
		
		Input: User clicks on the ``Upload" button.
		
		Output: The user views the ``Upload" page.
		
		Test Case Derivation: User should be able to view the ``Upload" page from the ``Home" page.
		
		How test will be performed: The tester will open Utrition and will click on ``Upload" found in the taskbar.
		
		\item{profile-page-1\\}
		
		Control: Functional, Dynamic, and Manual.
		
		Initial State: User is on the ``Home" page.
		
		Input: User clicks on the ``Profile" button.
		
		Output: The user views the ``Profile" page.
		
		Test Case Derivation: User should be able to view the ``Profile" page from the ``Home" page.
		
		How test will be performed: The tester will open Utrition and will click on ``Profile" found in the taskbar.
		
		\item{home-page-1\\}
		
		Control: Functional, Dynamic, and Manual.
		
		Initial State: User is on the ``Upload" page.
		
		Input: User clicks on the ``Utrition" button.
		
		Output: The user views the ``Home" page.
		
		Test Case Derivation: User should be able to view the ``Home" page from the ``Upload" page.
		
		How test will be performed: The tester will open Utrition and will click on ``Upload" found in the taskbar. The tester will click on ``Utrition" found in the taskbar.
		
		\item{home-page-2\\}
		
		Control: Functional, Dynamic, and Manual.
		
		Initial State: User is on the ``Profile" page.
		
		Input: User clicks on the ``Utrition" button.
		
		Output: The user views the ``Home" page.
		
		Test Case Derivation: User should be able to view the ``Home" page from the ``Profile" page.
		
		How test will be performed: The tester will open Utrition and will click on ``Profile" found in the taskbar. The tester will click on ``Utrition" found in the taskbar.
		
		\item{profile-page-2\\}
		
		Control: Functional, Dynamic, and Manual.
		
		Initial State: User is on the ``Upload" page.
		
		Input: User clicks on the ``Profile" button.
		
		Output: The user views the ``Profile" page.
		
		Test Case Derivation: User should be able to view the ``Profile" page from the ``Upload" page.
		
		How test will be performed: The tester will open Utrition and will click on ``Upload" found in the taskbar. The tester will click on ``Profile" found in the taskbar.
		
		\item{upload-page-2\\}
		
		Control: Functional, Dynamic, and Manual.
		
		Initial State: User is on the ``Profile" page.
		
		Input: User clicks on the ``Upload" button.
		
		Output: The user views the ``Upload" page.
		
		Test Case Derivation: User should be able to view the ``Upload" page from the ``Profile" page.
		
		How test will be performed: The tester will open Utrition and will click on ``Profile" found in the taskbar. The tester will click on ``Upload" found in the taskbar.
		
	\end{enumerate}
	
	\subsection{Traceability Between Test Cases and Functional Requirements}
	
	
		\begin{table}[H]
		\caption {Traceability between Tests and Functional Requirements}
		\begin{tabular}{|p{3.5cm}|p{6.5cm}|p{4.5cm}|}
			
			\hline
			\textbf{Requirement \#} & \textbf{Description}                                                                                                                                                                  & \textbf{Test ID(s)}                                                                                          \\ \hline
			FR1                     & The user must have the ability to upload a 
			digital image of a standard image type to the 
			system.                                                                     
			&
			\begin{tabular}[c]{@{}l@{}}one-upload-1\\
				one-upload-2\\
				one-upload-auto-1\\ 
				one-upload-auto-2\end{tabular}                                                  
			\\                                                    \hline
			FR2                     & The system will be able to identify the type of 
			food that is captured in an 
			image.                                                                      
			&
			\begin{tabular}[c]{@{}l@{}}one-upload-1\\
				one-upload-2\\
				image-identification-1\\ 
				image-identification-2\\ 
				image-identification-3\\current-nutrition-1\end{tabular}                   
			
			\\ 
			\hline
			FR3                     & The system will be able to make an API call to an 
			external database of nutrition facts for a variety of foods, including 
			their macro-nutrients, micro-nutrients, and caloric details. & 
			\begin{tabular}[c]{@{}l@{}}
				api-1\\ 
				current-nutrition-1\\
				axios-call-1\end{tabular}                       
			\\ \hline
		\end{tabular}
	\end{table}
	
	\begin{table}[H]
		\begin{tabular}{|p{3.5cm}|p{6.5cm}|p{4.5cm}|}
			
			\hline
			\textbf{Requirement \#} & \textbf{Description}                                                                                                                                                                  & \textbf{Test ID(s)}                                                                                                                               
			\\ \hline
			FR4                     & The system will be able to retrieve the nutrition 
			facts for a specific 
			food.                                                                       
			&
			\begin{tabular}[c]{@{}l@{}}one-upload-1\\
				api-1\\ 
				log-data-read-data-1\\ 
				current-nutrition-1\\
				text-upload-1\\
				text-upload-2\\
				text-upload-6\\
				text-upload-7\\
				text-upload-8\\
				multi-text-upload-1\\
				multi-text-upload-2\\
				multi-text-upload-3\\
				voice-upload-1\\
				voice-upload-2\\
				voice-upload-3\\
				voice-upload-9\\
				voice-upload-10\\
				voice-upload-11\\
				voice-upload-12\\
				multi-voice-upload-1\\
				multi-voice-upload-2\\
				multi-voice-upload-3\\
				profile-4\\
				profile-5\end{tabular}                                        
			\\
			\hline
			
		\end{tabular}
	\end{table}
	
	\begin{table}[H]
		\begin{tabular}{|p{3.5cm}|p{6.5cm}|p{4.5cm}|}
			
			\hline
			\textbf{Requirement \#} & \textbf{Description}                                                                                                                                                                  & \textbf{Test ID(s)}                                                                                                                               
			\\ \hline
			FR5                     & The system will log the nutritional data of a 
			food.                                                                       
			&  
			\begin{tabular}[c]{@{}l@{}}api-1\\ 
				log-data-read-data-1\\ 
				current-nutrition-1\\
				profile-1\\
				profile-2\\
				profile-3\\
				profile-4\\
				profile-5\end{tabular}                                        
			\\ \hline
			FR6                     & The user will be able to access the nutritional 
			data of previously logged 
			foods.                                                                      
			&
			\begin{tabular}[c]{@{}l@{}}log-data-read-data-1\\
				past-nutrition-read-data-2\\
				profile-1\\
				profile-2\\
				profile-3\\
				profile-4\\
				profile-5\\
				profile-6\\
				profile-7\\
				profile-8\\
				profile-9\\
				profile-10\\
				profile-11\\
				profile-12\\
				profile-13\end{tabular}                       \\ \hline
			
		\end{tabular}
	\end{table}
	
	\begin{table}[H]
		\begin{tabular}{|p{3.5cm}|p{6.5cm}|p{4.5cm}|}
			
			\hline
			\textbf{Requirement \#} & \textbf{Description}                                                                                                                                                                  & \textbf{Test ID(s)}                                                                                                                               
			\\ \hline
			
			FR7                     & The system will display the nutritional information of a food to the user.                                                                                                            & \begin{tabular}[c]{@{}l@{}}current-nutrition-1\\ past-nutrition-read-data-2\\
				text-upload-1\\
				text-upload-2\\
				text-upload-6\\
				text-upload-7\\
				text-upload-8\\
				multi-text-upload-1\\
				multi-text-upload-2\\
				multi-text-upload-3\\
				voice-upload-1\\
				voice-upload-2\\
				voice-upload-3\\
				voice-upload-9\\
				voice-upload-10\\
				voice-upload-11\\
				voice-upload-12\\
				multi-voice-upload-1\\
				multi-voice-upload-2\\
				multi-voice-upload-3\\
				display-elements-1\\
				axios-call-1\end{tabular} \\ \hline
			
			FR8                     & The user must have the ability to upload the text of a food input to the system.                                               & \begin{tabular}[c]{@{}l@{}}text-upload-1\\
				text-upload-2\\
				text-upload-3\\
				text-upload-4\\
				text-upload-5\\
				text-upload-6\\
				text-upload-7\\
				text-upload-8\end{tabular} \\ \hline
			
		\end{tabular}
	\end{table}
	
	\begin{table}[H]
		\begin{tabular}{|p{3.5cm}|p{6.5cm}|p{4.5cm}|}
			
			\hline
			\textbf{Requirement \#} & \textbf{Description}                                                                                                                                                                  & \textbf{Test ID(s)}                                                                                          \\ \hline
			FR9                     & The user must have the ability to upload multiple texts of food inputs to the system.
			& \begin{tabular}[c]{@{}l@{}}multi-text-upload-1\\
				multi-text-upload-2\\
				multi-text-upload-3\\
				profile-5\end{tabular} \\ \hline
			
			FR10                     & The user must have the ability to upload a food input by speaking to the system.
			& \begin{tabular}[c]{@{}l@{}}voice-upload-1\\
				voice-upload-2\\
				voice-upload-7\\
				voice-upload-8\\
				voice-upload-9\\
				voice-upload-10\\
				voice-upload-11\\
				voice-upload-12\\
				profile-4\end{tabular} \\ \hline
			
			FR11                     & The user must have the ability to upload multiple food inputs by speaking to the system.                                                                                                            & \begin{tabular}[c]{@{}l@{}}voice-upload-3\\
				multi-voice-upload-1\\
				multi-voice-upload-2\\
				multi-voice-upload-3\end{tabular} \\ \hline
			
			FR12                     & The system will be able to convert vocal input into text.                                                                                                            & \begin{tabular}[c]{@{}l@{}}voice-upload-1\\
				voice-upload-2\\
				voice-upload-3\\
				voice-upload-8\\
				voice-upload-9\\
				voice-upload-10\\
				voice-upload-11\\
				voice-upload-12\\
				multi-voice-upload-1\\
				multi-voice-upload-2\\
				multi-voice-upload-3\\
				profile-4\end{tabular} \\ \hline
			
		\end{tabular}
	\end{table}
	
	\begin{table}[H]
		\begin{tabular}{|p{3.5cm}|p{6.5cm}|p{4.5cm}|}
			
			\hline
			\textbf{Requirement \#} & \textbf{Description}                                                                                                                                                                  & \textbf{Test ID(s)}                                                                                          \\ \hline
			
			FR13                     & The user must have the ability to restart their voice upload to the system.                                                                                                           & \begin{tabular}[c]{@{}l@{}}voice-upload-4\\
				voice-upload-5\\
				voice-upload-6\end{tabular} \\ \hline
			
			FR14                     & The system will be able to identify the type of food(s) that is captured in a complex sentence.                                                                                                            & \begin{tabular}[c]{@{}l@{}}text-upload-2\\
				text-upload-7\\
				text-upload-8\\
				multi-text-upload-3\\
				voice-upload-1\\
				voice-upload-2\\
				voice-upload-3\\
				voice-upload-10\\
				voice-upload-11\\
				multi-voice-upload-1\\
				multi-voice-upload-2\\
				multi-voice-upload-3\\
				profile-4\end{tabular} \\ \hline
			
			FR15                     & The user must have the ability to select between uploading food(s) through image, text, or voice.                                                                                                    & \begin{tabular}[c]{@{}l@{}}voice-upload-12\\
				correct-path-1\end{tabular} \\ \hline
			
			FR16                     & The system will display the user's most logged food.                                                                                                            & \begin{tabular}[c]{@{}l@{}}profile-1\\
				profile-2\\
				profile-3\\
				profile-4\\
				profile-5\\
				profile-13\\
				display-elements-1\\
				axios-call-1\end{tabular} \\ \hline
			
		\end{tabular}
	\end{table}
	
	\begin{table}[H]
		\begin{tabular}{|p{3.5cm}|p{6.5cm}|p{4.5cm}|}
			
			\hline
			\textbf{Requirement \#} & \textbf{Description}                                                                                                                                                                  & \textbf{Test ID(s)}                                                                                          \\ \hline
			FR17                     & The system will display the user's total calories consumed for the day.                                                                                                           & \begin{tabular}[c]{@{}l@{}}profile-1\\
				profile-2\\
				profile-3\\
				profile-4\\
				profile-5\\
				profile-13\\
				display-elements-1\\
				axios-call-1\end{tabular} \\ \hline
			
			FR18                     & The system will display a summary of the foods consumed and their total calories for each of the past 7 days.                                                                                                            & \begin{tabular}[c]{@{}l@{}}profile-1\\
				profile-2\\
				profile-3\\
				profile-4\\
				profile-5\\
				profile-9\\
				profile-10\\
				profile-11\\
				profile-12\\
				profile-13\\
				display-elements-1\\
				axios-call-1\end{tabular} \\ \hline
			
			FR19                     & The user will be able to access the summary of the foods consumed and their total calories for all days with a food inputted.                                                                                                            & \begin{tabular}[c]{@{}l@{}}profile-1\\
				profile-2\\
				profile-3\\
				profile-4\\
				profile-5\\
				profile-9\\
				profile-10\\
				profile-11\\
				profile-12\\
				profile-13\end{tabular} \\ \hline
			
		\end{tabular}
	\end{table}
	
	\begin{table}[H]
		\begin{tabular}{|p{3.5cm}|p{6.5cm}|p{4.5cm}|}
			
			\hline
			\textbf{Requirement \#} & \textbf{Description}                                                                                                                                                                  & \textbf{Test ID(s)}                                                                                          \\ \hline
			FR20                     & The user will be able to access the Home page, the Upload page, and the Profile page on every page.                                                                                                            & \begin{tabular}[c]{@{}l@{}}upload-page-1\\ profile-page-1\\
				home-page-1\\
				home-page-2\\
				profile-page-2\\
				upload-page-2\\
				correct-path-1\end{tabular} \\ \hline
			
			FR21                     & The user will be able to delete a previously logged food.                                                                                                            & \begin{tabular}[c]{@{}l@{}}
				profile-14\end{tabular} \\ \hline
			
			FR22                     & The user will be able to input profile statistics related to their health and measurements.                                                                                                            & \begin{tabular}[c]{@{}l@{}}settings-page-1\\
				settings-page-3\end{tabular} \\ \hline
			
			FR23                     &The system shall log the user profile data.                                                                                                            & \begin{tabular}[c]{@{}l@{}}settings-page-1\\
				settings-page-3\end{tabular} \\ \hline
			
			FR24                     &The user will be able to modify the profile statistics.                                                                                                            & \begin{tabular}[c]{@{}l@{}}
				settings-page-3\end{tabular} \\ \hline
			
			FR25                     &The system shall be able to perform BMI related calculations                                                                                                            & \begin{tabular}[c]{@{}l@{}}
				settings-page-2\\
				settings-page-4\end{tabular} \\ \hline
			
			FR26                     &The system shall be able to display BMI related conclusions based on the user profile
			data.                                                                                                          & \begin{tabular}[c]{@{}l@{}}
				settings-page-2\\
				settings-page-4\end{tabular} \\ \hline
			
			FR27                     &The system shall be able to display a graph of user caloric trends.                                                                                                            & \begin{tabular}[c]{@{}l@{}}
				profile-1\\
				profile-2\\
				profile-3\\
				profile-4\\
				profile-5\\
				settings-page-2\\
				settings-page-4\end{tabular} \\ \hline
			
			FR28                     &The system shall be verify the user's request to delete a logged food item.                                                                                                            & \begin{tabular}[c]{@{}l@{}}
				profile-14\\
				profile-15\end{tabular} \\ \hline
		\end{tabular}
	\end{table}
	
	\subsection{Tests for Nonfunctional Requirements}
	
	\wss{The nonfunctional requirements for accuracy will likely just reference 
	the
		appropriate functional tests from above.  The test cases should mention
		reporting the relative error for these tests.  Not all projects will
		necessarily have nonfunctional requirements related to accuracy}
	
	\wss{Tests related to usability could include conducting a usability test 
	and
		survey.  The survey will be in the Appendix.}
	
	\wss{Static tests, review, inspections, and walkthroughs, will not follow 
	the
		format for the tests given below.}

	
	\subsubsection{Look and Feel Testing}
	
	%\paragraph{Title for Test}
	
	\begin{enumerate}
		\item{LF-1} 
		
		Type: Functional, Dynamic, and Manual.
		
		Initial State: A developer from Durum Wheat Semolina has launched the Utrition app.
		
		Input/Condition: The developer measures the distance between the displayed user interface components.
		
		Output/Result: The distance between user interface components must surpass 20 pixels.
		
		How test will be performed: Each developer launches the Utrition app and opens the \href{https://www.rapidtables.com/web/tools/pixel-ruler.html}{Pixel Measuring Tool}. The developer will use the tool by viewing a page on Utrition, then clicking “Print Screen'' on their keyboard. The developer then pastes the screenshot into the tool. The developer clicks the two spots he would like to measure distance from.
		
	\end{enumerate}
	
	\subsubsection{Usability and Humanity Testing}
	
	%\paragraph{Title for Test}
	
	\begin{enumerate}
		
		\item{UH-1}
		
		Type: Functional, Dynamic, and Manual.
		
		Initial State: The user is viewing a page of the user interface.
		
		Input/Condition: The user clicks on the “Menu” button located at the bottom of the user interface.
		
		Output/Result: The user will be brought to the main menu screen.
		
		How test will be performed: A developer of Durum Wheat Semolina will access every page listed in “Initial State” and will attempt to reach the main menu in 2 or less clicks from the respective page.
		
		
		\item{UH-2}
		
		Type: Functional, Dynamic, and Manual.
		
		Initial State: The user has previously entered 5 items in Utrition and is now viewing the main menu.
		
		Input: The user makes the request to view past nutritional data.
		
		Output: The user views a list displaying all 5 food items with their respective details in: food name, calories, fat, sodium, proteins, carbohydrates, sugars, and date entered.
		
		How test will be performed: A developer of Durum Wheat Semolina will open Utrition on their personal device and access the main menu. The developer will upload an image of different food items 5 times, with a minimum 1 minute time difference between each input. The uploaded images of the food items will be randomly selected by the developer in the utrition/test/testPhotos directory. The developer navigates to the past inputted foods area and views the list displaying the 5 different food items and their respective details.
		
		\item{UH-3}
		
		Type: Functional, Dynamic, and Manual.
		
		Initial State: The user has previously entered 5 items in Utrition and has requested to view their past inputted food items.
		
		Input/Condition: The user navigates to the screen where they can view their past nutritional data in a chart format.
		
		Output/Result: The user views a chart displaying all 5 food items with their respective details in: food name, calories, proteins, carbohydrates, sugars, and date entered.
		
		How test will be performed: A developer of Durum Wheat Semolina will open Utrition on their personal device and access the main menu. The developer will upload an image of different food items 5 times, with a minimum 1 minute time difference between each input. The uploaded images of the food items will be randomly selected by the developer in the utrition/test/testPhotos directory. The developer navigates to the past inputted foods area where they can view their statistics as a chart. The developer views a chart displaying 5 different food items and their respective details.
		
		\item{UH-4}
		
		Type: Functional, Dynamic, and Manual.
		
		Initial State: Utrition’s main menu is opened on the user’s personal device.
		
		Input/Condition: The user uploads a photo of food.
		
		Output/Result: The user views the food’s nutritional information.
		
		How test will be performed: Each developer of Durum Wheat Semolina will load Utrition on to their personal device. Each developer will ask 2 people aged 14 or older to upload and retrieve the nutritional data of the food item “pineapple.jpg” found in the utrition/test/testPhotos directory. The developer will track the time it takes for each test. 90\% of the panel must complete the task in under 2 minutes.
		
		\item{UH-5}
		
		Type: Functional, Dynamic, and Manual.
		
		Initial State: The user loads into Utrition’s main menu.
		
		Input/Condition: The user views every page of the user interface.
		
		Output/Result: The user will see the respective symbol associated with every mention of calories, fat, sodium carbohydrates, sugar, and protein.
		
		How test will be performed: On each screen, including the main menu, a developer on Durum Wheat Semolina will check if there is a symbol associated with every mention of calories, fat, sodium carbohydrates, sugar, and protein. The developer will open Utrition on their personal device. The developer will upload a .png image of food found in the utrition/test/testPhotos directory. The developer will view the nutritional information venture through each page to confirm the correct symbols appear.
		
		\item{UH-6}
		
		Type: Functional, Dynamic, and Manual.
		
		Initial State: The user has previously entered 2 items in Utrition and is now viewing the main menu.
		
		Input/Condition: The user views every page of the user interface.
		
		Output/Result: The user does not see any backend calculations.
		
		How test will be performed: On each screen, a developer on Durum Wheat Semolina will check if any backend calculations are displayed to the user. The developer will open Utrition on their personal device and access the main menu. The developer will upload an image of two different food items, with a minimum of 1 minute between the inputs. The uploaded images of the food items will be randomly selected by the developer in the utrition/test/testPhotos directory. The developer will view all pages to ensure no backend calculations are visible.
		
		\item{UH-7}
		
		Type: Functional, Dynamic, and Manual.
		
		Initial State: The user opens Utrition with a stable internet connection.
		
		Input/Condition: The user views every page of the user interface.
		
		Output/Result: 
		\begin{itemize}
			\item Utrition is available to be used by users at all times.
			\item The user does not hear Utrition generate any sound.
		\end{itemize}
		
		How test will be performed: On each screen, a developer on Durum Wheat Semolina will check if a sound plays. The developer will open Utrition on their personal device. The developer will upload an image of food. The uploaded image of the food item will be randomly selected by the developer in the utrition/test/testPhotos directory. The developer will view the nutritional information. The developer then will view the past inputs and the nutritional information chart.
		
		\item{UH-8}
		
		Type: Static and Manual.
		
		Initial State: A developer on Durum Wheat Semolina views Utrition’s GitHub page. 
		
		Input/Condition: The developer checks every directory for an audio file.
		
		Output/Result: The developer removes all audio files found in Utrition’s GitHub.
		
		How test will be performed: The developer loads into Utrition’s \href{https://github.com/jeff-rey-wang/utrition/}{GitHub} and views the files in every directory. For each directory the developer will check for the following file types:
		
		\begin{itemize}
			\item MP3
			\item AAC
			\item Ogg Vorbis
			\item FLAC
			\item ALAC
			\item WAV
			\item AIFF
			\item DSD
			\item PCM
		\end{itemize}
	\end{enumerate}

	Semi-structured interviewing will also be conducted to further receive insights and feedback regarding Utrition's overall implementation.
	
	\subsubsection{Performance Testing}
	
	%\paragraph{Title for Test}
	
	\begin{enumerate}
		
		\item{PT-1} 
		
		Type: Functional, Dynamic, and Manual.
		
		Initial State: The user has previously entered 10 items in Utrition.
		
		Input/Condition: The user views every page of the user interface.
		
		Output/Result: The user will be able to access every page of the user interface in 2 seconds or less.
		
		How test will be performed: A developer on Durum Wheat Semolina will measure the time it takes for each new screen to load using a stopwatch. The developer will open Utrition on their personal device. The developer will upload an image of different food items 10 times, with a minimum 1 minute time difference between the inputs. The uploaded images of the food items will be randomly selected by the developer in the utrition/test/testPhotos directory. The developer will navigate to each screen, timing each transition.
		
		\item{PT-2}
		
		Type: Functional, Dynamic, Automatic
		
		Initial State: Nutritional data of a food item is contained in the 
		system.
		
		Input/Condition: The user requests to display the nutritional information of a 
		food item.
		
		Output/Result: The system will generate the HTML body to be displayed on 
		the user 
		interface.
		
		How test will be performed: The test will measure if this interaction 
		was completed in 5 seconds or less.
		
		\item{PT-3}
		
		Type: Functional, Dynamic, and Automatic.
		
		Initial State: System contains a set of images; all with foods, and of 
		extension .jpg, .png, or .jpeg.
		
		Input/Condition: System prompted to process the image, for each image 
		within the provided set (images found in utrition/test/testPhotos directory).
		
		Output/Result: For each image, the system should identify the food 
		present in an image and return the name of the food item as a string.
		
		How test will be performed: The test will check if among all the 
		provided images, more than 90\% of the images have been assessed 
		correctly.
		
		\item{PT-4}
		
		Type: Functional, Dynamic, and Automatic.
		
		Initial State: System contains a set of images; some with and others 
		without foods, and of extension .jpg, .png, or 
		.jpeg.
		
		Input/Condition: System prompted to process image, for each image 
		within the provided set (images found in utrition/test/testPhotos directory).
		
		Output/Result: The system should identify the food present in an image, 
		and return the name of the food item as a string, and prompt the user 
		if no food is detected.
		
		How test will be performed: The test will check if among all the 
		provided images, more than 70\% of the images have been assessed 
		correctly.
		
		\item{PT-5} 
		
		Type: Static and Manual.
		
		Initial State: A developer on Durum Wheat Semolina opens their internet browser.
		
		Input/Condition: The developer is able to access Utrition’s  \href{https://github.com/jeff-rey-wang/utrition}{GitHub} and download the repository on to their personal device.
		
		Output/Result: Utrition is able to be downloaded.
		
		How test will be performed: The developer will access Utrition’s GitHub on their personal device’s web browser. The developer will click on “Code”, and then “Download ZIP”. The developer will verify that Utrition has been downloaded on to their device.
		
	\end{enumerate}
	
	\subsubsection{Operational and Environmental Testing}
	
	%\paragraph{Title for Test}
	
	\begin{enumerate}
		
		\item{OE-1} 
		
		Type: Static and Manual.
		
		Initial State: A user aged 14 or older has read Utrition’s “README.md” located at the bottom of the \href{https://github.com/jeff-rey-wang/utrition/}{GitHub} page.
		
		Input/Condition: The user installs Utrition on their personal device.
		
		Output/Result: The intended demographic is able to install Utrition on their computer.
		
		How test will be performed: Each developer on Durum Wheat Semolina will test 5 individuals aged 14 and above to install Utrition. The test passes if 95\% of users can install Utrition.
		
		\item{OE-2} 
		
		Type: Static and Manual.
		
		Initial State: A user visits Utrition’s \href{https://github.com/jeff-rey-wang/utrition/}{GitHub} page.
		
		Input/Condition: The user checks that the latest change to Utrition’s GitHub is not after April 30th, 2022.
		
		Output/Result: Utrition will not have any updates after the final release.
		
		How test will be performed: A developer on Durum Wheat Semolina will visit Utrition’s GitHub page bi-yearly to ensure no updates are made to Utrition’s GitHub.
		
	\end{enumerate}
	
	\subsubsection{Maintainability and Support Testing}
	
	%\paragraph{Title for Test}
	
	\begin{enumerate}
		
		\item{MS-1} 
		
		Type: Functional, Dynamic, and Manual.
		
		Initial State: The developer is in the utrition/src/database directory.
		
		Input/Condition: The developer switches the database file found in the database directory for another database.
		
		Output/Result: Backend developers are able to upload new data to Utrition’s database.
		
		How test will be performed: The developer travels to the utrition/src/database directory on their device. The developer replaces the database found in the directory with “testdatabase.jpg”, which can be found in the utrition/test/testPhotos directory. The developer launches Utrition on their device to see that the system still runs.	
		
		\item{MS-2} 
		
		Type: Functional, Dynamic, and Manual.
		
		Initial State: The user has installed Utrition on to their Windows, macOS, or Linux device.
		
		Input/Condition: The user is able to upload an image of food and view all pages of the user interface.
		
		Output/Result: All of Utrition’s functionality can be accessed by Windows, macOS, and Linux devices.
		
		How test will be performed: Developers of Durum Wheat Semolina will install Utrition on one of each Windows, macOS, and Linux device. The developer will open Utrition on the device and upload a random image of a food item found in the utrition/test/testPhotos directory. The developer sees their foods’ nutritional information. The developer can view past inputs' nutritional data and chart.
		
	\end{enumerate}
	
	\subsubsection{Security Testing}
	
	%\paragraph{Title for Test}
	
	\begin{enumerate}
		
		\item{ST-1} 
		
		Type: Functional, Dynamic, and Manual.
		
		Initial State: The user is viewing the nutritional facts for a food item they uploaded.
		
		Input/Condition: After idling for 3 minutes, the user closes Utrition and reopens it.
		
		Output/Result: The user will open Utrition and will view the nutritional facts of the image in the past inputs section.
		
		How test will be performed: A developer on Durum Wheat Semolina will open Utrition and will upload “waffle.jpg” found in the utrition/test/testPhotos directory. The developer views the nutritional facts of a waffle, and then idles for 3 minutes. After the elapsed time, the developer closes their instance of Utrition and opens it again. The developer can view the nutritional facts of the waffle in the past inputs section.
		
		\item{ST-2} 
		
		Type: Static and Manual.
		
		Initial State: A Durum Wheat Semolina developer visits Utrition’s \href{https://github.com/jeff-rey-wang/utrition/}{GitHub} page.
		
		Input/Condition: The developer checks the userdata directory for any files.
		
		Output/Result: The userdata folder will be empty on GitHub, so Utrition will only contain personalized data once downloaded on a user's device.
		
		How test will be performed: The developer loads into Utrition’s \href{https://github.com/jeff-rey-wang/utrition/}{GitHub} and goes into the utrition/src/userdata directory. The developer ensures that no userdata is preloaded into Utrition’s GitHub.
		
		
	\end{enumerate}
	
	\subsubsection{Cultural Testing}
	
	%\paragraph{Title for Test}
	
	\begin{enumerate}
		\item{CT-1} 
		
		Type: Functional, Dynamic, and Manual.
		
		Initial State: Utrition’s main menu is being displayed to a user.
		
		Input/Condition: The user is told by a Durum Wheat Semolina developer to find culturally insensitive material.
		
		Output/Result: The user will be questioned after 5 minutes of searching if they have found any culturally insensitive material while using Utrition.
		
		How test will be performed: Each developer will approach 5 peers with Utrition’s main menu open on the developer’s device. The developer will give full control over to their peers and will tell their peers to search for culturally insensitive material. After 5 minutes are up, the developer will ask their peers if they have found any culturally insensitive material. This test passes if 95\% of users could not find any culturally insensitive material.
		
	\end{enumerate}
	
	\subsubsection{Legal Testing}
	
	%\paragraph{Title for Test}
	
	\begin{enumerate}
		\item{LT-1}
		
		Type: Static and Manual.
		
		Initial State: A developer of Durum Wheat Semolina views Utrition’s \href{https://github.com/jeff-rey-wang/utrition/}{GitHub} repository.
		
		Input/Condition: The developer inspects all code to ensure Utrition is following Canada’s Anti-Spam Legislation Requirements for Installing Computer Programs, Canada’s Guide for Publishing Open Source Code, Google’s HTML/CSS Style Guide, and Google’s guide for material design.
		
		Output/Result: Utrition is in compliance with all the standards laid out by the SRS document.
		
		How test will be performed: The developer goes into the utrition/src directory, and then opens the link to one of the following guides:
		\begin{itemize}
			\item \href{https://crtc.gc.ca/eng/internet/install.htm}{Canada’s Anti-Spam Legislation Requirements for Installing Computer Programs}
			\item \href{https://www.canada.ca/en/government/system/digital-government/digital-government-innovations/open-source-software/guide-for-publishing-open-source-code.html}{Guide for Publishing Open Source Code}
			\item \href{https://google.github.io/styleguide/jsguide.html}{Google JavaScript Style Guide}
			\item \href{https://google.github.io/styleguide/htmlcssguide.html}{Google HTML/CSS Style Guide}
			\item \href{https://material.io/design}{Material Design Guide}
		\end{itemize}
		The developer will go into each module found in the utrition/src directory, and will inspect the code to ensure that each rule outlined in Canada’s or Google’s guide is strictly adhered to.
		
		\item{LT-2}
		
		Type: Functional, Dynamic, and Manual.
		
		Initial State: A Durum Wheat Semolina developer has completed a module for Utrition, and has Pylint installed on their device.
		
		Input/Condition: The developer runs Pylint on their code.
		
		Output/Result: Pylint will notify the developer if there needs to be changes to the code to ensure Utrition adheres to the PEP8 Python coding conventions.
		
		How test will be performed: The developer will open their command prompt and travel to the directory with the finished code (utrition/src) by using “cd”. The developer runs Pylint on the finished code by typing “pylint finishedmodule.py” in their command prompt. The name “finishedmodule.py” is a placeholder for any future modules that Utrition will implement in the future. The command prompt will notify the developer to make any necessary changes.
		
		
	\end{enumerate}
	
	\subsection{Traceability Between Test Cases and Non-Functional Requirements}
	
	\wss{Provide a table that shows which test cases are supporting which
		requirements.}
	
	\begin{table}[H]
		\caption{Traceability between System Tests and Non-Functional Requirements}
		\begin{tabularx}{\linewidth}{|l|X|l|}
			\hline
			{\bf Requirement \#} & {\bf Description of Fit Criterion} & {\bf Test ID(s)}\\
			\hline
			LF1 & The distance between user interface components must exceed 20 pixels. & LF-1 \\
			\hline
			UH1 & The user will be able to navigate to the main menu from any page in 2 clicks. & UH-1 \\
			\hline
			UH2 & The user must be able to view every past inputted food item. & UH-2, UH-3 \\
			\hline
			UH3 & 90\% of a test panel of the target demographic should be able to upload their food and retrieve the corresponding nutritional facts in under 2 minutes. & UH-4 \\
			\hline
			UH4 & Every clickable object will have a symbol associated with it. Calories, fat, sodium, carbohydrates, sugar, and protein will have a symbol associated next to it.
			& UH-5 \\
			\hline
			UH5 & All backend calculations, such as identifying the uploaded food and retrieving the nutritional information, will not be displayed to the user. & UH-6 \\
			\hline
			UH6 & There will be no sound used by the Utrition web application.  & UH-7, UH-8 \\
			\hline
			PR1 & The time between the user’s click and displaying the new UI will not exceed 2 seconds. & PT-1 \\
			\hline
		\end{tabularx}
		\label{tab:Non-Functional Traceability}
	\end{table}
	
	\begin{table}[H]
		%identification-accuracy-2
		\begin{tabularx}{\textwidth}{|l|X|l|}
			\hline
			{\bf Requirement \#} & {\bf Description of Fit Criterion} & {\bf Test ID(s)}\\
			\hline
			PR2 & In a test with multiple images, Utrition will identify and 
			display the correct food with an average accuracy of 70\%. & 
			\begin{tabular}[c]{@{}l@{}}PT-4 \end{tabular}\\
			\hline
			PR3 & In a test with multiple images, Utrition will display the 
			correct nutritional information for an identified food 90\% of the 
			time. & \begin{tabular}[c]{@{}l@{}}PT-3 \end{tabular}\\
			\hline
			PR4 & Utrition is a local web app and does not depend on a server. & PT-5\\
			\hline
			PR5 & Utrition will be available to use on GitHub for an indefinite amount of time. & PT-5\\
			\hline
			OE1 & Users will be able to find and pull the repository on to their personal device. & PT-5\\
			\hline
			OE2 & 95\% of a test panel with the intended demographic will be able to install Utrition after reading the installation guide. & OE-1\\
			\hline
			OE3 & Utrition’s GitHub repository will receive no updates after final implementation. & OE-2\\
			\hline
			MS1 & Any backend developer is able to successfully upload new data to Utrition’s machine learning model at any time during development.  & MS-1\\
			\hline
		\end{tabularx}
	\end{table}
	
	\begin{table}[H]
		\begin{tabularx}{\textwidth}{|l|X|l|}
			\hline
			{\bf Requirement \#} & {\bf Description of Fit Criterion} & {\bf Test ID(s)}\\
			\hline
			MS2 & All users will be able to view the support manuals on how to install and use Utrition. & OE-1\\
			\hline
			MS3 & 100\% of the tested Windows, macOS, and Linux devices should be able to use all functionality provided for Utrition. & MS-2\\
			\hline
			SR1 & Anyone will be able to pull the open-source project from GitHub. & PT-8, OE-1 \\
			\hline
			SR2 & Utrition will save user data every 3 minutes throughout the 
			application runtime. & ST-1\\
			\hline
			SR3 & Utrition will not contain personalized data from one user on another user's system. & ST-2\\
			\hline
			CP1 & 95\% of the test panel agrees that Utrition does not provide any culturally insensitive content to the user. & CT-1\\
			\hline
			LR1 & No malicious software will be present in Utrition’s GitHub repository. & LT-1\\
			\hline
			LR2 & Durum Wheat Semolina will not commit infractions when following Canada’s guide. & LT-1\\
			\hline
			LR3 & Pylint will be used throughout development to ensure every pull request contains properly formatted code. & LT-2\\
			\hline
			LR4 & Before every pull request, code will be reviewed to ensure no violations of the standards occur.  & LT-1\\
			\hline
			LR5 & Utrition's user interface will be reviewed to ensure all fonts and colours adhere to the style guide.  & LT-1\\
			\hline
		\end{tabularx}
	\end{table}
	
	\section{Unit Test Description}
	
	\wss{Reference your MIS (detailed design document) and explain your overall
		philosophy for test case selection.}  
	\wss{This section should not be filled in until after the MIS (detailed 
	design
		document) has been completed.}
	
	\subsection{Unit Testing Scope}
	
	\wss{What modules are outside of the scope.  If there are modules that are
		developed by someone else, then you would say here if you aren't 
		planning on
		verifying them.  There may also be modules that are part of your 
		software, but
		have a lower priority for verification than others.  If this is the 
		case,
		explain your rationale for the ranking of module importance.}
		
		We have decided to test anything that falls under the displaying of nutritional data and the response of the API as these are of a high priority then other modules in our software. Additionally for this project, the majority of testing for functional and nonfunctional requirements was conducted through manual testing. The developers conducted manual tests to measure how the system reacted to possible scenarios and interactions. Only modules of great importance have been selected to have additional unit tests. 
	
	\subsection{Tests for Functional Requirements}
	
	\wss{Subsets of the tests may be in related, so this section is divided into
		different areas.  If there are no identifiable subsets for the tests, this
		level of document structure can be removed.}
		
		The automated unit testing of specific core functions in the backend of the application can be seen in the Functional Requirements Evaluation of Nutritional Facts Display (3.4), and Call and Fetch API Response (3.5).
	
	\wss{Include a blurb here to explain why the subsections below
		cover the requirements.  References to the SRS would be good.}

	
	
	\wss{It would be nice to have a blurb here to explain why the subsections below
		cover the requirements.  References to the SRS would be good.  If a section
		covers tests for input constraints, you should reference the data constraints
		table in the SRS.}
	

	
	\subsection{Tests for Nonfunctional Requirements}
	
	\wss{If there is a module that needs to be independently assessed for
		performance, those test cases can go here.  In some projects, planning 
		for
		nonfunctional tests of units will not be that relevant.}
	
	\wss{These tests may involve collecting performance data from previously
		mentioned functional tests.}
		
		There are no written unit tests for nonfunctional requirements as all testing done for the functional requirements have been done through different methods of testing. 
	
	\subsection{Traceability Between Test Cases and Modules}
	
	
	\newpage
	\begin{table}[H]
	
	\begin{tabular}{|p{3.5cm}|p{6.5cm}|p{4.5cm}|}

	\hline
	\textbf{Module \#} & \textbf{Name and Description}                                                                                                                                                                  & \textbf{Test ID(s)}                                                                                          \\ \hline
	M16                     & Logging Data -  tests convertion from a list format to a JSON                                                                     
	                                   &
	 \begin{tabular}[c]{@{}l@{}}log-data-read-data-1\\ 
		\end{tabular}                                        
		                                                    \\
		 \hline
	M14                     & Nutritional Facts Display - tests the whole process of classifying an image and retrieving JSON                                                                       
	                                                            &  
	\begin{tabular}[c]{@{}l@{}}current-nutrition-1\\ 
		\end{tabular}                                        
		                                               \\ \hline
	M16                     & Nutritional Facts Display - sample data is logged to a CSV file                                                                      
	                                &
	 \begin{tabular}[c]{@{}l@{}}past-nutrition-text-read-data-2\end{tabular}                       \\ \hline
	M2 M3 M4 M11                     & Nutritional Facts Display - check for successful rendering of elements on page                                                                                                            & \begin{tabular}[c]{@{}l@{}}display-elements-1\end{tabular} \\ \hline
	M11                     & Nutritional Facts Display - attempt to mock a user input event clicking onto the specified hyperlink area                                                                                                            & \begin{tabular}[c]{@{}l@{}}correct-path-1\end{tabular} \\ \hline
	\end{tabular}
	\caption {Traceability between Unit Tests and Modules}
	\end{table}

	
	
	\wss{Provide evidence that all of the modules have been considered.}
	
	\bibliographystyle{plainnat}
	
	\bibliography{../../refs/References}
	
	\newpage
	
	\section{Appendix}
	
	\subsection{Symbolic Parameters}
	
	N/A
	
	\subsection{Usability Survey Questions}
	
	\wss{This is a section that would be appropriate for some projects.}
	The following questions will be asked to participants during usability testing.
	\begin{itemize}
		\item What are your expectations (ex. features) for the application? 
		\item How was your experience performing a typical use case? What did you struggle with? 
		\item What did you like about the application? 
		\item What should be improved upon? 
		\item What additional features do you think will be useful to have in the application? 
	\end{itemize}
	
	\newpage{}
	\section*{Appendix --- Reflection}
	
	\subsection*{Skills}
	
	\subsubsection*{Usage of Pytest framework}
	Utrition's testing process will include unit tests, integration tests, and functional tests as a part of the verification and validation process. Pytest is a framework that is used for writing simple and scalable test cases for databases, APIs, and user interfaces. This framework can be used for anything from simple unit tests to complex functional tests. To master the use of Pytest, a team member can study the Pytest documentation, reference guides, and refer to the Pytest support community posts when necessary. Following tutorials and how-to guides can also prove to be helpful when learning how to use this framework.  
	
	\subsubsection*{CI/CD with automatic test cases in GitHub}
	
	During the development phase, new code should be continuously tested in 
	order to mitigate the introduction of bugs. GitHub Actions is an effective 
	tool that can be leveraged in order to automate testing every time new 
	code is added to the repository. By automating test cases, unit tests can 
	be executed whenever new pull requests are open to ensure new additions 
	have not broken any existing functionality. To implement this, team members 
	can consult online articles on the use of these CI/CD tools, or watch a 
	video walk-through on how to add automated testing to an existing GitHub 
	project.
	
	\subsubsection*{Usage of Jest framework} 
	Developers of Durum Wheat Semolina must be able to test the frontend development of Utrition which is made using JavaScript, HTML, and CSS. Jest is a testing framework that can write tests specifically for JavaScript, HTML, and CSS that can be run from the command line The team has decided to use this framework to ensure the correctness and usability of the code for Utrition. This tool will be able to automatically catch errors in the code so that developers do not need to manually go over the code themselves. To master using the Jest framework, team members can watch online tutorials or read Jest supporting documentation.
	
	\subsubsection*{Knowledge of and performing effective static code inspections}
	This skill describes the ability to understand the necessary components for a static code inspection and be able to perform one effectively. These inspections are done for all new code attempting to merge into the main branch of the team's GitHub through a PR. When static code inspections are not done properly, sloppy code will enter the main project which can cause errors as implementation progresses. Some approaches for learning what should be included in a code inspection and how to perform said inspection is by reading documentation on what should be done during a code inspection and asking more experienced team members of Utrition for guidance.
	
	\subsubsection*{Installing and using Pylint}
	When coding in Python, we do not have a visible compiler to catch syntax errors and bugs prior to running our program as we might have in Java or Haskell. Pylint is required for catching errors like this, as well as for ensuring consistent and syntactically correct code that follows established coding practices. Pylint can be installed through a \textbf{pip install pylint} command in the terminal, and can be used through the command \textbf{pylint filename.py} to evaluate a file. Pylint can be set up to ignore or only use certain coding rules in its extensive coding practice ruleset, making it customizable to our own specific uses. This skill can be developed by reading the documentation of Pylint and its syntax rules to see which ones are relevant to Utrition. Practicing coding with Pylint can also help acclimatize a team member and help develop good habits for using Pylint before pushing changes.
	
	\subsubsection*{Knowledge of Google Style Guides for JavaScript, HTML, and CSS}
	Frontend developers of Durum Wheat Semolina need to understand and implement the Google style guides for frontend technologies in order for all Utrition code to be consistent. By following these style guides, developers do not need to take extra time attempting to decipher unorganized code. Once the developers understand the Google style guides, they will be able to do static code reviews of the written modules. Therefore, they will now be equipped to complete the LT-1 test for the validation and verification of the project. One method for learning the style guides is by approaching the task like a school test. The person learning should take notes on the guide, and to quiz themselves at random intervals. Another way of acquiring the knowledge of the style guides is to use online video resources to help teach them the material.
	
	\subsection*{Teammate Specific Learning Approaches}
	\subsubsection*{Alexander Moica}
	Alexander would like to write code with greater consistency in its syntax and that follows established coding standards, even those he does not know about. To help him develop these skills and habits, he will incorporate Pylint into his existing projects and create personal coding projects that make use of Pylint from the very start. Alexander will learn to master using Pylint through practice since he learns fastest by making mistakes and remembering them in the future.
	
	\subsubsection*{Yasmine Jolly}
	Yasmine has had no previous experience working with the Jest testing framework and requires this knowledge for the testing of Utrition. She will be watching video tutorials on how to use and write effective test cases in Jest. This method of learning was chosen because Yasmine finds it effective to learn by watching examples. With videos, she can pause and rewind steps that she did not fully understand.
	
	\subsubsection*{Jeffrey Wang}
	
	Jeffrey has sparsely worked with automated testing frameworks during his 
	previous work experience, and seen the ways it has helped to prevent the 
	introduction of bugs. As a result, he is interested in leveraging CI/CD 
	tools to automate test cases, and set up mitigations to prevent the 
	introduction of bugs during the development of Utrition. In order to 
	complete this successfully, Jeffrey plans to read online tutorials about 
	the potential ways to automate test cases using GitHub Actions. He has chosen this strategy because the tutorials will walk him through the process step-by-step which he finds most effective for learning.
	
	\subsubsection*{Jack Theriault}
	Jack has chosen to develop his knowledge in the Google style guides. Jack has made this choice because he has never used any frontend coding technologies such as JavaScript, HTML, or CSS. Since he is using these languages for the first time, it is best that he develops proper coding style while working in the new languages. To become familiar with the style guides, Jack will read through the guides and quiz himself at regular intervals. He has chosen this learning strategy because he usually learns this way during school and has found it effective.
	
	\subsubsection*{Catherine Chen}
	
	Catherine has had limited experience when writing Python tests using Pytest. She'd like to develop her scalable test writing skills, which include unit tests and functional tests. Her goal is to become familiar with the Pytest framework in order to achieve her goals. Catherine has chosen to read Pytest documentation to learn this skill because of how easily she can access supporting documentation is online.
	
	\subsubsection*{Justina Srebrnjak}
	Justina has identified that she struggles with performing static code inspections on other team members' code. More specifically, she is confused about what should be inspected during these reviews (ex. code functionality, potential errors, coding conventions). To master this skill, Justina will seek guidance from fellow teammates who have performed many code inspections in the past (from co-op experiences). To learn most effectively, she will watch one of her teammates perform a static code inspection of another team member's work. She has chosen this strategy because she finds it helpful to watch an example be completed where she can take notes on important steps in the process. With these notes, she will replicate and refer to them when she performs her own reviews.
	
\end{document}
